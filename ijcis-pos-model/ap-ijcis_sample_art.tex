%% International Journal of Computational Intelligence Systems ---
%%%%%%%%%%%%%%%%%%%%%%%%%%%%%%%%%%%%%%%%%%%%%%%%%%%%%%%%%%%%%%%%%%%%%%%%%%%
\documentclass[11pt,twoside]{article}
\usepackage{ijcis}
%--------------------- ADDITIONAL PACKAGES HERE ---------------------------
\usepackage{amsmath}
\usepackage{amssymb}
\usepackage{color}

\hyphenation{pro-per-ties}
\hyphenation{ge-ne-ral-ly}
\hyphenation{pre-fe-ren-ces}
\hyphenation{u-sing}
\hyphenation{pu-nish-ment}
\newcommand{\Pow}{\mathcal{P}}
\newcommand{\N}{\operatorname{N}}
\newcommand{\bool}{\operatorname*{\mathcal{B}}}
\newcommand{\Pos}{\operatorname{Pos}}
\newcommand{\Nec}{\operatorname{Nec}}
\newcommand{\T}{\mathcal{T}}
\newcommand{\C}{\mathcal{C}}
\newcommand{\Nat}{\mathbb{N}}
\newcommand{\Q}{\mathbb{Q}}
\newcommand{\R}{\mathbb{R}}
\newcommand{\Z}{\mathbb{Z}}
%--------------------------------------------------------------------------
%
\def\labart{yourLabel}      % put a label from your choice here
%\Vol{1}                    % number of the Volume
%\Issue{1}                  % number of the issue
%\Month{January}            % month
%\Year{2008}                % year
%\received{...}
%\revised{...}
%
%---------------------------------------------------------------------------
\thispagestyle{empty}
%%------------------------- YOUR HEADINGS HERE -----------------------------
% Author's initials of first names+last names
\shortauthors{J. Pons}
% Short title
\shorttitle{A possibilistic general valid-time model}
%---------------------------------------------------------------------------
%
%---------------------- YOUR TITLE ----------------------------------------
\title{%
PoVaTiM: A Possibilistic Valid-Time Model
 \\ \vspace*{0.04truein}
%Using \LaTeX\footnote{For the title, try not to use more than 3 lines.
%Typeset the title in 12 pt Times Roman, uppercase and boldface.}
}
%-------------------------- AUTHOR'S NAMES ----------------------------------
\author{%
Jose Enrique Pons\,\up{1}\,\,,
%\author{
Second Author\,\up{2}
}
%----------------------------- ADDRESSES ----------------------------------
\addresses{%
\up{1}
Department of Computer Science and Artificial Intelligence, University of Granada\\
C/ Periodista Daniel Saucedo Aranda, S/N,E-18071\\
Granada, Spain%
\footnote{
Jos\'e Enrique Pons Frías, Universidad de Granada, C/ Periodista Daniel Saucedo Aranda S/N, E-18071, Granada, Spain.
%State completely without abbreviations, the affiliation
%and mailing address, including country. Typeset in 10~pt Times Italic.
}
%\\ \vspace*{0.04truein}
%E-mail: jpons@decsai.ugr.es
%%}\address{%
%\\ \vspace*{0.05truein}
%\up{2}
%Group, Company,\\
%Address,\\ City, State ZIP/Zone, Country
%\\ \vspace*{0.04truein}
%E-mail: address
}
%---------------------------------------------------------------------------
\pagestyle{myheadings}
\begin{document}
\label{\labart-FirstPage}

\maketitle
%-------------------------- ABSTRACT ---------------------------------------
\abstracts{%
%The abstract should summarize the context, content and conclusions of the
%paper in less than 80 words. It should not contain any references or
%displayed equations. Typeset the abstract in 9~pt Times Roman with
%baselineskip of 11 pt, making an indentation of 2.5 picas on the left
%and right margins.
}
\par\bigskip\par
%-------------------------- KEYWORDS ---------------------------------------
\keywords{Possibilistic fuzzy temporal database valid time}

\vspace*{10pt}\textlineskip
%-------------------------- BEGIN BODY OF TEXT -----------------------------
\begin{multicols}{2}

\section{Introduction}


\section{Preliminaries}


\section{Possibilistic Valid-Time Model for Relational DB}
%%%%%%%%%%%%%%%%%%%%%%%%%%%%%%%%
%
% Temporal model.tex
%
%%%%%%%%%%%%%%%%%%%%%%%%%%%%%%%



%
%\vglue13pt
%%\begin{table}[htbp]
%\tcap{Valid-Time interval data types}
%\centerline{\small VALID TIME DATA TYPES}
%\vglue-6pt
%\centerline{\small\baselineskip=13pt
%\begin{tabular}{c p{2cm} p{2cm}}\\
%\hline
%Type for S & Type for E & Semantics \\
%\hline
%Unknown & Unkown & Unknown interval. \\
%\hline
%1,2 & 1,2 & A possibilistic time interval. \\
%\hline\\
%\end{tabular}
%\label{tbl:valid-time-interval}}
%%\end{table}
%
%The following is a study of the constraints in the range of values for both starting and ending times to represent a valid time interval.
%
%The time period inside a valid time interval is denoted by two constraints:
%\begin{eqnarray}
%C_1\stackrel{\triangle}{=}\left(\geq,X\right)\\
%C_2\stackrel{\triangle}{=}\left(\leq,Y\right).
%\end{eqnarray}
%Applying the inference of uncertainty as proposed in our general reasoning, we find for the first constraint that:
%\begin{eqnarray}
%\Pos\left(C_1([a,b])\right) &=& \min_{r\in[a,b]}\Big(\sup_{r\leq w}\pi_{X}(w)\Big)\\
%\Nec\left(C_1([a,b])\right) &=& \min_{r\in[a,b]}\Big(\inf_{r>w}1-\pi_{X}(w)\Big)
%\end{eqnarray}
%which can be simplified to:
%\begin{eqnarray}
%\Pos\left(C_1([a,b])\right) &=& \sup_{a\leq w}\pi_{X}(w)\\
%\Nec\left(C_1([a,b])\right) &=& \inf_{a>w}1-\pi_{X}(w).
%\end{eqnarray}
%For the second constraint, we find that:
%\begin{eqnarray}
%\Pos\left(C_2([a,b])\right) &=& \min_{r\in[a,b]}\Big(\sup_{r\geq w}\pi_{Y}(w)\Big)\\
%\Nec\left(C_2([a,b])\right) &=& \min_{r\in[a,b]}\Big(\inf_{r<w}1-\pi_{Y}(w)\Big)
%\end{eqnarray}
%which can be simplified to:
%\begin{eqnarray}
%\Pos\left(C_2([a,b])\right) &=& \sup_{b\geq w}\pi_{Y}(w)\\
%\Nec\left(C_2([a,b])\right) &=& \inf_{b<w}1-\pi_{Y}(w).
%\end{eqnarray}
%The uncertainty about the inclusion of an interval $I=[a,b]$ in the interval with ill-known boundaries can now be found by evaluating $[a,b]$ against the evaluation function $\lambda$ with $\bool=\wedge$. Application of \eqref{eq:conjunctive1} and \eqref{eq:conjunctive2} leads to:
%\begin{eqnarray}
%\Pos\left(\lambda([a,b])\right)&=&\min\bigg(\Pos(C_1([a,b])),\Pos(C_2([a,b]))\bigg)\\
%\Nec\left(\lambda([a,b])\right)&=&\min\bigg(\Nec(C_1([a,b])),\Nec(C_2([a,b]))\bigg).
%\end{eqnarray}
%These last expressions can also be expanded as:
%\begin{eqnarray}
%\label{eq:interval-pos}
%\Pos\left(\lambda([a,b])\right)&=&\min\bigg(\sup_{a\leq w}\pi_{X}(w),\sup_{b\geq w}\pi_{Y}(w)\bigg)\\
%\label{eq:interval-nec}
%\Nec\left(\lambda([a,b])\right)&=&\min\bigg(\inf_{a>w}1-\pi_{X}(w),\inf_{b<w}1-\pi_{Y}(w)\bigg).
%\end{eqnarray}
%Note that the interval $[X,Y]$ used here, is certainly not a fuzzy interval. Instead, we are dealing with an ill-known interval, i.e. it is a crisp interval, but it is partially unknown which values are in this interval. The uncertainty stems from the fact that the interval boundaries are ill-known.
%
%Within this framework for the reasoning of the knowledge about the interval, we are going to study the semantics from the point of view of the valid time as well as from the point of view of the temporal database.
%
%\subsubsection{S = Unknown, E = Unknown}
%Consider that the possibility distribution for S is Unknown and for E is also Unknown. The possibility distributions are $\pi_S$ and $\pi_E$ respectively:
%
%\begin{eqnarray}
%\pi_{S}(t \in \T) = 1\\
%\pi_{E}(t \in \T) = 1
%\end{eqnarray}
%
%Therefore the constraints $C_1$ and $C_2$ are:
%
%\begin{eqnarray}
%\Pos\left(C_1([a,b])\right) &=& \sup_{a\leq w}\pi_{S}(w) = 1\\
%\Nec\left(C_1([a,b])\right) &=& \inf_{a>w}1-\pi_{S}(w) = 0\\
%\Pos\left(C_2([a,b])\right) &=& \sup_{b\geq w}\pi_{E}(w)=1\\
%\Nec\left(C_2([a,b])\right) &=& \inf_{b<w}1-\pi_{E}(w)=0
%\end{eqnarray}
%
%
%\begin{eqnarray}
%\Pos\left(\lambda([a,b])\right)&=&\min\bigg(\sup_{a\leq w}\pi_{S}(w),\sup_{b\geq w}\pi_{E}(w)\bigg) \\
%&=& \min(1,1)  = 1\\
%\Nec\left(\lambda([a,b])\right)&=&\min\bigg(\inf_{a>w}1-\pi_{S}(w),\inf_{b<w}1-\pi_{E}(w)\bigg)\\
%&=& \min(0,0) = 0
%\end{eqnarray}
%
%\begin{example}
%A constant object in the database. It is an object that is always valid in the database.
%The main issue in a valid time database is that the update sentence have to modify the starting or ending point distribution. The most natural way to convert a snapshot database to a temporal database is to consider the objects belonging to the interval [Unknown, Unknown].
%\end{example}
%
%\subsubsection{[Unknown,value] or [value,Unknown]}
%These two cases are managed in this section. Here we will note $\pi_{S}(t)$ or $\pi_{E}(t)$ as the possibility distribution for 'value' in the starting or ending points.
%
%First of all the case [Unknown,value] in \eqref{eq:interval-pos} and in \eqref{eq:interval-nec}:
%
%\begin{eqnarray}
%\Pos\left(\lambda([a,b])\right)&=&\min (1,\sup_{b\geq w}\pi_{E}(w))\\
% &=& \sup_{b\geq w}\pi_{E}(w)\\
%\Nec\left(\lambda([a,b])\right)&=&\min (0,\inf_{b<w}1-\pi_{E}(w))\\
%&=& 0
%\end{eqnarray}
%
%Similarily for [value,Unknown] the possibility and neccesity are:
%\begin{eqnarray}
%\Pos\left(\lambda([a,b])\right)&=&\min (\sup_{a\leq w}\pi_{S}(w),1)\\
% &=& \sup_{a\leq w}\pi_{S}(w)\\
%\Nec\left(\lambda([a,b])\right)&=&\min (\inf_{a>w}1-\pi_{S}(w),0)\\
%&=& 0
%\end{eqnarray}
%
%The semantics for an object with valid time given by [Unknown,value] is an object that has been valid from the beggining to an ending time. Analogously, the semantics for an object with a valid time given by the interval [value,Unknown] is an object that it is still valid in the database. 
%
%
%\subsubsection{[Undefined,Undefined]}
%Maybe this combination has no sense for a valid time object. The possibility distributions for both starting and ending points are:
%
%\begin{eqnarray}
%\pi_{S}(t \in \T) = 0\\
%\pi_{E}(t \in \T) = 0
%\end{eqnarray}
%
%Therefore the constraints $C_1$ and $C_2$ are:
%
%\begin{eqnarray}
%\Pos\left(C_1([a,b])\right) &=& \sup_{a\leq w}\pi_{S}(w) = 0\\
%\Nec\left(C_1([a,b])\right) &=& \inf_{a>w}1-\pi_{S}(w) = 1\\
%\Pos\left(C_2([a,b])\right) &=& \sup_{b\geq w}\pi_{E}(w)=0\\
%\Nec\left(C_2([a,b])\right) &=& \inf_{b<w}1-\pi_{E}(w)=1
%\end{eqnarray}
%
%Taking these values into  \eqref{eq:interval-pos} and in \eqref{eq:interval-nec}:
%\begin{eqnarray}
%\Pos\left(\lambda([a,b])\right)&=&\min (0,0)\\
% &=& 0\\
%\Nec\left(\lambda([a,b])\right)&=&\min (1,1)\\
%&=& 1
%\end{eqnarray}
%
%\textcolor{red}{TAKE A LOOK INTO THIS}
%
%Which is an inconsistency.
%
%\subsection{[Undefined,value] or [value,Undefined]}
%
%The possibility and neccesity measures for [Undefined,value] are :
%\begin{eqnarray}
%\Pos\left(\lambda([a,b])\right)&=&\min (0,\sup_{b\geq w}\pi_{E}(w))\\
% &=& 0\\
%\Nec\left(\lambda([a,b])\right)&=&\min (1,\inf_{b<w}1-\pi_{E}(w))\\
%&=& \inf_{b<w}1-\pi_{E}(w)
%\end{eqnarray}
%
%The other way around, [value,Undefined]:
%\begin{eqnarray}
%\Pos\left(\lambda([a,b])\right)&=&\min (\sup_{a\leq w}\pi_{S}(w),0)\\
% &=& 0\\
%\Nec\left(\lambda([a,b])\right)&=&\min (\inf_{a>w}1-\pi_{S}(w),1)\\
%&=& \inf_{a>w}1-\pi_{S}(w)
%\end{eqnarray}
%
%\textcolor{red}{TAKE A LOOK INTO THIS}
%This is also an inconsistency.
%
%\subsubsection{[NULL,NULL] and maybe any combination with NULL}
%I think the framework we have can not deal with NULL computations. Maybe just for non-specified time or something like that.
%
%\subsubsection{[Undefined,Unknown] or [Unknown,Undefined]}
%The semantic for this could be also unconsistent.
%
%
%\begin{example}
%Consider a snapshot database (table \ref{tbl:snapshot-car-models}) that stores information about car models. The primary key is given by the set $\lbrace$ Brand, Model $\rbrace$.
%
%\vglue13pt
%%\begin{table}[htbp]
%\tcap{Car model snapshot database}
%\centerline{\small CAR MODELS}
%\vglue-6pt
%\centerline{\small\baselineskip=13pt
%\begin{tabular}{c c c}\\
%\hline
%\textbf{Brand} & \textbf{Model} & Segment \\
%\hline
%Peugeot & 205 & B \\
%Peugeot & 206 & B \\
%Peugeot & 207 & B \\
%Peugeto & 208 & B \\
%Citroen & C2 & B \\
%\hline\\
%\end{tabular}
%\label{tbl:snapshot-car-models}
%}
%
%Consider now that the user wants to make that table a valid-time table, then, from the point of view of the storage, the following changes should be done:
%
%\begin{itemize}
%\item
%A new definition for the primary key, including a version field.
%\item
%The addition of two new columns to store the valid-time interval. By default the initial values are Unknown for both starting and ending points of the interval.
%\end{itemize}
%Therefore, the table \ref{tbl:snapshot-car-models} is transformed into table \ref{tbl:initial-valid-time-car-models}:
%
% \vglue13pt
%%\begin{table}[htbp]
%\tcap{Car model valid-time database. Note that U is the abbreviation for the Unknown constant.}
%\centerline{\small VALID TIME CAR MODELS}
%\vglue-6pt
%\centerline{\small\baselineskip=13pt
%\begin{tabular}{c c c c c c}\\
%\hline
%\textbf{Brand} & \textbf{Model} & \textbf{VersionID} & Segment & S & E \\
%\hline
%Peugeot & 205 & 001 & B & U & U \\
%Peugeot & 206 & 001 & B & U & U \\
%Peugeot & 207 & 001 & B & U & U \\
%Peugeto & 208 & 001 & B & U & U \\
%Citroen & C2 & 001 & B & U & U \\
%\hline\\
%\end{tabular}
%\label{tbl:initial-valid-time-car-models}
%}
%
%
%
%\end{example}


\subsection{\label{subsec:temporal-model}The generalized temporal model}

\begin{definition}
Generalized fuzzy temporal domain.\\
If $\T$ is the temporal domain, $\tilde \Pow\left( \T\right)$ is the set of all possibility distributions defined on $\T$.
The \textbf{Generalized Fuzzy Temporal Domain}, $\T_G$ is
\begin{equation}
\T_G \subseteq \left \lbrace \tilde \Pow\left( \T\right) \cup NULL \right \rbrace
\end{equation}
\end{definition}



%\subsection{Sub-headings}
%%\vspace*{-4pt}  %only when needed
%Sub-headings should be typeset in bolditalic with the first
%letter of first word capitalized and section number in boldface.
%
%%\vspace*{-1pt}  %only when needed
%\subsubsection{Sub-subheadings}
%%\vspace*{-4pt}  %only when needed
%
%Typeset in italic (Section No. to be in Roman) and capitalize
%the first letter of the first word only.
%
%%\vspace*{-1pt}  %only when needed
%\subsection{Numbering and spacing}
%%\vspace*{-4pt}  %only when needed
%
%Sections, sub-sections and sub-subsections are numbered in Arabic.
%Use double spacing after major and subheadings, and single spacing
%after\break sub-subheadings.
%
%%\vspace*{-1pt}  %only when needed
%\subsection{Lists of items}
%%\vspace*{-4pt}  %only when needed
%
%{Lists may be laid out with each item marked by\hfilneg}
%
%\noindent a dot:
%\begin{itemize}
%\item item one,
%\item item two.
%\end{itemize}
%
%\setcounter{footnote}{0}
%\renewcommand{\thefootnote}{\alph{footnote}}
%
%Items may also be numbered in lowercase Roman numerals:
%\begin{enumerate}[(i)]\Nospacing
%\item item one
%\item item two
%    \begin{enumerate}[(a)]\Nospacing
%    \item Lists within lists can be numbered with lowercase Roman letters,
%    \item second item.
%    \end{enumerate}
%\end{enumerate}
%
%\subsection{Running Heads}
%
%Please provide a shortened running head (not more than four words,
%each starting with a Capital) for the title of your paper. This will
%appear with page numbers on the top right-hand side of your paper on
%odd pages.
%
%For the running heads for the authors names should appear on your paper
%on the even  pages. Please apply the following rules for theses running
%heads:
%\begin{itemize}
%\item for one author: only the initial plus the full last name (e.g., D. Ruan),
%\item for two authors: D. Ruan, T. Li,
%\item for three authors or more: D. Ruan \emph{et al.}
%\end{itemize}
%
%\section{Equations}
%
%Displayed equations should be numbered consecutively in each
%section, with the number set flush right and enclosed in
%parentheses.
%\begin{equation}
%\mu(n,t) = \dfrac{\sum^\infty_{i=1} 1(d_i < t, N(d_i) = n)}
%{\int^t_{\sigma=0} 1(N(\sigma) = n)\mathrm{d}\sigma}\,\, .
%\label{this}
%\end{equation}
%Equations should be referred to in abbreviated form,
%e.g.~``Eq.~(\ref{this})'' or ``(2)''. In multiple-line equations,
%the number should be given on the last line.
%
%Displayed equations are to be centered on the page width.
%Standard English letters like x are to appear as $x$ (italicized)
%in the text if they are used as mathematical symbols. Punctuation
%marks are used at the end of equations as if they appeared directly in the text.
%
%\begin{theorem}
%Theorems, lemmas, etc. are to be numbered consecutively in the
%paper, just by using \\
%\verb+\begin{env}My text here...\end{env}+\\
%for the environments \emph{theorem}, \emph{lemma},
%\emph{proposition}, and \emph{corollary}.\\
%\end{theorem}
%
%\vspace*{-5pt}
%
%\begin{proof}
%Proofs are produced with the command\\
%\verb+\begin{proof}{My proof...}\end{proof}+.\\
%It should end with\ $\qed$\kern0.3pt.
%\end{proof}
%
%\section{Illustrations and Photographs}
%
%Figures are to be inserted in the text nearest their first reference.
%Original india ink drawings or glossy prints are preferred. Please
%send one set of originals with copies. If the author requires the
%publisher to reduce the figures, ensure that {the figures (including
%letterings and numbers) are large enough to be\hfilneg}
%
%%\begin{figure}[htbp] %ORIGINAL SIZE: width=1.4TRUEIN; height=1.5TRUEIN
%%\vspace*{13pt}
%%\centerline{\psfig{file=ap-ijcis.eps}} %100 percent
%\vspace*{10pt}
%\fcaption{Labeled tree {\it T}.}
%%\end{figure}
%\vspace*{13pt}
%
%\noindent
%clearly seen after reduction. If photographs are to be used, only
%black and white ones are acceptable.
%
%Figures are to be sequentially numbered in Arabic numerals. The
%caption must be placed below the figure. For those figures with
%multiple parts which appear on different pages, it is best to
%place the full caption below the first part, and have
%e.g.~``Fig.~1. ({\it Continued})'' below the last part. Typeset
%in 9~pt Times Roman with baselineskip of 11~pt. Use double
%spacing between a caption and the text that follows immediately.
%
%Previously published material must be accompanied by written
%permission from the author and publisher.
%
%\section{Tables}
%
%Tables should be inserted in the text as close to the point of
%reference as possible. Some space should be left above and below
%the table. Tables should be numbered sequentially in the text in
%Arabic numerals. Captions are to be centralized above the
%tables. Typeset tables and captions in 9~pt Times Roman with
%baselineskip of 11~pt.
%
%If tables need to extend over to a second page, the continuation
%of the table should be preceded by a caption, e.g.~``Table~2.
%({\it Continued})''
%
%\vglue13pt
%%\begin{table}[htbp]
%\tcap{Number of tests for WFF triple NA = 5, or\break NA = 8.}
%\centerline{\small NP}
%\vglue-6pt
%\centerline{\small\baselineskip=13pt
%\begin{tabular}{l c c c c c}\\
%\hline
%{} &{} &3 &4 &8 &10\\
%\hline
%{} &\phantom03 &1200 &2000 &\phantom02500 &\phantom03000\\
%NC &\phantom05 &2000 &2200 &\phantom02700 &\phantom03400\\
%{} &\phantom08 &2500 &2700 &16000 &22000\\
%{} &10 &3000 &3400 &22000 &28000\\
%\hline\\
%\end{tabular}}
%%\end{table}
%
%\section{References}
%
%References in the text are to be numbered consecutively in Arabic
%numerals, in the order of first appearance. They are to be cited as
%superscripts without parentheses or brackets after punctuation marks
%like commas and periods but before punctuation marks like colons,
%semi-colons and question marks. Where superscripts might cause
%ambiguity, cite references in parentheses in abbreviated form,
%e.g.~(Ref.~12).
%
%\section{Footnotes}
%
%Footnotes should be numbered sequentially in superscript
%lowercase Roman letters.\fnm{a}\fnt{a}{Footnotes should be
%typeset in 8~pt Times Roman at the bottom of the page.}
%
%\section*{Note Added}
%
%Additional note can be added before Acknowledgment.
%
%\section*{Acknowledgments}
%
%This section should come before the References. Funding
%information may also be included here.
%
%\appendix{}
%
%Appendices should be used only when absolutely necessary. They should
%come immediately\break before the References. If there is more than
%one appendix, number them alphabetically. Number%\break
%displayed equations occurring in the Appendix in this way, e.g.~(\ref{that}),
%(A.2), etc.
%\begin{equation}
%\mu(n,t) = {\displaystyle\sum^\infty_{i=1} 1(d_i < t, N(d_i) = n) \over
%\displaystyle\int^t_{\sigma=0} 1(N(\sigma) = n)d\sigma}\,\, .\label{that}
%\end{equation}
%
%\section*{References}
%References are to be listed in the order cited in the text. Use the style shown
%in the following examples. For journal names, use the standard abbreviations.
%Typeset references in 9~pt Times Roman.
%
%%----------------------------- END BODY OF TEXT -----------------------------
%
%\begin{thebibliography}{000}
%\bibitem{1}
%J. J. Hopfield, ``Neurons with graded response have collective
%computational properties like two-state neurons,'' {\it Proc. Natl. Acad. Sci.},
%{\bf 81}, 3088--3092 (1984).
%
%\bibitem{2}
%D. W. Tank and J. J. Hopfield, ``Simple `neural' optimization networks:
%An A/D converter, signal decision circuit, and a linear programming circuit,''
%{\it IEEE Trans. on Circuits and Systems}, {\bf 33}, 533--541 (1986).
%
%\bibitem{3}
%Y. S. Foo and Y. Takefuji, ``Integer linear programming neural networks for
%job-shop scheduling,'' {\it Proc. IEEE Intl. Conf. on Neural Networks},
%{\bf II}, 341--348 (1988).
%%\phantom{00}
%\label{\labart-LastPage}
%\end{thebibliography}
\end{multicols}
\end{document}
