%% International Journal of Computational Intelligence Systems ---
%%%%%%%%%%%%%%%%%%%%%%%%%%%%%%%%%%%%%%%%%%%%%%%%%%%%%%%%%%%%%%%%%%%%%%%%%%%
\documentclass[11pt,twoside]{article}
\usepackage{ijcis}
%--------------------- ADDITIONAL PACKAGES HERE ---------------------------
% \usepackage{...}
%--------------------------------------------------------------------------
%
\def\labart{yourLabel}      % put a label from your choice here
%\Vol{1}                    % number of the Volume
%\Issue{1}                  % number of the issue
%\Month{January}            % month
%\Year{2008}                % year
%\received{...}
%\revised{...}
%
%---------------------------------------------------------------------------
\thispagestyle{empty}
%%------------------------- YOUR HEADINGS HERE -----------------------------
% Author's initials of first names+last names
\shortauthors{First Author, Second Author}
% Short title
\shorttitle{Instructions for Typesetting Manuscripts Using \LaTeX}
%---------------------------------------------------------------------------
%
%---------------------- YOUR TITLE ----------------------------------------
\title{%
PoVaTiM: A Possibilistic Valid-Time Model
 \\ \vspace*{0.04truein}
%Using \LaTeX\footnote{For the title, try not to use more than 3 lines.
%Typeset the title in 12 pt Times Roman, uppercase and boldface.}
}
%-------------------------- AUTHOR'S NAMES ----------------------------------
\author{%
Jose Enrique Pons\,\up{1}\,\,,
%\author{
Second Author\,\up{2}
}
%----------------------------- ADDRESSES ----------------------------------
\addresses{%
\up{1}
Department of Computer Science and Artificial Intelligence, University of Granada\\
C/ Periodista Daniel Saucedo, S/N,\\
Granada, , Spain%
\footnote{State completely without abbreviations, the affiliation
and mailing address, including country. Typeset in 10~pt Times Italic.}
\\ \vspace*{0.04truein}
E-mail: jpons@decsai.ugr.es
%}\address{%
\\ \vspace*{0.05truein}
\up{2}
Group, Company,\\
Address,\\ City, State ZIP/Zone, Country
\\ \vspace*{0.04truein}
E-mail: address
}
%---------------------------------------------------------------------------
\pagestyle{myheadings}
\begin{document}
\label{\labart-FirstPage}

\maketitle
%-------------------------- ABSTRACT ---------------------------------------
\abstracts{%
The abstract should summarize the context, content and conclusions of the
paper in less than 80 words. It should not contain any references or
displayed equations. Typeset the abstract in 9~pt Times Roman with
baselineskip of 11 pt, making an indentation of 2.5 picas on the left
and right margins.
}
\par\bigskip\par
%-------------------------- KEYWORDS ---------------------------------------
\keywords{List of four to six keywords which characterize the article.}

\vspace*{10pt}\textlineskip
%-------------------------- BEGIN BODY OF TEXT -----------------------------
\begin{multicols}{2}

\section{Introduction}
Contributions are to be in English. Authors are encouraged to
have their contribution checked for grammar. American or British
spelling should be used. Abbreviations are allowed but should be
spelt out in full when first used. Integers ten and below are to
be spelt out. Italicize foreign language phrases (e.g.~Latin,
French).

\section{The Main Text}
The text is to be typeset in 10 pt Times Roman, single spaced with
baselineskip of 13 pt. Text area (excluding running title) is 6.75
inches across and 8.8 inches deep. Final pagination and insertion of
running titles will be done by the publisher.

\section{Major Headings}
Major headings should be typeset in boldface with the first
letter of important words capitalized.

\subsection{Sub-headings}
%\vspace*{-4pt}  %only when needed
Sub-headings should be typeset in bolditalic with the first
letter of first word capitalized and section number in boldface.

%\vspace*{-1pt}  %only when needed
\subsubsection{Sub-subheadings}
%\vspace*{-4pt}  %only when needed

Typeset in italic (Section No. to be in Roman) and capitalize
the first letter of the first word only.

%\vspace*{-1pt}  %only when needed
\subsection{Numbering and spacing}
%\vspace*{-4pt}  %only when needed

Sections, sub-sections and sub-subsections are numbered in Arabic.
Use double spacing after major and subheadings, and single spacing
after\break sub-subheadings.

%\vspace*{-1pt}  %only when needed
\subsection{Lists of items}
%\vspace*{-4pt}  %only when needed

{Lists may be laid out with each item marked by\hfilneg}

\noindent a dot:
\begin{itemize}
\item item one,
\item item two.
\end{itemize}

\setcounter{footnote}{0}
\renewcommand{\thefootnote}{\alph{footnote}}

Items may also be numbered in lowercase Roman numerals:
\begin{enumerate}[(i)]\Nospacing
\item item one
\item item two
    \begin{enumerate}[(a)]\Nospacing
    \item Lists within lists can be numbered with lowercase Roman letters,
    \item second item.
    \end{enumerate}
\end{enumerate}

\subsection{Running Heads}

Please provide a shortened running head (not more than four words,
each starting with a Capital) for the title of your paper. This will
appear with page numbers on the top right-hand side of your paper on
odd pages.

For the running heads for the authors names should appear on your paper
on the even  pages. Please apply the following rules for theses running
heads:
\begin{itemize}
\item for one author: only the initial plus the full last name (e.g., D. Ruan),
\item for two authors: D. Ruan, T. Li,
\item for three authors or more: D. Ruan \emph{et al.}
\end{itemize}

\section{Equations}

Displayed equations should be numbered consecutively in each
section, with the number set flush right and enclosed in
parentheses.
\begin{equation}
\mu(n,t) = \dfrac{\sum^\infty_{i=1} 1(d_i < t, N(d_i) = n)}
{\int^t_{\sigma=0} 1(N(\sigma) = n)\mathrm{d}\sigma}\,\, .
\label{this}
\end{equation}
Equations should be referred to in abbreviated form,
e.g.~``Eq.~(\ref{this})'' or ``(2)''. In multiple-line equations,
the number should be given on the last line.

Displayed equations are to be centered on the page width.
Standard English letters like x are to appear as $x$ (italicized)
in the text if they are used as mathematical symbols. Punctuation
marks are used at the end of equations as if they appeared directly in the text.

\begin{theorem}
Theorems, lemmas, etc. are to be numbered consecutively in the
paper, just by using \\
\verb+\begin{env}My text here...\end{env}+\\
for the environments \emph{theorem}, \emph{lemma},
\emph{proposition}, and \emph{corollary}.\\
\end{theorem}

\vspace*{-5pt}

\begin{proof}
Proofs are produced with the command\\
\verb+\begin{proof}{My proof...}\end{proof}+.\\
It should end with\ $\qed$\kern0.3pt.
\end{proof}

\section{Illustrations and Photographs}

Figures are to be inserted in the text nearest their first reference.
Original india ink drawings or glossy prints are preferred. Please
send one set of originals with copies. If the author requires the
publisher to reduce the figures, ensure that {the figures (including
letterings and numbers) are large enough to be\hfilneg}

%\begin{figure}[htbp] %ORIGINAL SIZE: width=1.4TRUEIN; height=1.5TRUEIN
%\vspace*{13pt}
%\centerline{\psfig{file=ap-ijcis.eps}} %100 percent
\vspace*{10pt}
\fcaption{Labeled tree {\it T}.}
%\end{figure}
\vspace*{13pt}

\noindent
clearly seen after reduction. If photographs are to be used, only
black and white ones are acceptable.

Figures are to be sequentially numbered in Arabic numerals. The
caption must be placed below the figure. For those figures with
multiple parts which appear on different pages, it is best to
place the full caption below the first part, and have
e.g.~``Fig.~1. ({\it Continued})'' below the last part. Typeset
in 9~pt Times Roman with baselineskip of 11~pt. Use double
spacing between a caption and the text that follows immediately.

Previously published material must be accompanied by written
permission from the author and publisher.

\section{Tables}

Tables should be inserted in the text as close to the point of
reference as possible. Some space should be left above and below
the table. Tables should be numbered sequentially in the text in
Arabic numerals. Captions are to be centralized above the
tables. Typeset tables and captions in 9~pt Times Roman with
baselineskip of 11~pt.

If tables need to extend over to a second page, the continuation
of the table should be preceded by a caption, e.g.~``Table~2.
({\it Continued})''

\vglue13pt
%\begin{table}[htbp]
\tcap{Number of tests for WFF triple NA = 5, or\break NA = 8.}
\centerline{\small NP}
\vglue-6pt
\centerline{\small\baselineskip=13pt
\begin{tabular}{l c c c c c}\\
\hline
{} &{} &3 &4 &8 &10\\
\hline
{} &\phantom03 &1200 &2000 &\phantom02500 &\phantom03000\\
NC &\phantom05 &2000 &2200 &\phantom02700 &\phantom03400\\
{} &\phantom08 &2500 &2700 &16000 &22000\\
{} &10 &3000 &3400 &22000 &28000\\
\hline\\
\end{tabular}}
%\end{table}

\section{References}

References in the text are to be numbered consecutively in Arabic
numerals, in the order of first appearance. They are to be cited as
superscripts without parentheses or brackets after punctuation marks
like commas and periods but before punctuation marks like colons,
semi-colons and question marks. Where superscripts might cause
ambiguity, cite references in parentheses in abbreviated form,
e.g.~(Ref.~12).

\section{Footnotes}

Footnotes should be numbered sequentially in superscript
lowercase Roman letters.\fnm{a}\fnt{a}{Footnotes should be
typeset in 8~pt Times Roman at the bottom of the page.}

\section*{Note Added}

Additional note can be added before Acknowledgment.

\section*{Acknowledgments}

This section should come before the References. Funding
information may also be included here.

\appendix{}

Appendices should be used only when absolutely necessary. They should
come immediately\break before the References. If there is more than
one appendix, number them alphabetically. Number%\break
displayed equations occurring in the Appendix in this way, e.g.~(\ref{that}),
(A.2), etc.
\begin{equation}
\mu(n,t) = {\displaystyle\sum^\infty_{i=1} 1(d_i < t, N(d_i) = n) \over
\displaystyle\int^t_{\sigma=0} 1(N(\sigma) = n)d\sigma}\,\, .\label{that}
\end{equation}

\section*{References}
References are to be listed in the order cited in the text. Use the style shown
in the following examples. For journal names, use the standard abbreviations.
Typeset references in 9~pt Times Roman.

%----------------------------- END BODY OF TEXT -----------------------------

\begin{thebibliography}{000}
\bibitem{1}
J. J. Hopfield, ``Neurons with graded response have collective
computational properties like two-state neurons,'' {\it Proc. Natl. Acad. Sci.},
{\bf 81}, 3088--3092 (1984).

\bibitem{2}
D. W. Tank and J. J. Hopfield, ``Simple `neural' optimization networks:
An A/D converter, signal decision circuit, and a linear programming circuit,''
{\it IEEE Trans. on Circuits and Systems}, {\bf 33}, 533--541 (1986).

\bibitem{3}
Y. S. Foo and Y. Takefuji, ``Integer linear programming neural networks for
job-shop scheduling,'' {\it Proc. IEEE Intl. Conf. on Neural Networks},
{\bf II}, 341--348 (1988).
%\phantom{00}
\label{\labart-LastPage}
\end{thebibliography}
\end{multicols}
\end{document}
