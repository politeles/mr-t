%%%%%%%%%%%%%%%%%%%%%%%%
%
% ABSTRACT
%
%%%%%%%%%%%%%%%%%%%%%%%%
In reality, some objects or concepts have properties with a time-variant or time-related nature. Modelling these kinds of objects or concepts in a (relational) database schema is possible, but time-variant and time-related attributes have an impact on the consistency of the entire database. Therefore, temporal database models have been proposed to deal with this. Time itself can be at the source of imprecision, vagueness and uncertainty, since existing time measuring devices are inherently imperfect. Accordingly, human beings manage time using temporal indications and temporal notions, which may contain imprecision, vagueness and uncertainty. However, the imperfection in human-used temporal indications is supported by human interpretation, whereas information systems need support for this. Several proposals for dealing with such imperfections when modelling temporal aspects exist. Some of these proposals transform the temporal expression into a compact representation. Other proposals consider the temporal indications in the used temporal expressions to be the source of imperfection. 
In this work we present a novel model to deal with imperfections in valid-time databases. Next to that, the data manipulation language, \emph{DML} is defined and implemented.