%Introductory text: present the entire structure (and purpose?) of the preliminaries section
In this section, first some basic concepts and vocabulary concerning relational databases are reviewed. After this, some attention is given to temporal databases in general and to valid-time relations in particular. An example is presented, which will be used throughout the entire paper. Next, some concepts concerning possibility theory and fuzzy numbers and intervals are explained and defined, along with their terminology. The framework of set evaluation by ill-known constraints~\cite{Pons2011} is presented. Finally, some attention is given to the modelling and handling of imperfections in temporal modelling.


%Something about relational databases in general and relations in specific (to make the vocabulary clear)
\subsection{Relational Databases}
In this subsection, a few structural and behavioral aspects of the relational database model will be presented.

In a relational database, information is structured in \emph{relations}. As explained in the introduction, a collection of similar entities is modelled by an entitytype, which, using the relational database model, will be modelled by a relation. A relation consists of a \emph{relation schema} and an \emph{extention}. The relation schema basically determines the relation structure, by dictating which data (describing entities) will be kept and how these data will be represented. Each (atomic) part of these data is a result value of a measurement of a property of an entity or a description of a property of an entity. The extention contains these data. A relation schema consists of a \emph{relation name} and a finite set of \emph{attributes}. The relation name is of course the name of the entire relation, while the attributes describe common properties of the entities modelled by the relation. Every attribute consists of an attribute name and an associated \emph{data type}. Each data type has its \emph{domain} and its operators. The domain of a data type is the set of values which data of the data type in question can assume. Operators of a data type can be applied to data of a data type. The extention of a relation is a set of \emph{tuples}. Each tuple represents an enitity, by containing values: for every attribute in the attribute set of the relation schema, the tuple contains the value describing the exact state of the property of the entity. %Remember: a tuple is just an ordered tuple of values.

In this work, a specific variant of the relational database model will be presented.