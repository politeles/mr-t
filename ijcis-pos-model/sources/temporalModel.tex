%%%%%%%%%%%%%%%%%%%%%%%%%%%%%%%%
%
% Temporal model.tex
%
%%%%%%%%%%%%%%%%%%%%%%%%%%%%%%%

\subsection{\label{subsec:temporal-model}The generalized temporal model}
The model is based on the GEFRED~\cite{Medina1994} (Generalized Model of Fuzzy Relational DB) model. This model is extended by adding valid-time support. The information in the system is defined by the following elements:

\begin{definition}
\label{def:generalized-fuzzy-domain}
$D$ is the discourse domain, $\tilde \Pow\left(D \right)$ is the set of all possibility distributions defined on $D$, plus the NULL constant. The generalized fuzzy domain $D_G$ is defined as:
\begin{equation}
\label{eq:generalized-fuzzy-domain}
D_G \subseteq \tilde \Pow\left(D \right)\cup \text{NULL}
\end{equation}
\end{definition}
The datatypes for $D_G$ are shown in table \ref{tbl:gefred-data-types}. 

\begin{definition}
\label{def:typeof-domain}
The function typeof$\left(D \right)$ returns the datatype associated with the domain $D$. 

\end{definition}



\vglue13pt
%\begin{table}[htbp]
\tcap{\label{tbl:gefred-data-types}Data types}
%\centerline{\small TITLE}
\vglue-6pt
\centerline{\small\baselineskip=13pt
\begin{tabular}{c p{4cm} }\\
No. & Datatype \\ \hline
1 & A single scalar. \\
2 & A single number. \\
3 & A set of mutually exclusive possible scalar assignations. \\
4 & A set of mutually exclusive possible numeric assignations. \\
5 & A possibility distribution in a scalar domain. \\
6 & A possibility distribution in a numeric domain. \\
7 & A real number in $\left[0, 1 \right]$ referring to degree of matching. \\
8 & An \emph{UNKNOWN} value. \\
9 &  An \emph{UNDEFINED} value. \\
10 & A \emph{NULL} value. \\
\hline\\
\end{tabular}
} 
 



It is possible to define a more specific generalized temporal domain, $\T_G$:

\begin{definition}
\label{def:generalized-fuzzy-temporal-domain}
Generalized fuzzy temporal domain.\\
If $\T$ is the temporal domain, $\tilde \Pow\left( \T\right)$ is the set of all \emph{normalized} possibility distributions (see Section \ref{subsec:possibility-theory}, equation \eqref{NormalizationProperty}) defined on $\T$.
The \textbf{Generalized Fuzzy Temporal Domain}, $\T_G$ is
\begin{equation}
\T_G \subseteq \left \lbrace \tilde \Pow\left( \T\right) \cup \text{NULL} \right \rbrace
\end{equation}
\end{definition}

Note that $\T_G \subseteq D_G$. The datatypes for this domain have been studied previously in section \ref{sec:time-rep} and are shown in tables \ref{tbl:time-point-types},\ref{tbl:time-interval-types}.



\begin{definition}
A generalized fuzzy temporal relation $R_{FTG}$ is given by:
\label{def:generalized-fuzzy-temporal-relation}
\begin{equation}
\label{eq:generalized-fuzzy-temporal-relation}
R_{FTG} = \left(\Head, \Body \right)
\end{equation}
Where $\Head$ is the Head of the relation and consist on a fixed set of triplets attribute- domain - compatibility with an optional the valid-time attribute:

\begin{align}
\label{eq:head-valid-time}
\Head = \big \lbrace \left(A_{G1}:D_{G1}\left[,C_{A_{G1}} \right] \right),\\
\nonumber
 \ldots,\\
 \nonumber
  \left(A_{Gn}:D_{Gn}\left[,C_{A_{Gn}} \right] \right),\\
  \nonumber
  \Big[  \left( \text{PVP}, D_{\text{PVP}}\left[,C_{A_{\text{PVP}}} \right] \right) \Big] \big \rbrace
\end{align}
Note that $D_{Gj}$ ($j = 1, \ldots, n$) is the domain for the attribute $A_{Gj}$. $C_{A_{Gj}}$ is the compatibility attribute in the unit interval $\left[0, 1 \right]$.

$\Body$ is the body of the relation and it consists on a set of tuples. Each tuple is a triplet attribute- value- degree with an optional valid-time attribute:

\begin{align}
\label{eq:body-valid-time}
\Body = \big \lbrace \left(A_{G1}:\tilde{d}_{i1}\left[,c_{i1} \right] \right),\\
\nonumber
 \ldots,\\
 \nonumber
  \left(A_{Gn}:\tilde{d}_{in}\left[,c_{in} \right] \right),\\
  \nonumber
   \Big[  \left( \text{PVP}, \tilde{d}_{\text{PVP}} \left[,C_{A_{\text{PVP}}} \right] \right)  \Big] \big \rbrace
\end{align}

\end{definition}


The definition in \cite{Medina1994} for $R_{FTG}$ shows that classical relations are a particular case of this model. 

\begin{definition}
\label{def:value-component}
The value component $R^{v}_{FTG}$ of a fuzzy temporal relation $R_{FTG}$ is a set with the value components for both the head and the body of the relation:
\begin{align}
\label{eq:value-component}
R^{v}_{FTG} = \left \lbrace \Head^{v},\Body^{v} \right \rbrace \\
\nonumber
\text{Where: } \\
\nonumber
\Head^{v} = \left \lbrace A_{G1}:D_{G1}, \ldots,  A_{Gn}:D_{Gn} \right \rbrace \\
\nonumber
\Body^ {v} = \left \lbrace A_{G1}:\tilde{d}_{i1}, \ldots,  A_{Gn}:\tilde{d}_{in}\ \right \rbrace \\
\end{align}
\end{definition}

\begin{definition}
\label{def:compatibility-component}
The compatibility component $R^{c}_{FTG}$ of a fuzzy temporal relation $R_{FTG}$ is a set with the compatibility components for both the head and the body of the relation:
\begin{align}
\label{eq:compatibility-component}
R^{c}_{FTG} = \left \lbrace \Head^{c},\Body^{c} \right \rbrace \\
\nonumber
\text{Where: } \\
\nonumber
\Head^{c} = \left \lbrace \left[C_{A_{G1}} \right], \ldots,  \left[C_{A_{Gn}} \right] \right \rbrace \\
\nonumber
\Body^ {c} = \left \lbrace \left[ c_{i1} \right], \ldots, \left[ c_{in} \right] \right \rbrace \\
\end{align}
\end{definition}


\begin{definition}
\label{def:temporal-component}
The temporal component $R^{t}_{FTG}$ of a fuzzy temporal relation $R_{FTG}$ is a set with the temporal components for both the head and the body of the relation:
\begin{align}
\label{eq:temporal-component}
R^{t}_{FTG} = \left \lbrace \Head^{t},\Body^{t} \right \rbrace \\
\nonumber
\text{Where: } \\
\nonumber
\Head^{t} = \left \lbrace \left( \text{PVP}, D_{\text{PVP}}\left[,C_{A_{\text{PVP}}} \right] \right) \right \rbrace \\
\nonumber
\Body^ {t} = \left \lbrace  \left[ \text{PVP}, \tilde{d}_{\text{PVP}} \left[,C_{A_{\text{PVP}}} \right] \right]  \right \rbrace \\
\end{align}
\end{definition}

Analogously, it is possible to define the the value component for the temporal part and the compatibility component for the temporal part. 

\begin{definition}
\label{def:generalized-primary-key}
A generalized primary key, $K_G$ is a subset of the head:
\begin{align}
\label{eq:generalized-primary-key}
K_G \subseteq \Head, K_G = \left \lbrace  \left(A_{Gs}:D_{Gs} \right) \right \rbrace \\
\nonumber
s\in S \subseteq \left(1, \ldots, n \right) \\
\nonumber
\forall s \in S, \text{Typeof } \left(D_{Gs} \right) \in \left \lbrace 1, 2 \right \rbrace \\
\nonumber
\forall i, i' \in \left \lbrace 1, \ldots, m\right \rbrace , \exists s \in S: \\
\nonumber
\left(A_{Gs}:d_{is} \right) \neq \left(A_{Gs}:d_{i's} \right)
\end{align}
\end{definition}


\begin{definition}
\label{def:generalized-fuzzy-temporal-key}
A generalized fuzzy temporal key, $K_{GT}$ is a subset of the head. An attribute, namely \emph{version} $V$ is added when dealing with valid-time.
\begin{align}
\label{eq:generalized-fuzzy-temporal-key}
K_{GT} \subseteq \Head, K_{GT} = \left \lbrace  \left(A_{Gs}:D_{Gs} \right) \right.  \\
\nonumber
 \left. \cup  \left(V_{ID}:D_{ID} \right)	\right \rbrace \\
\nonumber
\text{Typeof }\left(D_{ID} \right) = \\
\nonumber
s\in S \subseteq \left(1, \ldots, n \right) \\
\nonumber
\forall s \in S, \text{Typeof } \left(D_{Gs} \right) \in \left \lbrace 1, 2 \right \rbrace \\
\nonumber
\forall i, i' \in \left \lbrace 1, \ldots, m\right \rbrace , \exists s \in S: \\
\nonumber
\left(A_{Gs}:d_{is} \right) \neq \left(A_{Gs}:d_{i's} \right)
\end{align}

\end{definition}


\begin{example}


\end{example}


\subsection{\label{subsec:data-manipulation}Data manipulation language}

The algebra for the data manipulation is defined in \cite{Medina1994}. The Generalized Fuzzy Relational Algebra manipulates relations like $R_{FG}$. The operations defined are: \emph{Union, Intersection, Difference, Cartesian Product, Projection, Join} and \emph{Selection}. Thus, in this section we will describe and implement the following operations for temporal databases, as described in \ref{subsubsec:Understanding-valid-time-databases}. The operations implemented are: \emph{Insert, Modify, Delete, } and \emph{Revise}. The semantics of the operations will be the same that those defined for a crisp temporal database, whereas the temporal representation is made by the possibilistic valid-time period and the ill-known constraints (see sections \ref{sec:prelim} and \ref{sec:time-rep}.

It is important to nottice that whereas the result of the evaluation of any comparison between crisp time intervals is boolean, the evaluation of any comparison between \emph{PVPs} is in the unit interval.  Therefore, for a valid-time object, say $A \in R$, in a crisp temporal database it is not possible an overlaping among any valid-time interval ($\forall I, \not \exists J: \left(A, I \right) \in R \wedge \left(A, J \right) \in R \wedge \not \Overlap\left(I, J \right)$). Thus, from the point of view of a fuzzy temporal database, there are the following options:

\begin{itemize}
\item [\textbf{Strictly consistent}] It is not possible for two PVPs of the same object to be valid at the same time. In other words, if $A$ is an object, $R$ is a relation and $\pi_{I}\left(t \right)$ and $\pi_{J}\left(t \right)$ are the possibilities for a time point $t \in \T$ to be in the PVPs $I$ and $J$ respectively,
\begin{align}
\label{eq:stricly-consistent}
\forall I,J, \forall t \in \T :\\
%\nonumber
%\left(A, I \right) \in R \wedge \\
%\nonumber
%\left(A, J \right) \in R : \\
\nonumber
\pi_{I}(t) + \pi_{J}(t) \leq 1
\end{align}

\item [\textbf{Ill-consistent}] The overlapping of two PVPs of the same object do overlap. It can be distinghised among the following sub-types:
	\begin{itemize}
	\item [\emph{Co-existence}] Two versions of the same object $A$ may exist at the same time.
	\begin{align}
	\label{eq:co-existence}
	\forall I, \exists ! J, \exists t \in \T : \\
	\nonumber
	\pi_{I}(t) + \pi_{J}(t) \geq 1
	\end{align}
	\item [\emph{Weak-consistence}] Several versions of the same object $A$ may exist at the same time.
	\begin{align}
	\label{eq:weak-consistence}
	\forall I, \exists J, \exists t \in \T : \\
	\nonumber
	\pi_{I}(t) + \pi_{J}(t) \geq 1
	\end{align}
	\end{itemize}
\end{itemize}

In the implementation of the DML operations we will consider a strictly consistent temporal database. Since the time intervals are now possibilistic valid-time periods, PVPs, the auxiliary functions defined in equations \eqref{eq:close-a-crisp-interval} to \eqref{eq:close-current} should be re-defined.

\begin{definition}
\label{def:close-r-a-pvp}
Consider two PVPs given by $I = \left(X, Y \right)$ and $J = \left(X', Y' \right)$. The CloseR function closes the PVP given by $I$ with a convex combination of ill-known constraints (see section \ref{subsec:interval-evaluation-by-ill-known-constraints}):
\begin{align}
\label{eq:close-r-a-pvp}
\text{CloseR}\left(I, J\right) = \\
\begin{cases}
\nonumber
I   & \mbox{ if } Y \not \in \text{UNKNOWN}  \\
I = \left[X, Z \right] & \mbox{ if } Y \in \text{UNKNOWN}, \\
& Z \triangleq \left \lbrace C_1\left(>, X \right), C_2\left(<, X' \right) \right \rbrace
\end{cases}
\end{align}
\end{definition}

\begin{definition}
\label{def:pvp-current-in-relation}
Consider an object $A$ and a relation $R$ in a fuzzy temporal database. The function Current$\left(R, A \right)$ obtains the tuple $\left(A, I \right), I = \left(X, Y \right)$ that is current in the relation $R$, that is it, $I = \left[X , \text{UNKNOWN} \right]$.

\begin{align}
\label{eq:pvp-current-in-relation}
\mbox{Current} \left(R, A \right) &=& \\ 
\begin{cases}
\nonumber
I & \mbox{ if } \exists  \in I : \\
&  (A,I) \in R \wedge Y \in \text{UNKNOWN}\\
\emptyset & \mbox{ in any other case }
\end{cases}
\end{align}
\end{definition}

Now it is possible to close the current version of an object by using \eqref{eq:close-r-a-pvp} and \eqref{eq:pvp-current-in-relation}.

\begin{definition}
\label{def:pvp-close-current-version}
Consider an object $A$, the relation $R$ and a pvp, $J$. Then, the function close - current$\left(R, A, J \right)$ closes any current version of the object $A$ if it exists and add the new version $\left(A, J \right)$.

\begin{eqnarray}
\label{eq:pvp-close-current}
\text{Close-current} \left(R, A, J \right) =\\
\begin{cases}
\nonumber
\big \lbrace R - \left(A, I \right) \cup \left \lbrace \left(A, \mbox{ CloseR } \left(I, J\right) \right) \cup \left(A, J\right)\right \rbrace  \big \rbrace \\
\nonumber
\mbox{ if } I = \mbox{ Current } \left(R, A \right) \wedge J > \mbox{ CloseR } \left(I, J \right)   \\
\nonumber R , \text{ in any other case}
\end{cases}
\end{eqnarray}
\end{definition}

\subsubsection{\label{subsubsec:insert-fuzzy-temporal}Insert}

\subsubsection{\label{subsubsec:modify-fuzzy-temporal}Modify}

\subsubsection{\label{subsubsec:delete-fuzzy-temporal}Delete}

\subsubsection{\label{subsubsec:revise-fuzzy-temporal}Revise}

