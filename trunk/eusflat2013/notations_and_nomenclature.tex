Often, time is thought of as following an axis (a time axis) and thus it is often seen as an ordered set of infinitely short `moments' or `points' in time, called \emph{instants}~\cite{Dyreson1994},~\cite{Jensen1998}. This is also the case in the presented work. A \emph{time interval} is then an interval subset of this ordered set of instants. In this paper, instants are noted using lowercase letters and time intervals using uppercase letters. Also, a time interval $I$ starting at (and containing) instant $s$ and ending at (and containing) instant $e$ is noted using square brackets: $I = \left[s, e\right]$.

The presented work allows time intervals to be subject to uncertainties. These uncertainties are always seen as caused by a (partial) lack of knowledge: the exact, intended time interval is not known, eventhough there is only one correct intended time interval and as such no variability. Confidence about exactly which interval is the intended one in the context of such uncertainties is modelled using possibility theory~\cite{DidierDubois1988a}. In the presented work, possibility is always interpreted as plausibility given a (partial) lack of knowledge.

To be able to distinguish easily, time intervals not subject to uncertainties will be called \emph{crisp time intervals} in the presented work.