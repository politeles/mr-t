%%%%%%%%%%%%%%%%%%%%%%%%%%%%%%%%%%%%%%%%%%%%%%%%%%%%%%%%%%%%%%%%%%%%%%
%
% Introduction
%
%%%%%%%%%%%%%%%%%%%%%%%%%%%%%%%%%%%%%%%%%%%%%%%%%%%%%%%%%%%%%%%%%%%%%%

The concept of time itself is very complex to handle and interpret~\cite{Klein1994},~\cite{Shackle1961}, though it is very natural and omnipresent. As information systems often attempt the modelling of natural objects, concepts or processes, they often require modelling temporal aspects or concepts. Thus, several proposals have been concerned with the obtaining of theoretical models that allow the modelling or representation of time~\cite{Bolour1982},~\cite{VanderCruyssen1997}.

A very specific type of information systems are database systems, which are computer systems designed to manage databases. A database contains data representing real objects or concepts. Each (atomic) part of these data is a result value of a measurement of a property or a description of a property of a real object or concept. In reality, some aspects or properties of objects or concepts are time-variant or time-related. e.g., the moment of a bank transaction is traditionally a moment in time and thus a time-related notion, the function of an employee in a company can change through recorded history and is thus time-variant. A temporal database schema is a database schema that models real objects or concepts with time-related or time-variant properties. However, the modelling of temporal aspects has a direct impact on the consistency of the temporal database, because the temporal nature of these aspects imposes extra integrity constraints. An example. Consider a relation in a relational library database, modelling the presence of books in the library. Every physical book is represented by a unique identifier. Every record in the relation contains such an identifier, a date on which the corresponding book was loaned and a date on which it was subsequently returned (if it was returned). Without further precautions, a library employee could add several records with the same book identifier, different `loaned'-dates and no `returned'-dates. This group of records would represent the same physical book being loaned several times on different dates and never returned, which is of course impossible. A temporal database model will typically constrain record insertion and prevent similar modelling inconsistencies.

A lot of research concerns temporal database models and their approaches to the modelling of time. The first efforts were towards the representation of historical information related to objects represented by records in a database~\cite{Clifford1985}. Some proposals tried to extend the Entity Relationship Model~\cite{Klopprogge1983}, without impact on any database standards like SQL~\cite{Sarda1990}.

Notably, in 1994, ``A Consensus Glossary of Temporal Database Concepts'' was published~\cite{Dyreson1994}. For this publication, 44 temporal database researchers, among which some of the main researchers in this field, cooperated to reach a consensus on the nature and definitions of some of the main temporal database concepts and terminology. This glossary was subsequently updated in 1998~\cite{Dyreson1998}.

An interesting issue in temporal modelling concerns relationships between temporal notions. Notably, Allen~\cite{Allen1983} studied temporal relationships between time intervals (and as a special case time points). Among others, the querying of temporal databases has greatly profited from these temporal relationships, because they allowed for richer and more complex user-specified temporal query demands, by allowing to express more complex relationships between the temporal notions in the temporal expressions in the query and the temporal indications in the database. e.g., in a relation modelling who was department head of an institution during which periods of time, a query like `who were the department heads when Thomas worked for the institution' can be evaluated using similar relationships.

Humans handle temporal information using certain temporal notions like time intervals or time points~\cite{Dyreson1994}, and they often have to deal with imperfections like imprecisions, vaguenesses, uncertainties or inconsistencies possibly contained in the descriptions of these temporal notions. Among many others, these possible imperfections in descriptions of temporal notions determine an important issue in temporal modelling. e.g., the description of the temporal notion in a sentence like `The Belfry of Bruges was finished on a day somewhere between 01/01/1201 A.D. and 31/12/1300 A.D.' contains imperfection because of the uncertainty in the used time-related expression. It is known that the building was finished on a single day, but it is not known precisely which day this was.

To allow information systems to cope with these and similar data imperfections, many approaches adopt fuzzy sets~\cite{Zadeh1965} for the representation of temporal information~\cite{Mitra1994},~\cite{Nagypal2003},~\cite{Billiet2011},~\cite{Dubois2003}. The temporal relationships studied by Allen were fuzzified by several authors~\cite{Ohlbach2004},~\cite{Nagypal2003},~\cite{Schockaert2008}. Garrido et al. ~\cite{Garrido2009} present different temporal operators, defined by a combination of regular fuzzy comparisons. Both~\cite{Garrido2009} and~\cite{Pons2011} deal with uncertainty in temporal expressions concerning time intervals. Other approaches, like~\cite{Qiang2009}, use rough sets~\cite{Pawlak1995} to represent time intervals.

Next to temporal modelling, some attention has been spent on temporal reasoning~\cite{Allen1983}. Although temporal reasoning is not discussed in this chapter, it should be noted that, among others, Dubois and Prade et al.~\cite{Dubois2003},~\cite{DuBois1989} have dealt with fuzziness and uncertainty in temporal reasoning.

The aim of this work is to present and explain the main operations of the manipulation language for a temporal database which stores the valid-time periods of the objects with uncertainty. The rest of the work is organized as follows. Section \ref{sec:prelim} presents some background concepts about both possibility theory and temporal databases. In section \ref{sec:time-rep} the representation of the valid-time intervals in the database is explained. Section \ref{sec:temporal-model} explains the main concepts and behaviour of the temporal model. The Data Manipulation Language (DML) is described and implemented. Finally in Section \ref{sec:conclusions} presents the conclusions and some possibilities for future research work.

