%A database schema models a part of reality by modelling real objects or concepts. For this, a database contains data. 
A database contains data representing real objects or concepts. Every one of these data is a measurement or description of a property of a real object or concept. In reality, some aspects or properties of objects or concepts are time-variant or time-related. E.g. the moment of a bank transaction is traditionally a moment in time and thus a time-related notion, the function of an employee in a company can change through recorded history and is thus time-variant. A \emph{temporal database schema} \cite{Dyreson1994} is a database schema that models real objects or concepts with time-related or -variant properties. However, the modelling of temporal aspects has a direct impact on the consistency of the temporal database, because the temporal nature of these aspects imposes extra integrity constraints. An example. Consider a relation in a relational library database, modelling the presence of books in the library. Every physical book is represented by a unique identifier. Every record in the relation contains such an identifier, a date on which the corresponding book was loaned and a date on which it was subsequently returned (if it is returned). Without further precautions, a library employee could add several records with the same book identifier, different `loaned'-dates and no `returned'-dates. This would represent a situation in which the same physical book was loaned several times on different dates and never returned, which is of course impossible. A temporal database schema will typically constrain record insertion and prevent similar modelling inconsistencies.

Humans handle temporal information using certain temporal notions like time intervals or time points \cite{Dyreson1994}. The modelling and handling of these temporal notions in information systems has always been seen as a difficult task \cite{Bolour82}. Klein \cite{klein94} studied the concept of time in language and Devos \cite{devos94} modelled vague temporal expressions by means of fuzzy sets \cite{zadeh65}. Notably, temporal relationships between time intervals (and as a special case instants~\cite{Dyreson1994}) were studied by Allen \cite{Allen83}. Among many others, an issue in temporal modelling is the possible imperfection in (descriptions of) temporal notions. E.g. the temporal notion in a sentence like `The Belfry of Bruges was finished on one single day somewhere between 1/01/1201 A.D. and 31/12/1300 A.D.' contains imperfection because of the uncertainty in the used time-related expression. It is known that the building was finished on a single day, but it is not known precisely which day this was.

To allow information systems to cope with these and similar imperfections, many approaches adopt fuzzy sets for the representation of temporal information \cite{343607}, \cite{nagypal03}, \cite{Billiet:Pons:Matthe:DeTre:Pons:2011:BipolarFuzzy}, \cite{Dubois:jucs_9_9:fuzziness_and_uncertainty_in}. The temporal relationships studied by Allen were fuzzified by several authors \cite{ohlbach04}, \cite{nagypal03}, \cite{schockaert08}.

Next to this, the inclusion of temporal indications in a relational database leads to several practical problems. In some specific works, authors consider the necessity of allowing imprecision in the representation of temporal indications \cite{Cru97}, \cite{Garrido2009}. Most of these approaches use the concepts of fuzzy numbers and fuzzy intervals to represent imprecise temporal indications and considerable attention has thus been given to the problem of transforming two fuzzy numbers which represent the boundaries of an imprecise time interval into one fuzzy interval representing this interval. There are several proposals for this transformation, but some of them might show signs of minor issues, as pointed out in \cite{Pon11}. 

In this work, an approach is proposed that allows users to query a valid-time relation using temporal demands containing uncertainty. The novelty of this approach is that it allows uncertainty in the query's temporal demand. The presented approach is based on the possibilistic framework for set evaluation found in \cite{Pon11}. In section \ref{sec:preliminaries}, some main concepts are explained and the part of the framework on which this work is based, is briefly presented. In section \ref{sec:proposal}, this paper's approach is presented. Finally, in section \ref{sec:futher-research}, conclusions are drawn and some directions for future research are presented.

%In the attempts at temporal modelling, one issue is the inherent complexities of the concept of time. Klein~\cite{klein94} studied the concept of time in language and Devos~\cite{devos94} modelled vague temporal expressions by means of fuzzy sets~\cite{zadeh65}. 
%\\
%\\
%Transition between temporal granularities~\cite{Lin97} is also considered as a source of imprecision in temporal modelling and some proposals consider granularity as the base of the temporal model~\cite{Cru97}.
%Different approaches have been proposed \cite{Jensen91,Snodgrass84,Nascimento95} that consider time as a crisp notion.

%\\
%Specifically towards information systems, several proposals have been concerned with the obtaining of theoretical models that allow the representation of time in information systems~\cite{Cru97}. Notably, temporal relations between time intervals and to a lesser extent time points, were first studied by Allen~\cite{Allen83}.
%\\
%To allow information systems to cope with these imperfections, many approaches adopt fuzzy sets for the representation of temporal information~\cite{343607,nagypal03,Billiet:Pons:Matthe:DeTre:Pons:2011:BipolarFuzzy,Dubois:jucs_9_9:fuzziness_and_uncertainty_in}. The temporal relations studied by Allen were recently fuzzified by several authors~\cite{ohlbach04,nagypal03,schockaert08}.

%Next to the issues in temporal modelling, the addition of temporal notions to a relational database leads to several practical problems, as the attempt to model or introduce a time-variant object in a relational databse has an impact on the database consistency. Different approaches have been proposed \cite{Jensen91,Snodgrass84,Nascimento95} that consider time as a crisp notion. In recent works, some authors consider the necessity of allowing imprecision in temporal representation~\cite{Cru97,Garrido2009}.
%\\
%Most of these approaches use the concepts of fuzzy numbers and fuzzy intervals to represent temporal imprecision and considerable attention has thus been given to the problem of transforming two fuzzy numbers which represent the boundaries of an imprecise time interval into one fuzzy interval representing this interval. There are several proposals for this transformation, but some, like those presented in~\cite{Garrido2009}, are misleading. 

%In this work, an approach is proposed that allows users to query a Valid-time database using imperfectly defined temporal demands. This approach is based on the possibilistic framework found in \cite{Pon11}. In section \ref{sec:preliminaries}, some main concepts are explained and the part of the framework on which this work is based, is briefly presented.