%%Created by jPicEdt 1.4.1_03: mixed JPIC-XML/LaTeX format
%%Wed Jan 04 20:59:56 CET 2012
%%Begin JPIC-XML
%<?xml version="1.0" standalone="yes"?>
%<jpic x-min="5" x-max="85" y-min="-2.5" y-max="140" auto-bounding="true">
%<text text-vert-align= "center-v"
%	 anchor-point= "(5,65)"
%	 fill-style= "none"
%	 text-frame= "noframe"
%	 text-hor-align= "center-h"
%	 >
%I Before J
%</text>
%<multicurve fill-style= "none"
%	 points= "(50,75);(50,75);(80,75);(80,75)"
%	 />
%<multicurve fill-style= "none"
%	 points= "(25,65);(25,65);(45,65);(45,65)"
%	 />
%<text text-vert-align= "center-v"
%	 anchor-point= "(35,67.5)"
%	 fill-style= "none"
%	 text-frame= "noframe"
%	 text-hor-align= "center-h"
%	 >
%I
%</text>
%<text text-vert-align= "center-v"
%	 anchor-point= "(65,77.5)"
%	 fill-style= "none"
%	 text-frame= "noframe"
%	 text-hor-align= "center-h"
%	 >
%J
%</text>
%<text text-vert-align= "center-v"
%	 anchor-point= "(5,55)"
%	 fill-style= "none"
%	 text-frame= "noframe"
%	 text-hor-align= "center-h"
%	 >
%I Equal J
%</text>
%<text text-vert-align= "center-v"
%	 anchor-point= "(20,140)"
%	 fill-style= "none"
%	 text-frame= "noframe"
%	 text-hor-align= "center-h"
%	 >
%
%</text>
%<multicurve fill-style= "none"
%	 points= "(20,85);(20,85);(20,0);(20,0)"
%	 />
%<multicurve fill-style= "none"
%	 points= "(20,0);(20,0);(85,0);(85,0)"
%	 right-arrow= "head"
%	 />
%<text text-vert-align= "center-v"
%	 anchor-point= "(82.5,-2.5)"
%	 fill-style= "none"
%	 text-frame= "noframe"
%	 text-hor-align= "center-h"
%	 >
%Time
%</text>
%<multicurve fill-style= "none"
%	 points= "(25,55);(25,55);(45,55);(45,55)"
%	 />
%<text text-vert-align= "center-v"
%	 anchor-point= "(35,57.5)"
%	 fill-style= "none"
%	 text-frame= "noframe"
%	 text-hor-align= "center-h"
%	 >
%J
%</text>
%<text text-vert-align= "center-v"
%	 anchor-point= "(5,45)"
%	 fill-style= "none"
%	 text-frame= "noframe"
%	 text-hor-align= "center-h"
%	 >
%I Meets J
%</text>
%<text text-vert-align= "center-v"
%	 anchor-point= "(40,40)"
%	 fill-style= "none"
%	 text-frame= "noframe"
%	 text-hor-align= "center-h"
%	 >
%
%</text>
%<multicurve fill-style= "none"
%	 points= "(45,45);(45,45);(65,45);(65,45)"
%	 />
%<text text-vert-align= "center-v"
%	 anchor-point= "(55,47.5)"
%	 fill-style= "none"
%	 text-frame= "noframe"
%	 text-hor-align= "center-h"
%	 >
%J
%</text>
%<text text-vert-align= "center-v"
%	 anchor-point= "(7.5,35)"
%	 fill-style= "none"
%	 text-frame= "noframe"
%	 text-hor-align= "center-h"
%	 >
%I Overlaps J
%</text>
%<multicurve fill-style= "none"
%	 points= "(30,35);(30,35);(50,35);(50,35)"
%	 />
%<text text-vert-align= "center-v"
%	 anchor-point= "(40,37.5)"
%	 fill-style= "none"
%	 text-frame= "noframe"
%	 text-hor-align= "center-h"
%	 >
%J
%</text>
%<text text-vert-align= "center-v"
%	 anchor-point= "(5,25)"
%	 fill-style= "none"
%	 text-frame= "noframe"
%	 text-hor-align= "center-h"
%	 >
%I During J
%</text>
%<multicurve fill-style= "none"
%	 points= "(20,25);(20,25);(50,25);(50,25)"
%	 />
%<text text-vert-align= "center-v"
%	 anchor-point= "(35,27.5)"
%	 fill-style= "none"
%	 text-frame= "noframe"
%	 text-hor-align= "center-h"
%	 >
%J
%</text>
%<text text-vert-align= "center-v"
%	 anchor-point= "(5,15)"
%	 fill-style= "none"
%	 text-frame= "noframe"
%	 text-hor-align= "center-h"
%	 >
%I Starts J
%</text>
%<multicurve fill-style= "none"
%	 points= "(25,15);(25,15);(55,15);(55,15)"
%	 />
%<text text-vert-align= "center-v"
%	 anchor-point= "(7.5,5)"
%	 fill-style= "none"
%	 text-frame= "noframe"
%	 text-hor-align= "center-h"
%	 >
%I Finishes J
%</text>
%<text text-vert-align= "center-v"
%	 anchor-point= "(35,17.5)"
%	 fill-style= "none"
%	 text-frame= "noframe"
%	 text-hor-align= "center-h"
%	 >
%J
%</text>
%<text text-vert-align= "center-v"
%	 anchor-point= "(35,7.5)"
%	 fill-style= "none"
%	 text-frame= "noframe"
%	 text-hor-align= "center-h"
%	 >
%J
%</text>
%<multicurve fill-style= "none"
%	 points= "(25,5);(25,5);(45,5);(45,5)"
%	 />
%</jpic>
%%End JPIC-XML
%LaTeX-picture environment using emulated lines and arcs
%You can rescale the whole picture (to 80% for instance) by using the command \def\JPicScale{0.8}
\ifx\JPicScale\undefined\def\JPicScale{1}\fi
\unitlength \JPicScale mm
\begin{picture}(85,140)(0,0)
\put(5,65){\makebox(0,0)[cc]{I Before J}}

\linethickness{0.3mm}
\put(50,75){\line(1,0){30}}
\linethickness{0.3mm}
\put(25,65){\line(1,0){20}}
\put(35,67.5){\makebox(0,0)[cc]{I}}

\put(65,77.5){\makebox(0,0)[cc]{J}}

\put(5,55){\makebox(0,0)[cc]{I Equal J}}

\put(20,140){\makebox(0,0)[cc]{}}

\linethickness{0.3mm}
\put(20,0){\line(0,1){85}}
\linethickness{0.3mm}
\put(20,0){\line(1,0){65}}
\put(85,0){\vector(1,0){0.12}}
\put(82.5,-2.5){\makebox(0,0)[cc]{Time}}

\linethickness{0.3mm}
\put(25,55){\line(1,0){20}}
\put(35,57.5){\makebox(0,0)[cc]{J}}

\put(5,45){\makebox(0,0)[cc]{I Meets J}}

\put(40,40){\makebox(0,0)[cc]{}}

\linethickness{0.3mm}
\put(45,45){\line(1,0){20}}
\put(55,47.5){\makebox(0,0)[cc]{J}}

\put(7.5,35){\makebox(0,0)[cc]{I Overlaps J}}

\linethickness{0.3mm}
\put(30,35){\line(1,0){20}}
\put(40,37.5){\makebox(0,0)[cc]{J}}

\put(5,25){\makebox(0,0)[cc]{I During J}}

\linethickness{0.3mm}
\put(20,25){\line(1,0){30}}
\put(35,27.5){\makebox(0,0)[cc]{J}}

\put(5,15){\makebox(0,0)[cc]{I Starts J}}

\linethickness{0.3mm}
\put(25,15){\line(1,0){30}}
\put(7.5,5){\makebox(0,0)[cc]{I Finishes J}}

\put(35,17.5){\makebox(0,0)[cc]{J}}

\put(35,7.5){\makebox(0,0)[cc]{J}}

\linethickness{0.3mm}
\put(25,5){\line(1,0){20}}
\end{picture}
