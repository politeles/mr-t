%%%%%%%%%%%%%%%%%%%%%%%%%%%%%%%%%%%%%%%%%%%%%%%%%%%%%%%%%%%%%%%%%%%%%%%%%%%
%
% Fuzzy time domain
%
%%%%%%%%%%%%%%%%%%%%%%%%%%%%%%%%%%%%%%%%%%%%%%%%%%%%%%%%%%%%%%%%%%%%%%%%%%%

Humans usually deals with time in a imprecise way. e.g. `A few days ago we received this package'. This issue has been studied in the field of Artificial Intelligence ~\cite{Tre97} and language understanding~\cite{DeCaluwe:1997:FTI:285506.285516}. Some proposals ~\cite{knight1993},~\cite{Cru97},~\cite{nagypal2003} conclude that the best representation for incomplete temporal knowledge are time intervals instead of time points. This means that, as Allen proprosed in \cite{Allen83} the primitive units (the chronons) in the temporal system are intervals.

When modelling temporal knowledge, the following types of information might be found~\cite{nagypal2003}:

\begin{description}
\item[\emph{Uncertainty.}] The temporal specification of the event is uncertain. This is usually found in historical documents when some documents state contradictory facts about some event.


\item[\emph{Vagueness.}] Some of the temporal specifications are defined in a vague way. e.g. `during summer', `late night'. Note also that temporal specifications become vague when the granularity is changed ~\cite{Cru97},~\cite{nagypal2003} . e.g. `The book was written by July 2012' is usefull if we are interested in the month and the year but it turns into a incomplete expression when we try to obtain the exact day. It is possible to set the following classification~\cite{Dev98}:

\begin{itemize}
\item
Non-numerical indications. e.g. \emph{often, usually}.
\item
Approximative time indications. e.g. \emph{around 10 a.m.}
\item
Indications on temporal relations. e.g. \emph{after 12, during summer}.
\end{itemize}


\item[\emph{Subjectivity} or \emph{Ambiguity}.] The specification of a temporal fact may be affected by subjectivity.  It is possible to distinguish between:
\begin{itemize}
\item Historical periods~\cite{nagypal2003} like `late romanticism' or `early middle ages'. 
\item The interpretation of the temporal event~\cite{Dev98} depends on the side of the hearer. e.g. A speaker says `We will meet each other at six', the hearer has to choose between 6 p.m. or 6 a.m. The speaker, however knows exactly which one is meant. 
\end{itemize}
\end{description}




Another important task in the modelling of incomplete knowledge is when dealing with transitions among the different granularities.







%%%%%%%%%%%%%%%%%%%%%%%%%%%%%%%%%%%%%%%%%%%%%%%%%%%%%%%%%%%%%%%%%%%%%%%%%%%
%
% End fuzzy time domain
%
%%%%%%%%%%%%%%%%%%%%%%%%%%%%%%%%%%%%%%%%%%%%%%%%%%%%%%%%%%%%%%%%%%%%%%%%%%%%%