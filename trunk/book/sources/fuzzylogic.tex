%Possibility theory, like probability theory, deals with uncertainty about the outcome of an experiment. In probability theory, this uncertainty is caused by the \emph{variability} in the outcomes, while in possibility theory, the uncertainty is caused by \emph{incomplete knowledge} about the experiment. The quantification of confidence in a theory of uncertainty is achieved using a confidence measure\cite{Pon11}. In probability theory this is a measure of chance, in possibility theory, possibility and necessity measures are used.

%\begin{definition}
%Consider a set of outcomes $\Omega$. Let $\wp(\Omega)$ denote the powerset of $\Omega$ and let $A$ and $B$ be elements of $\wp(\Omega)$. A \emph{confidence measure on $\Omega$} is defined by a function
%	\begin{align}
%	g : \wp(\Omega) & \rightarrow \left[0,1\right]
%	\end{align}
%that satisfies
%	\begin{align}
%	g(\emptyset) &= 0 \\
%	g(\Omega) &= 1 	\label{NormalizationProperty} \\
%	A \subseteq B &\Rightarrow g(A) \leq g(B) \label{MonotonicityProperty}
%	\end{align}
%\end{definition}

%Both possibility measures and necessity measures are special cases of confidence measures.

%\begin{definition}
%Consider a confidence measure $\Pi$ on a set of outcomes $\Omega$. Let $J$ be a countable index set and let $\{ A_{j} | j \in J \wedge A_{j} \subseteq \Omega \}$ be a family of elements of $\wp(\Omega)$. $\Pi$ is now a \emph{possibility measure on $\Omega$} if it satisfies:
%	\begin{align}
%	\Pi\left(\bigcup_{j \in J} A_{j} \right) = \sup_{j \in J} \Pi(A_{j})
%	\end{align}
%\end{definition}

%In this work, the interpretation is as follows. The possibility of an event expresses how plausible the occurrence of the event seems to an observer of the experiment, given the (partial) knowledge of the observer about the experiment.

%Information on the possibility of distinct elements of the universe of discourse $\Omega$ can now be given by a \emph{possibility distribution} $\pi$ on $\Omega$, defined by:

%\begin{definition}
%Consider a possibility measure $\Pi$ on $\Omega$. A \emph{possibility distribution} $\pi$ on $\Omega$ underlying the possibility measure $\Pi$ is then a function defined by:
%	\begin{align}
%	\pi : \Omega \rightarrow \left[0, 1\right] : \pi(u) = \Pi(\{u\})
%	\end{align}
%\end{definition}

%\begin{definition}
%Consider a confidence measure $N$ on a set of outcomes $\Omega$. Let $J$ be a countable index set and let $\{ A_{j} | j \in J \wedge A_{j} \subseteq \Omega \}$ be a family of elements of $\wp(\Omega)$. $N$ is now a \emph{necessity measure} on $\Omega$ if it satisfies:
%	\begin{align}
%	N\left(\bigcap_{j \in J} A_{j} \right) = \inf_{j \in J} N(A_{j})
%	\end{align}
%\end{definition}

%In this work, the interpretation is as follows. The necessity of an event expresses how necessary the occurrence of the event seems to an observer of the experiment, given the (partial) knowledge of the observer about the experiment.

%Possibility and necessity measures are dual in the sense that:

%\begin{align}
%\forall A \subseteq \Omega : N(A) = 1 - \Pi(\bar{A})
%\end{align}

%Regarding interpretation, the above can be seen as: the degree to which an event is necessary is the degree to which every other possible event is not plausible.

\subsection*{\label{subsec:fuzzy-numbers}Fuzzy numbers and fuzzy intervals}
Dubois and Prade~\cite{Dubois1983} proposed the following definition of \emph{fuzzy interval}.
%\begin{definition}
A fuzzy interval is a fuzzy set $M$ on the set of real numbers $\mathbb{R}$ such that:
\begin{eqnarray}
\forall (u,v)\in\mathbb{R}^2:&\\
\nonumber
\forall w \in [u,v]:&\mu_M(w) \geq\min(\mu_M(u),\mu_M(v))  \\
\exists m \in \mathbb{R}:& \mu_M(m)=1 
\end{eqnarray}
%\end{definition}
If $m$ is unique, then $M$ is referred to as a \emph{fuzzy number}, instead of a \emph{fuzzy interval}. In other words, if the core of a fuzzy interval is a singleton, it can be seen as a fuzzy number. In their work, Dubois and Prade propose four different functions (two possibility and two necessity functions) to asses the position of a fuzzy number N relative to  a fuzzy number M taken as a reference.

The most convenient representation for the membership function of a fuzzy number is a triangular function (fig. \ref{fig:triangular}). The membership function $\mu_M$ for the fuzzy set $M$ has also the properties of convexity and normalization. Three values represent a triangular function: 
\begin{itemize}
\item
$D$ is the singleton value in the core of $\mu_M(x)$.
\item
$D-a$ is the lower value in the support of $\mu_M(x)$. 
\item
$D+b$ is the upper value in the support of $\mu_M(x)$.
\end{itemize}

\begin{figure}
\centering
%%Created by jPicEdt 1.4.1_03: mixed JPIC-XML/LaTeX format
%%Thu Jan 12 17:26:56 CET 2012
%%Begin JPIC-XML
%<?xml version="1.0" standalone="yes"?>
%<jpic x-min="-2.5" x-max="60" y-min="-2" y-max="32.5" auto-bounding="true">
%<multicurve right-arrow= "head"
%	 fill-style= "none"
%	 points= "(0,0);(0,0);(55,0);(55,0)"
%	 />
%<multicurve right-arrow= "head"
%	 fill-style= "none"
%	 points= "(0,0);(0,0);(0,30);(0,30)"
%	 />
%<text right-arrow= "head"
%	 fill-style= "none"
%	 text-vert-align= "center-v"
%	 anchor-point= "(-2.5,27.5)"
%	 text-frame= "noframe"
%	 text-hor-align= "center-h"
%	 >
%1
%</text>
%<text right-arrow= "head"
%	 fill-style= "none"
%	 text-vert-align= "center-v"
%	 anchor-point= "(-2.5,0)"
%	 text-frame= "noframe"
%	 text-hor-align= "center-h"
%	 >
%0
%</text>
%<text right-arrow= "head"
%	 fill-style= "none"
%	 text-vert-align= "center-v"
%	 anchor-point= "(15,32.5)"
%	 text-rotation= "135"
%	 text-frame= "noframe"
%	 text-hor-align= "center-h"
%	 >
%Membership Degree
%</text>
%<multicurve fill-style= "none"
%	 points= "(4,0);(4,0);(15,27.5);(15,27.5)"
%	 />
%<multicurve fill-style= "none"
%	 points= "(15,27.5);(15,27.5);(42,0);(42,0)"
%	 />
%<text fill-style= "none"
%	 text-vert-align= "center-v"
%	 anchor-point= "(60,0)"
%	 text-frame= "noframe"
%	 text-hor-align= "center-h"
%	 >
%Time
%</text>
%<text fill-style= "none"
%	 text-vert-align= "center-v"
%	 anchor-point= "(16,-2)"
%	 text-frame= "noframe"
%	 text-hor-align= "center-h"
%	 >
%$D$
%</text>
%<text fill-style= "none"
%	 text-vert-align= "center-v"
%	 anchor-point= "(4,-2)"
%	 text-frame= "noframe"
%	 text-hor-align= "center-h"
%	 >
%$D-a$
%</text>
%<text fill-style= "none"
%	 text-vert-align= "center-v"
%	 anchor-point= "(42,-2)"
%	 text-frame= "noframe"
%	 text-hor-align= "center-h"
%	 >
%$D+b$
%</text>
%</jpic>
%%End JPIC-XML
%LaTeX-picture environment using emulated lines and arcs
%You can rescale the whole picture (to 80% for instance) by using the command \def\JPicScale{0.8}
\ifx\JPicScale\undefined\def\JPicScale{1}\fi
\unitlength \JPicScale mm
\begin{picture}(60,32.5)(0,0)
\linethickness{0.3mm}
\put(0,0){\line(1,0){55}}
\put(55,0){\vector(1,0){0.12}}
\linethickness{0.3mm}
\put(0,0){\line(0,1){30}}
\put(0,30){\vector(0,1){0.12}}
\put(-2.5,27.5){\makebox(0,0)[cc]{1}}

\put(-2.5,0){\makebox(0,0)[cc]{0}}

\put(15,32.5){\makebox(0,0)[cc]{Membership Degree}}

\linethickness{0.3mm}
\multiput(4,0)(0.12,0.3){92}{\line(0,1){0.3}}
\linethickness{0.3mm}
\multiput(15,27.5)(0.12,-0.12){225}{\line(0,-1){0.12}}
\put(60,0){\makebox(0,0)[cc]{Time}}

\put(16,-2){\makebox(0,0)[cc]{$D$}}

\put(4,-2){\makebox(0,0)[cc]{$D-a$}}

\put(42,-2){\makebox(0,0)[cc]{$D+b$}}

\end{picture}

\caption{Triangular possibility distribution. }
\label{fig:triangular}
\end{figure}

Note that the values $a$ and $b$ are values in the underlying ordered domain. E.g. $(a,b) \in \mathbb{R}^2$. The notation for a triangular function adopted here is $[D,a,b]$.

The most simple representation for the membership function of a fuzzy interval is a trapezoidal function. This membership function $\mu_T$ for the fuzzy interval $T$ is convex and normalized. Four values represent a trapezoidal membership function (figure  \ref{fig:trapezoidal}):
 $\left[\alpha,\ \beta,\ \gamma,\ \delta\right]$. The membership function is:

\begin{align}
\mu_T(x)
\begin{cases}
1 & \mbox{ if } x \in [\beta,\gamma] \\
0 & \mbox{ if } x > \delta \vee x < \alpha \\
\frac{x-\alpha}{\beta - \alpha} & \mbox{ if } x \in [\alpha,\beta[ \\
\frac{\delta -x}{\delta - \gamma} & \mbox{ if } x \in ]\gamma,\delta] \\
\end{cases}
\end{align}

 
\begin{figure}
\centering
%LaTeX with PSTricks extensions
%%Creator: inkscape 0.47
%%Please note this file requires PSTricks extensions
\psset{xunit=.5pt,yunit=.5pt,runit=.5pt}
\begin{pspicture}(744.09448242,1052.36218262)
{
\newrgbcolor{curcolor}{0 0 0}
\pscustom[linewidth=1.17799997,linecolor=curcolor]
{
\newpath
\moveto(640,849.08323262)
\lineto(-81.57238,849.08323262)
\lineto(-81.57238,1109.08323262)
\lineto(-81.57238,1109.08323262)
}
}
{
\newrgbcolor{curcolor}{0 0 0}
\pscustom[linestyle=none,fillstyle=solid,fillcolor=curcolor]
{
\newpath
\moveto(-98.15238,1049.88323)
\curveto(-98.13238002,1049.62323026)(-98.07237914,1048.96323)(-97.21238,1048.96323)
\curveto(-96.33238088,1048.96323)(-96.27237998,1049.60323028)(-96.25238,1049.88323)
\lineto(-96.25238,1062.36323)
\curveto(-96.27237998,1062.60322976)(-96.29238078,1063.10323)(-97.07238,1063.10323)
\curveto(-97.53237954,1063.10323)(-97.63238016,1062.86322964)(-97.79238,1062.50323)
\curveto(-98.31237948,1061.14323136)(-99.1123818,1060.50322976)(-100.91238,1060.26323)
\curveto(-101.25237966,1060.20323006)(-101.49238,1060.14322952)(-101.49238,1059.66323)
\curveto(-101.49238,1058.92323074)(-101.07237974,1058.92322998)(-100.81238,1058.90323)
\lineto(-98.15238,1058.90323)
\lineto(-98.15238,1049.88323)
}
}
{
\newrgbcolor{curcolor}{0 0 0}
\pscustom[linestyle=none,fillstyle=solid,fillcolor=curcolor]
{
\newpath
\moveto(-92.89238,856.16323)
\curveto(-92.89238,857.24322892)(-92.95238026,858.30323104)(-93.21238,859.34323)
\curveto(-93.75237946,861.5432278)(-95.15238264,863.18323)(-97.79238,863.18323)
\curveto(-98.53237926,863.18323)(-100.59238132,863.12322788)(-101.91238,861.00323)
\curveto(-102.99237892,859.26323174)(-103.01238,856.98322902)(-103.01238,856.00323)
\curveto(-103.01238,854.7032313)(-103.01237792,851.50322834)(-100.93238,849.84323)
\curveto(-100.11238082,849.16323068)(-99.07237894,848.88323)(-98.01238,848.88323)
\curveto(-92.89238512,848.88323)(-92.89238,854.7632314)(-92.89238,856.16323)
\moveto(-97.93238,850.48323)
\curveto(-98.87237906,850.48323)(-99.89238058,850.92323124)(-100.47238,852.16323)
\curveto(-100.99237948,853.28322888)(-101.07238,854.94323096)(-101.07238,855.90323)
\curveto(-101.07238,858.34322756)(-100.71237714,861.62323)(-97.85238,861.62323)
\curveto(-94.83238302,861.62323)(-94.83238,857.66322862)(-94.83238,856.28323)
\curveto(-94.83238,853.04323324)(-95.39238158,851.2432294)(-96.97238,850.64323)
\curveto(-97.29237968,850.5432301)(-97.61238032,850.48323)(-97.93238,850.48323)
}
}
{
\newrgbcolor{curcolor}{0 0 0}
\pscustom[linestyle=none,fillstyle=solid,fillcolor=curcolor]
{
}
}
{
\newrgbcolor{curcolor}{0 0 0}
\pscustom[linestyle=none,fillstyle=solid,fillcolor=curcolor]
{
\newpath
\moveto(-268.06799993,1081.09200856)
\curveto(-268.06799993,1080.7560089)(-267.95599873,1079.83200856)(-266.75199993,1079.83200856)
\curveto(-266.08000061,1079.83200856)(-265.71599982,1080.22400882)(-265.60399993,1080.47600856)
\curveto(-265.52000002,1080.67200837)(-265.46399993,1080.89600876)(-265.46399993,1081.09200856)
\lineto(-265.46399993,1087.50400856)
\curveto(-265.46399993,1089.01600705)(-265.43599929,1090.05200957)(-264.79199993,1091.06000856)
\curveto(-264.12000061,1092.1240075)(-263.08399879,1092.54400856)(-261.93599993,1092.54400856)
\curveto(-259.6680022,1092.54400856)(-259.44399988,1090.64000809)(-259.38799993,1090.16400856)
\curveto(-259.33199999,1089.74400898)(-259.33199993,1089.49200725)(-259.33199993,1088.17600856)
\lineto(-259.33199993,1081.09200856)
\curveto(-259.33199993,1080.70000896)(-259.2479994,1080.25200831)(-258.71599993,1080.00000856)
\curveto(-258.49200016,1079.88800868)(-258.26799971,1079.83200856)(-258.04399993,1079.83200856)
\curveto(-257.90400007,1079.83200856)(-257.62399968,1079.8600087)(-257.37199993,1080.00000856)
\curveto(-256.84000047,1080.25200831)(-256.75599993,1080.70000896)(-256.75599993,1081.09200856)
\lineto(-256.75599993,1088.03600856)
\curveto(-256.75599993,1089.21200739)(-256.75599932,1090.13600949)(-256.13999993,1091.06000856)
\curveto(-255.69200038,1091.73200789)(-254.79599842,1092.54400856)(-253.28399993,1092.54400856)
\curveto(-250.73600248,1092.54400856)(-250.65199993,1090.38800764)(-250.65199993,1089.46400856)
\curveto(-250.65199993,1089.01600901)(-250.62399993,1088.23200772)(-250.62399993,1087.39200856)
\lineto(-250.62399993,1081.12000856)
\curveto(-250.62399993,1080.70000898)(-250.53999937,1080.25200831)(-249.97999993,1080.00000856)
\curveto(-249.78400013,1079.88800868)(-249.55999971,1079.83200856)(-249.33599993,1079.83200856)
\curveto(-248.13200114,1079.83200856)(-248.01999993,1080.75600893)(-248.01999993,1081.12000856)
\lineto(-248.01999993,1089.49200856)
\curveto(-248.01999993,1090.89200716)(-248.07600089,1092.18000971)(-249.02799993,1093.32800856)
\curveto(-250.0639989,1094.56000733)(-251.49200133,1094.81200856)(-252.89199993,1094.81200856)
\curveto(-255.01999781,1094.81200856)(-256.33600103,1093.86000725)(-257.42799993,1092.54400856)
\curveto(-257.62399974,1092.93600817)(-258.01600094,1093.7200091)(-259.02399993,1094.25200856)
\curveto(-259.75199921,1094.67200814)(-260.59200077,1094.81200856)(-261.43199993,1094.81200856)
\curveto(-263.19599817,1094.81200856)(-264.42800103,1094.19600722)(-265.51999993,1092.85200856)
\curveto(-265.51999993,1093.80400761)(-265.52000052,1094.25200884)(-266.10799993,1094.53200856)
\curveto(-266.49999954,1094.75600834)(-267.14400024,1094.7000084)(-267.45199993,1094.53200856)
\curveto(-267.9839994,1094.28000882)(-268.06799993,1093.86000817)(-268.06799993,1093.46800856)
\lineto(-268.06799993,1081.09200856)
}
}
{
\newrgbcolor{curcolor}{0 0 0}
\pscustom[linestyle=none,fillstyle=solid,fillcolor=curcolor]
{
\newpath
\moveto(-232.82956243,1086.55200856)
\curveto(-232.4655628,1086.55200856)(-231.56956243,1086.63600974)(-231.56956243,1087.81200856)
\curveto(-231.56956243,1089.380007)(-232.04556336,1091.20000985)(-232.96956243,1092.48800856)
\curveto(-233.83756157,1093.66400739)(-234.92956369,1094.28000887)(-236.18956243,1094.58800856)
\curveto(-236.77756185,1094.72800842)(-237.39356305,1094.78400856)(-238.00956243,1094.78400856)
\curveto(-242.96555748,1094.78400856)(-244.86956243,1090.66800504)(-244.86956243,1087.14000856)
\curveto(-244.86956243,1085.12401058)(-244.28156117,1083.10800714)(-243.02156243,1081.68000856)
\curveto(-241.59356386,1080.08401016)(-239.74556061,1079.72000856)(-237.92556243,1079.72000856)
\curveto(-237.14156322,1079.72000856)(-234.6215605,1079.72001075)(-232.68956243,1081.90400856)
\curveto(-232.38156274,1082.24000823)(-231.90556232,1082.8560091)(-231.79356243,1083.38800856)
\curveto(-231.76556246,1083.47200848)(-231.76556243,1083.55600868)(-231.76556243,1083.66800856)
\curveto(-231.76556243,1084.48000775)(-232.60556305,1084.84400856)(-233.22156243,1084.84400856)
\curveto(-233.83756182,1084.84400856)(-233.97756288,1084.56400781)(-234.42556243,1083.80800856)
\curveto(-234.92956193,1083.02400935)(-235.71356465,1081.87600856)(-237.92556243,1081.87600856)
\curveto(-239.04556131,1081.87600856)(-240.19356339,1082.07200974)(-241.14556243,1083.24800856)
\curveto(-241.90156168,1084.17200764)(-242.18156249,1085.29200982)(-242.23756243,1086.55200856)
\lineto(-232.82956243,1086.55200856)
\moveto(-242.20956243,1088.65200856)
\curveto(-242.06956257,1089.29600792)(-241.81756131,1090.64000954)(-240.69756243,1091.62000856)
\curveto(-239.71756341,1092.4880077)(-238.56956185,1092.54400856)(-237.98156243,1092.54400856)
\curveto(-235.32156509,1092.54400856)(-234.31356232,1090.50000672)(-234.20156243,1088.65200856)
\lineto(-242.20956243,1088.65200856)
}
}
{
\newrgbcolor{curcolor}{0 0 0}
\pscustom[linestyle=none,fillstyle=solid,fillcolor=curcolor]
{
\newpath
\moveto(-228.41956243,1081.09200856)
\curveto(-228.41956243,1080.7560089)(-228.30756123,1079.83200856)(-227.10356243,1079.83200856)
\curveto(-226.43156311,1079.83200856)(-226.06756232,1080.22400882)(-225.95556243,1080.47600856)
\curveto(-225.87156252,1080.67200837)(-225.81556243,1080.89600876)(-225.81556243,1081.09200856)
\lineto(-225.81556243,1087.50400856)
\curveto(-225.81556243,1089.01600705)(-225.78756179,1090.05200957)(-225.14356243,1091.06000856)
\curveto(-224.47156311,1092.1240075)(-223.43556129,1092.54400856)(-222.28756243,1092.54400856)
\curveto(-220.0195647,1092.54400856)(-219.79556238,1090.64000809)(-219.73956243,1090.16400856)
\curveto(-219.68356249,1089.74400898)(-219.68356243,1089.49200725)(-219.68356243,1088.17600856)
\lineto(-219.68356243,1081.09200856)
\curveto(-219.68356243,1080.70000896)(-219.5995619,1080.25200831)(-219.06756243,1080.00000856)
\curveto(-218.84356266,1079.88800868)(-218.61956221,1079.83200856)(-218.39556243,1079.83200856)
\curveto(-218.25556257,1079.83200856)(-217.97556218,1079.8600087)(-217.72356243,1080.00000856)
\curveto(-217.19156297,1080.25200831)(-217.10756243,1080.70000896)(-217.10756243,1081.09200856)
\lineto(-217.10756243,1088.03600856)
\curveto(-217.10756243,1089.21200739)(-217.10756182,1090.13600949)(-216.49156243,1091.06000856)
\curveto(-216.04356288,1091.73200789)(-215.14756092,1092.54400856)(-213.63556243,1092.54400856)
\curveto(-211.08756498,1092.54400856)(-211.00356243,1090.38800764)(-211.00356243,1089.46400856)
\curveto(-211.00356243,1089.01600901)(-210.97556243,1088.23200772)(-210.97556243,1087.39200856)
\lineto(-210.97556243,1081.12000856)
\curveto(-210.97556243,1080.70000898)(-210.89156187,1080.25200831)(-210.33156243,1080.00000856)
\curveto(-210.13556263,1079.88800868)(-209.91156221,1079.83200856)(-209.68756243,1079.83200856)
\curveto(-208.48356364,1079.83200856)(-208.37156243,1080.75600893)(-208.37156243,1081.12000856)
\lineto(-208.37156243,1089.49200856)
\curveto(-208.37156243,1090.89200716)(-208.42756339,1092.18000971)(-209.37956243,1093.32800856)
\curveto(-210.4155614,1094.56000733)(-211.84356383,1094.81200856)(-213.24356243,1094.81200856)
\curveto(-215.37156031,1094.81200856)(-216.68756353,1093.86000725)(-217.77956243,1092.54400856)
\curveto(-217.97556224,1092.93600817)(-218.36756344,1093.7200091)(-219.37556243,1094.25200856)
\curveto(-220.10356171,1094.67200814)(-220.94356327,1094.81200856)(-221.78356243,1094.81200856)
\curveto(-223.54756067,1094.81200856)(-224.77956353,1094.19600722)(-225.87156243,1092.85200856)
\curveto(-225.87156243,1093.80400761)(-225.87156302,1094.25200884)(-226.45956243,1094.53200856)
\curveto(-226.85156204,1094.75600834)(-227.49556274,1094.7000084)(-227.80356243,1094.53200856)
\curveto(-228.3355619,1094.28000882)(-228.41956243,1093.86000817)(-228.41956243,1093.46800856)
\lineto(-228.41956243,1081.09200856)
}
}
{
\newrgbcolor{curcolor}{0 0 0}
\pscustom[linestyle=none,fillstyle=solid,fillcolor=curcolor]
{
\newpath
\moveto(-202.00112493,1098.67600856)
\curveto(-202.00112493,1099.09600814)(-202.08512547,1099.51600884)(-202.61712493,1099.79600856)
\curveto(-202.81312474,1099.88000848)(-203.03712516,1099.93600856)(-203.26112493,1099.93600856)
\curveto(-203.93312426,1099.93600856)(-204.26912505,1099.54400834)(-204.38112493,1099.32000856)
\curveto(-204.46512485,1099.12400876)(-204.52112493,1098.90000834)(-204.52112493,1098.67600856)
\lineto(-204.52112493,1081.06400856)
\curveto(-204.52112493,1080.64400898)(-204.4371244,1080.25200828)(-203.90512493,1079.97200856)
\curveto(-203.70912513,1079.86000868)(-203.48512468,1079.83200856)(-203.23312493,1079.83200856)
\curveto(-202.61712555,1079.83200856)(-202.28112479,1080.16800882)(-202.14112493,1080.42000856)
\curveto(-202.00112507,1080.67200831)(-202.00112493,1080.92400921)(-202.00112493,1081.56800856)
\curveto(-201.16112577,1080.7280094)(-200.01312253,1079.69200856)(-197.60512493,1079.69200856)
\curveto(-193.88112866,1079.69200856)(-191.36112493,1082.52001327)(-191.36112493,1087.22400856)
\curveto(-191.36112493,1090.16400562)(-192.45312992,1094.78400856)(-197.43712493,1094.78400856)
\curveto(-200.37712199,1094.78400856)(-201.6371253,1093.16000809)(-202.00112493,1092.68400856)
\lineto(-202.00112493,1098.67600856)
\moveto(-194.04912493,1087.19600856)
\curveto(-194.04912493,1084.08801167)(-194.97312619,1082.77200795)(-196.23312493,1082.15600856)
\curveto(-196.7651244,1081.90400882)(-197.35312555,1081.79200856)(-197.96912493,1081.79200856)
\curveto(-201.3011216,1081.79200856)(-202.02912493,1085.09601078)(-202.02912493,1087.30800856)
\curveto(-202.02912493,1089.24000663)(-201.49712325,1091.28400949)(-199.81712493,1092.20800856)
\curveto(-199.22912552,1092.51600826)(-198.55712426,1092.65600856)(-197.88512493,1092.65600856)
\curveto(-194.07712874,1092.65600856)(-194.04912493,1088.37200739)(-194.04912493,1087.19600856)
}
}
{
\newrgbcolor{curcolor}{0 0 0}
\pscustom[linestyle=none,fillstyle=solid,fillcolor=curcolor]
{
\newpath
\moveto(-176.82956243,1086.55200856)
\curveto(-176.4655628,1086.55200856)(-175.56956243,1086.63600974)(-175.56956243,1087.81200856)
\curveto(-175.56956243,1089.380007)(-176.04556336,1091.20000985)(-176.96956243,1092.48800856)
\curveto(-177.83756157,1093.66400739)(-178.92956369,1094.28000887)(-180.18956243,1094.58800856)
\curveto(-180.77756185,1094.72800842)(-181.39356305,1094.78400856)(-182.00956243,1094.78400856)
\curveto(-186.96555748,1094.78400856)(-188.86956243,1090.66800504)(-188.86956243,1087.14000856)
\curveto(-188.86956243,1085.12401058)(-188.28156117,1083.10800714)(-187.02156243,1081.68000856)
\curveto(-185.59356386,1080.08401016)(-183.74556061,1079.72000856)(-181.92556243,1079.72000856)
\curveto(-181.14156322,1079.72000856)(-178.6215605,1079.72001075)(-176.68956243,1081.90400856)
\curveto(-176.38156274,1082.24000823)(-175.90556232,1082.8560091)(-175.79356243,1083.38800856)
\curveto(-175.76556246,1083.47200848)(-175.76556243,1083.55600868)(-175.76556243,1083.66800856)
\curveto(-175.76556243,1084.48000775)(-176.60556305,1084.84400856)(-177.22156243,1084.84400856)
\curveto(-177.83756182,1084.84400856)(-177.97756288,1084.56400781)(-178.42556243,1083.80800856)
\curveto(-178.92956193,1083.02400935)(-179.71356465,1081.87600856)(-181.92556243,1081.87600856)
\curveto(-183.04556131,1081.87600856)(-184.19356339,1082.07200974)(-185.14556243,1083.24800856)
\curveto(-185.90156168,1084.17200764)(-186.18156249,1085.29200982)(-186.23756243,1086.55200856)
\lineto(-176.82956243,1086.55200856)
\moveto(-186.20956243,1088.65200856)
\curveto(-186.06956257,1089.29600792)(-185.81756131,1090.64000954)(-184.69756243,1091.62000856)
\curveto(-183.71756341,1092.4880077)(-182.56956185,1092.54400856)(-181.98156243,1092.54400856)
\curveto(-179.32156509,1092.54400856)(-178.31356232,1090.50000672)(-178.20156243,1088.65200856)
\lineto(-186.20956243,1088.65200856)
}
}
{
\newrgbcolor{curcolor}{0 0 0}
\pscustom[linestyle=none,fillstyle=solid,fillcolor=curcolor]
{
\newpath
\moveto(-169.98356243,1093.49600856)
\curveto(-169.98356243,1093.83200823)(-170.09556353,1094.67200856)(-171.18756243,1094.67200856)
\curveto(-172.30756131,1094.67200856)(-172.41956243,1093.83200823)(-172.41956243,1093.49600856)
\lineto(-172.41956243,1081.00800856)
\curveto(-172.41956243,1080.61600896)(-172.30756129,1079.83200856)(-171.15956243,1079.83200856)
\curveto(-169.95556364,1079.83200856)(-169.87156243,1080.56000901)(-169.87156243,1081.00800856)
\lineto(-169.87156243,1086.58000856)
\curveto(-169.87156243,1088.26000688)(-169.87156059,1090.22000985)(-168.02356243,1091.50800856)
\curveto(-167.26756319,1091.98400809)(-166.53956173,1092.09600868)(-165.83956243,1092.20800856)
\curveto(-165.44756283,1092.26400851)(-164.66356243,1092.43200954)(-164.66356243,1093.41200856)
\curveto(-164.66356243,1094.6720073)(-165.89556274,1094.67200856)(-166.20356243,1094.67200856)
\curveto(-167.85556078,1094.67200856)(-169.45156297,1093.63600697)(-169.98356243,1092.04000856)
\lineto(-169.98356243,1093.49600856)
}
}
{
\newrgbcolor{curcolor}{0 0 0}
\pscustom[linestyle=none,fillstyle=solid,fillcolor=curcolor]
{
\newpath
\moveto(-157.98512493,1088.93200856)
\curveto(-159.27312365,1089.32400817)(-160.39312493,1089.66000985)(-160.39312493,1090.94800856)
\curveto(-160.39312493,1091.56400795)(-160.05712446,1092.12400887)(-159.58112493,1092.43200856)
\curveto(-159.04912547,1092.7960082)(-158.2931246,1092.82400856)(-157.95712493,1092.82400856)
\curveto(-155.74512715,1092.82400856)(-154.98912468,1091.50800809)(-154.73712493,1091.03200856)
\curveto(-154.31712535,1090.22000938)(-154.14912432,1089.91200856)(-153.53312493,1089.91200856)
\curveto(-153.11312535,1089.91200856)(-152.35712493,1090.1640094)(-152.35712493,1091.00400856)
\curveto(-152.35712493,1091.564008)(-153.11312967,1094.78400856)(-157.84512493,1094.78400856)
\curveto(-162.66112012,1094.78400856)(-162.94112493,1091.34000784)(-162.94112493,1090.61200856)
\curveto(-162.94112493,1087.92401125)(-160.56112339,1087.14000806)(-159.02112493,1086.63600856)
\lineto(-157.39712493,1086.16000856)
\curveto(-155.96912636,1085.74000898)(-154.68112493,1085.37600711)(-154.68112493,1083.92000856)
\curveto(-154.68112493,1082.60400988)(-155.91312659,1081.73600856)(-157.56512493,1081.73600856)
\curveto(-158.74112376,1081.73600856)(-159.91712558,1082.12800954)(-160.56112493,1083.10800856)
\curveto(-160.75712474,1083.41600826)(-160.78512533,1083.52800963)(-161.17712493,1084.59200856)
\curveto(-161.31712479,1084.87200828)(-161.48512558,1085.20800856)(-162.12912493,1085.20800856)
\curveto(-162.54912451,1085.20800856)(-163.50112493,1084.98400761)(-163.50112493,1084.03200856)
\curveto(-163.50112493,1083.64000896)(-163.24912365,1082.10000747)(-161.96112493,1081.00800856)
\curveto(-160.53312636,1079.80400977)(-158.54512412,1079.72000856)(-157.73312493,1079.72000856)
\curveto(-156.44512622,1079.72000856)(-154.96112387,1079.97200929)(-153.89712493,1080.70000856)
\curveto(-153.05712577,1081.260008)(-152.13312493,1082.35201036)(-152.13312493,1084.14400856)
\curveto(-152.13312493,1087.19600551)(-154.84912645,1087.98000901)(-156.36112493,1088.42800856)
\lineto(-157.98512493,1088.93200856)
}
}
{
\newrgbcolor{curcolor}{0 0 0}
\pscustom[linestyle=none,fillstyle=solid,fillcolor=curcolor]
{
\newpath
\moveto(-146.51868743,1098.67600856)
\curveto(-146.51868743,1099.01200823)(-146.65868861,1099.93600856)(-147.83468743,1099.93600856)
\curveto(-149.01068626,1099.93600856)(-149.12268743,1099.0120082)(-149.12268743,1098.64800856)
\lineto(-149.12268743,1081.09200856)
\curveto(-149.12268743,1080.72800893)(-149.01068626,1079.83200856)(-147.83468743,1079.83200856)
\curveto(-147.13468813,1079.83200856)(-146.79868732,1080.22400882)(-146.68668743,1080.47600856)
\curveto(-146.60268752,1080.6440084)(-146.54668743,1080.86800879)(-146.54668743,1081.09200856)
\lineto(-146.54668743,1087.70000856)
\curveto(-146.54668743,1089.01600725)(-146.49068643,1090.22000963)(-145.48268743,1091.28400856)
\curveto(-144.83868808,1091.95600789)(-143.80268623,1092.46000856)(-142.59868743,1092.46000856)
\curveto(-142.15068788,1092.46000856)(-140.86268679,1092.43200736)(-140.21868743,1091.22800856)
\curveto(-139.88268777,1090.55600924)(-139.82668743,1089.82800683)(-139.82668743,1088.09200856)
\lineto(-139.82668743,1081.09200856)
\curveto(-139.82668743,1080.72800893)(-139.71468626,1079.83200856)(-138.53868743,1079.83200856)
\curveto(-137.83868813,1079.83200856)(-137.50268732,1080.22400882)(-137.39068743,1080.47600856)
\curveto(-137.30668752,1080.6440084)(-137.25068743,1080.86800879)(-137.25068743,1081.09200856)
\lineto(-137.25068743,1088.87600856)
\curveto(-137.25068743,1090.22000722)(-137.25068833,1091.84400982)(-138.14668743,1093.10400856)
\curveto(-138.37068721,1093.41200826)(-139.40669009,1094.78400856)(-142.06668743,1094.78400856)
\curveto(-143.71868578,1094.78400856)(-145.51068844,1094.19600722)(-146.51868743,1092.85200856)
\lineto(-146.51868743,1098.67600856)
}
}
{
\newrgbcolor{curcolor}{0 0 0}
\pscustom[linestyle=none,fillstyle=solid,fillcolor=curcolor]
{
\newpath
\moveto(-130.82468743,1093.44000856)
\curveto(-130.82468743,1093.77600823)(-130.93668861,1094.67200856)(-132.11268743,1094.67200856)
\curveto(-133.26068629,1094.67200856)(-133.37268743,1093.77600823)(-133.37268743,1093.44000856)
\lineto(-133.37268743,1081.09200856)
\curveto(-133.37268743,1080.30800935)(-132.84068668,1079.83200856)(-132.08468743,1079.83200856)
\curveto(-131.41268811,1079.83200856)(-131.07668732,1080.22400882)(-130.96468743,1080.47600856)
\curveto(-130.88068752,1080.6440084)(-130.82468743,1080.86800879)(-130.82468743,1081.09200856)
\lineto(-130.82468743,1093.44000856)
\moveto(-133.68068743,1098.34000856)
\curveto(-133.68068743,1097.47200943)(-133.12068643,1096.80000856)(-132.11268743,1096.80000856)
\curveto(-131.16068839,1096.80000856)(-130.54468743,1097.38800952)(-130.54468743,1098.34000856)
\curveto(-130.54468743,1099.37600753)(-131.30068825,1099.93600856)(-132.11268743,1099.93600856)
\curveto(-132.98068657,1099.93600856)(-133.68068743,1099.34800756)(-133.68068743,1098.34000856)
}
}
{
\newrgbcolor{curcolor}{0 0 0}
\pscustom[linestyle=none,fillstyle=solid,fillcolor=curcolor]
{
\newpath
\moveto(-126.97424993,1075.85600856)
\curveto(-126.97424993,1075.43600898)(-126.89024937,1075.01600826)(-126.33024993,1074.70800856)
\curveto(-126.13425013,1074.62400865)(-125.91024971,1074.56800856)(-125.68624993,1074.56800856)
\curveto(-124.48225114,1074.56800856)(-124.37024993,1075.49200893)(-124.37024993,1075.85600856)
\lineto(-124.37024993,1081.70800856)
\curveto(-123.16625114,1080.64400963)(-122.07424761,1079.72000856)(-119.75024993,1079.72000856)
\curveto(-115.32625436,1079.72000856)(-113.81424993,1083.80801204)(-113.81424993,1087.28000856)
\curveto(-113.81424993,1092.06800378)(-116.47425352,1094.81200856)(-120.05824993,1094.81200856)
\curveto(-121.85024814,1094.81200856)(-123.25025105,1094.19600722)(-124.37024993,1092.85200856)
\curveto(-124.37024993,1093.69200772)(-124.37025058,1094.22400887)(-125.01424993,1094.53200856)
\curveto(-125.29424965,1094.67200842)(-125.57425005,1094.67200856)(-125.68624993,1094.67200856)
\curveto(-126.86224876,1094.67200856)(-126.97424993,1093.7760082)(-126.97424993,1093.41200856)
\lineto(-126.97424993,1075.85600856)
\moveto(-116.47424993,1087.22400856)
\curveto(-116.47424993,1085.90800988)(-116.64225072,1084.3400075)(-117.42624993,1083.27600856)
\curveto(-117.9582494,1082.54800929)(-118.91025142,1081.79200856)(-120.39424993,1081.79200856)
\curveto(-123.08224725,1081.79200856)(-124.42624993,1084.17201181)(-124.42624993,1087.42000856)
\curveto(-124.42624993,1090.64000534)(-122.9982473,1092.65600856)(-120.36624993,1092.65600856)
\curveto(-116.58625371,1092.65600856)(-116.47424993,1088.70800708)(-116.47424993,1087.22400856)
}
}
{
\newrgbcolor{curcolor}{0 0 0}
\pscustom[linestyle=none,fillstyle=solid,fillcolor=curcolor]
{
\newpath
\moveto(-255.60799993,1063.67600856)
\curveto(-255.60799993,1064.09600814)(-255.69200047,1064.51600884)(-256.22399993,1064.79600856)
\curveto(-256.50399965,1064.93600842)(-256.75600005,1064.93600856)(-256.86799993,1064.93600856)
\curveto(-257.45599935,1064.93600856)(-257.82000007,1064.60000828)(-257.95999993,1064.32000856)
\curveto(-258.04399985,1064.12400876)(-258.09999993,1063.90000834)(-258.09999993,1063.67600856)
\lineto(-258.09999993,1057.62800856)
\curveto(-258.46399957,1058.10400809)(-259.72400293,1059.78400856)(-262.71999993,1059.78400856)
\curveto(-267.19999545,1059.78400856)(-268.76799993,1055.89200501)(-268.76799993,1052.33600856)
\curveto(-268.76799993,1048.38801251)(-266.97599551,1044.72000856)(-262.55199993,1044.72000856)
\curveto(-261.37600111,1044.72000856)(-260.25599893,1044.97200924)(-259.24799993,1045.64400856)
\curveto(-258.63200055,1046.06400814)(-258.32399968,1046.42800887)(-258.07199993,1046.73600856)
\curveto(-258.09999991,1045.95200935)(-258.09999982,1045.72800831)(-257.98799993,1045.47600856)
\curveto(-257.84800007,1045.19600884)(-257.51199929,1044.83200856)(-256.86799993,1044.83200856)
\curveto(-255.72000108,1044.83200856)(-255.60799993,1045.70000893)(-255.60799993,1046.06400856)
\lineto(-255.60799993,1063.67600856)
\moveto(-258.07199993,1052.22400856)
\curveto(-258.07199993,1051.55200924)(-258.07200217,1048.41600742)(-260.31199993,1047.26800856)
\curveto(-260.87199937,1046.96000887)(-261.51600063,1046.84800856)(-262.21599993,1046.84800856)
\curveto(-265.49199666,1046.84800856)(-266.10799993,1049.788011)(-266.10799993,1052.22400856)
\curveto(-266.10799993,1052.86800792)(-266.05199982,1053.51200921)(-265.93999993,1054.15600856)
\curveto(-265.54800033,1056.1160066)(-264.45599772,1057.60000856)(-262.24399993,1057.60000856)
\curveto(-261.20800097,1057.60000856)(-259.83599901,1057.34800705)(-258.91199993,1055.83600856)
\curveto(-258.21200063,1054.71600968)(-258.07199993,1053.37200742)(-258.07199993,1052.22400856)
}
}
{
\newrgbcolor{curcolor}{0 0 0}
\pscustom[linestyle=none,fillstyle=solid,fillcolor=curcolor]
{
\newpath
\moveto(-240.37643743,1051.55200856)
\curveto(-240.0124378,1051.55200856)(-239.11643743,1051.63600974)(-239.11643743,1052.81200856)
\curveto(-239.11643743,1054.380007)(-239.59243836,1056.20000985)(-240.51643743,1057.48800856)
\curveto(-241.38443657,1058.66400739)(-242.47643869,1059.28000887)(-243.73643743,1059.58800856)
\curveto(-244.32443685,1059.72800842)(-244.94043805,1059.78400856)(-245.55643743,1059.78400856)
\curveto(-250.51243248,1059.78400856)(-252.41643743,1055.66800504)(-252.41643743,1052.14000856)
\curveto(-252.41643743,1050.12401058)(-251.82843617,1048.10800714)(-250.56843743,1046.68000856)
\curveto(-249.14043886,1045.08401016)(-247.29243561,1044.72000856)(-245.47243743,1044.72000856)
\curveto(-244.68843822,1044.72000856)(-242.1684355,1044.72001075)(-240.23643743,1046.90400856)
\curveto(-239.92843774,1047.24000823)(-239.45243732,1047.8560091)(-239.34043743,1048.38800856)
\curveto(-239.31243746,1048.47200848)(-239.31243743,1048.55600868)(-239.31243743,1048.66800856)
\curveto(-239.31243743,1049.48000775)(-240.15243805,1049.84400856)(-240.76843743,1049.84400856)
\curveto(-241.38443682,1049.84400856)(-241.52443788,1049.56400781)(-241.97243743,1048.80800856)
\curveto(-242.47643693,1048.02400935)(-243.26043965,1046.87600856)(-245.47243743,1046.87600856)
\curveto(-246.59243631,1046.87600856)(-247.74043839,1047.07200974)(-248.69243743,1048.24800856)
\curveto(-249.44843668,1049.17200764)(-249.72843749,1050.29200982)(-249.78443743,1051.55200856)
\lineto(-240.37643743,1051.55200856)
\moveto(-249.75643743,1053.65200856)
\curveto(-249.61643757,1054.29600792)(-249.36443631,1055.64000954)(-248.24443743,1056.62000856)
\curveto(-247.26443841,1057.4880077)(-246.11643685,1057.54400856)(-245.52843743,1057.54400856)
\curveto(-242.86844009,1057.54400856)(-241.86043732,1055.50000672)(-241.74843743,1053.65200856)
\lineto(-249.75643743,1053.65200856)
}
}
{
\newrgbcolor{curcolor}{0 0 0}
\pscustom[linestyle=none,fillstyle=solid,fillcolor=curcolor]
{
\newpath
\moveto(-224.09443743,1058.58000856)
\curveto(-224.09443743,1058.88800826)(-224.1784385,1059.64400856)(-225.24243743,1059.64400856)
\curveto(-225.83043685,1059.64400856)(-226.11043755,1059.33600837)(-226.22243743,1059.14000856)
\curveto(-226.33443732,1058.91600879)(-226.33443746,1058.77600764)(-226.36243743,1057.85200856)
\curveto(-226.67043713,1058.35600806)(-227.0344385,1058.88800904)(-228.09843743,1059.36400856)
\curveto(-228.6304369,1059.58800834)(-229.35843847,1059.78400856)(-230.39443743,1059.78400856)
\curveto(-233.44643438,1059.78400856)(-235.49043825,1058.07600579)(-236.30243743,1055.30400856)
\curveto(-236.55443718,1054.35200952)(-236.66643743,1053.40000758)(-236.66643743,1052.42000856)
\curveto(-236.66643743,1048.72401226)(-234.90243323,1045.50400856)(-230.70243743,1045.50400856)
\curveto(-228.15443998,1045.50400856)(-227.11843679,1046.70800932)(-226.47443743,1047.46400856)
\lineto(-226.50243743,1046.17600856)
\curveto(-226.55843738,1044.2440105)(-226.67044077,1042.11600856)(-230.00243743,1042.11600856)
\curveto(-230.64643679,1042.11600856)(-232.04643805,1042.20000926)(-232.66243743,1042.90000856)
\curveto(-232.85843724,1043.12400834)(-232.97043769,1043.34800907)(-233.22243743,1043.85200856)
\curveto(-233.44643721,1044.32800809)(-233.72643827,1044.91600856)(-234.56643743,1044.91600856)
\curveto(-235.15443685,1044.91600856)(-235.91043743,1044.5520077)(-235.91043743,1043.68400856)
\curveto(-235.91043743,1042.92800932)(-235.23843645,1041.72400786)(-234.25843743,1041.02400856)
\curveto(-233.27843841,1040.29600929)(-231.79443581,1039.96000856)(-230.17043743,1039.96000856)
\curveto(-228.23843937,1039.96000856)(-226.55843645,1040.43600946)(-225.57843743,1041.33200856)
\curveto(-224.45843855,1042.36800753)(-224.09443743,1043.85201162)(-224.09443743,1046.90400856)
\lineto(-224.09443743,1058.58000856)
\moveto(-226.44643743,1052.81200856)
\curveto(-226.44643743,1052.16800921)(-226.50243819,1050.34800747)(-227.25843743,1049.25600856)
\curveto(-227.87443682,1048.36000946)(-229.05043872,1047.71600856)(-230.33843743,1047.71600856)
\curveto(-231.59843617,1047.71600856)(-232.57843799,1048.30400935)(-233.13843743,1049.08800856)
\curveto(-233.81043676,1049.98400767)(-234.03443743,1051.35600982)(-234.03443743,1052.61600856)
\curveto(-234.03443743,1053.76400742)(-233.81043676,1055.08000954)(-233.13843743,1056.06000856)
\curveto(-232.2704383,1057.29200733)(-231.03843651,1057.57200856)(-230.11443743,1057.57200856)
\curveto(-226.4464411,1057.57200856)(-226.44643743,1053.54000784)(-226.44643743,1052.81200856)
}
}
{
\newrgbcolor{curcolor}{0 0 0}
\pscustom[linestyle=none,fillstyle=solid,fillcolor=curcolor]
{
\newpath
\moveto(-217.78043743,1058.49600856)
\curveto(-217.78043743,1058.83200823)(-217.89243853,1059.67200856)(-218.98443743,1059.67200856)
\curveto(-220.10443631,1059.67200856)(-220.21643743,1058.83200823)(-220.21643743,1058.49600856)
\lineto(-220.21643743,1046.00800856)
\curveto(-220.21643743,1045.61600896)(-220.10443629,1044.83200856)(-218.95643743,1044.83200856)
\curveto(-217.75243864,1044.83200856)(-217.66843743,1045.56000901)(-217.66843743,1046.00800856)
\lineto(-217.66843743,1051.58000856)
\curveto(-217.66843743,1053.26000688)(-217.66843559,1055.22000985)(-215.82043743,1056.50800856)
\curveto(-215.06443819,1056.98400809)(-214.33643673,1057.09600868)(-213.63643743,1057.20800856)
\curveto(-213.24443783,1057.26400851)(-212.46043743,1057.43200954)(-212.46043743,1058.41200856)
\curveto(-212.46043743,1059.6720073)(-213.69243774,1059.67200856)(-214.00043743,1059.67200856)
\curveto(-215.65243578,1059.67200856)(-217.24843797,1058.63600697)(-217.78043743,1057.04000856)
\lineto(-217.78043743,1058.49600856)
}
}
{
\newrgbcolor{curcolor}{0 0 0}
\pscustom[linestyle=none,fillstyle=solid,fillcolor=curcolor]
{
\newpath
\moveto(-198.97799993,1051.55200856)
\curveto(-198.6140003,1051.55200856)(-197.71799993,1051.63600974)(-197.71799993,1052.81200856)
\curveto(-197.71799993,1054.380007)(-198.19400086,1056.20000985)(-199.11799993,1057.48800856)
\curveto(-199.98599907,1058.66400739)(-201.07800119,1059.28000887)(-202.33799993,1059.58800856)
\curveto(-202.92599935,1059.72800842)(-203.54200055,1059.78400856)(-204.15799993,1059.78400856)
\curveto(-209.11399498,1059.78400856)(-211.01799993,1055.66800504)(-211.01799993,1052.14000856)
\curveto(-211.01799993,1050.12401058)(-210.42999867,1048.10800714)(-209.16999993,1046.68000856)
\curveto(-207.74200136,1045.08401016)(-205.89399811,1044.72000856)(-204.07399993,1044.72000856)
\curveto(-203.29000072,1044.72000856)(-200.769998,1044.72001075)(-198.83799993,1046.90400856)
\curveto(-198.53000024,1047.24000823)(-198.05399982,1047.8560091)(-197.94199993,1048.38800856)
\curveto(-197.91399996,1048.47200848)(-197.91399993,1048.55600868)(-197.91399993,1048.66800856)
\curveto(-197.91399993,1049.48000775)(-198.75400055,1049.84400856)(-199.36999993,1049.84400856)
\curveto(-199.98599932,1049.84400856)(-200.12600038,1049.56400781)(-200.57399993,1048.80800856)
\curveto(-201.07799943,1048.02400935)(-201.86200215,1046.87600856)(-204.07399993,1046.87600856)
\curveto(-205.19399881,1046.87600856)(-206.34200089,1047.07200974)(-207.29399993,1048.24800856)
\curveto(-208.04999918,1049.17200764)(-208.32999999,1050.29200982)(-208.38599993,1051.55200856)
\lineto(-198.97799993,1051.55200856)
\moveto(-208.35799993,1053.65200856)
\curveto(-208.21800007,1054.29600792)(-207.96599881,1055.64000954)(-206.84599993,1056.62000856)
\curveto(-205.86600091,1057.4880077)(-204.71799935,1057.54400856)(-204.12999993,1057.54400856)
\curveto(-201.47000259,1057.54400856)(-200.46199982,1055.50000672)(-200.34999993,1053.65200856)
\lineto(-208.35799993,1053.65200856)
}
}
{
\newrgbcolor{curcolor}{0 0 0}
\pscustom[linestyle=none,fillstyle=solid,fillcolor=curcolor]
{
\newpath
\moveto(-183.22799993,1051.55200856)
\curveto(-182.8640003,1051.55200856)(-181.96799993,1051.63600974)(-181.96799993,1052.81200856)
\curveto(-181.96799993,1054.380007)(-182.44400086,1056.20000985)(-183.36799993,1057.48800856)
\curveto(-184.23599907,1058.66400739)(-185.32800119,1059.28000887)(-186.58799993,1059.58800856)
\curveto(-187.17599935,1059.72800842)(-187.79200055,1059.78400856)(-188.40799993,1059.78400856)
\curveto(-193.36399498,1059.78400856)(-195.26799993,1055.66800504)(-195.26799993,1052.14000856)
\curveto(-195.26799993,1050.12401058)(-194.67999867,1048.10800714)(-193.41999993,1046.68000856)
\curveto(-191.99200136,1045.08401016)(-190.14399811,1044.72000856)(-188.32399993,1044.72000856)
\curveto(-187.54000072,1044.72000856)(-185.019998,1044.72001075)(-183.08799993,1046.90400856)
\curveto(-182.78000024,1047.24000823)(-182.30399982,1047.8560091)(-182.19199993,1048.38800856)
\curveto(-182.16399996,1048.47200848)(-182.16399993,1048.55600868)(-182.16399993,1048.66800856)
\curveto(-182.16399993,1049.48000775)(-183.00400055,1049.84400856)(-183.61999993,1049.84400856)
\curveto(-184.23599932,1049.84400856)(-184.37600038,1049.56400781)(-184.82399993,1048.80800856)
\curveto(-185.32799943,1048.02400935)(-186.11200215,1046.87600856)(-188.32399993,1046.87600856)
\curveto(-189.44399881,1046.87600856)(-190.59200089,1047.07200974)(-191.54399993,1048.24800856)
\curveto(-192.29999918,1049.17200764)(-192.57999999,1050.29200982)(-192.63599993,1051.55200856)
\lineto(-183.22799993,1051.55200856)
\moveto(-192.60799993,1053.65200856)
\curveto(-192.46800007,1054.29600792)(-192.21599881,1055.64000954)(-191.09599993,1056.62000856)
\curveto(-190.11600091,1057.4880077)(-188.96799935,1057.54400856)(-188.37999993,1057.54400856)
\curveto(-185.72000259,1057.54400856)(-184.71199982,1055.50000672)(-184.59999993,1053.65200856)
\lineto(-192.60799993,1053.65200856)
}
}
{
\newrgbcolor{curcolor}{0 0 0}
\pscustom[linewidth=1.17669678,linecolor=curcolor]
{
\newpath
\moveto(640,568.00001262)
\lineto(-80,568.00001262)
\lineto(-80,828.00000262)
\lineto(-80,828.00000262)
}
}
{
\newrgbcolor{curcolor}{0 0 0}
\pscustom[linestyle=none,fillstyle=solid,fillcolor=curcolor]
{
\newpath
\moveto(-94.57999993,768.80003908)
\curveto(-94.55999995,768.54003934)(-94.49999907,767.88003908)(-93.63999993,767.88003908)
\curveto(-92.76000081,767.88003908)(-92.69999991,768.52003936)(-92.67999993,768.80003908)
\lineto(-92.67999993,781.28003908)
\curveto(-92.69999991,781.52003884)(-92.72000071,782.02003908)(-93.49999993,782.02003908)
\curveto(-93.95999947,782.02003908)(-94.06000009,781.78003872)(-94.21999993,781.42003908)
\curveto(-94.73999941,780.06004044)(-95.54000173,779.42003884)(-97.33999993,779.18003908)
\curveto(-97.67999959,779.12003914)(-97.91999993,779.0600386)(-97.91999993,778.58003908)
\curveto(-97.91999993,777.84003982)(-97.49999967,777.84003906)(-97.23999993,777.82003908)
\lineto(-94.57999993,777.82003908)
\lineto(-94.57999993,768.80003908)
}
}
{
\newrgbcolor{curcolor}{0 0 0}
\pscustom[linestyle=none,fillstyle=solid,fillcolor=curcolor]
{
\newpath
\moveto(-89.31999993,575.07997805)
\curveto(-89.31999993,576.15997697)(-89.38000019,577.21997909)(-89.63999993,578.25997805)
\curveto(-90.17999939,580.45997585)(-91.58000257,582.09997805)(-94.21999993,582.09997805)
\curveto(-94.95999919,582.09997805)(-97.02000125,582.03997593)(-98.33999993,579.91997805)
\curveto(-99.41999885,578.17997979)(-99.43999993,575.89997707)(-99.43999993,574.91997805)
\curveto(-99.43999993,573.61997935)(-99.43999785,570.41997639)(-97.35999993,568.75997805)
\curveto(-96.54000075,568.07997873)(-95.49999887,567.79997805)(-94.43999993,567.79997805)
\curveto(-89.32000505,567.79997805)(-89.31999993,573.67997945)(-89.31999993,575.07997805)
\moveto(-94.35999993,569.39997805)
\curveto(-95.29999899,569.39997805)(-96.32000051,569.83997929)(-96.89999993,571.07997805)
\curveto(-97.41999941,572.19997693)(-97.49999993,573.85997901)(-97.49999993,574.81997805)
\curveto(-97.49999993,577.25997561)(-97.13999707,580.53997805)(-94.27999993,580.53997805)
\curveto(-91.26000295,580.53997805)(-91.25999993,576.57997667)(-91.25999993,575.19997805)
\curveto(-91.25999993,571.95998129)(-91.82000151,570.15997745)(-93.39999993,569.55997805)
\curveto(-93.71999961,569.45997815)(-94.04000025,569.39997805)(-94.35999993,569.39997805)
}
}
{
\newrgbcolor{curcolor}{0 0 0}
\pscustom[linewidth=1,linecolor=curcolor]
{
\newpath
\moveto(1.42856,850.00000262)
\lineto(96.42856,1050.00000262)
\lineto(176.42856,850.00000262)
}
}
{
\newrgbcolor{curcolor}{0 0 0}
\pscustom[linewidth=1,linecolor=curcolor]
{
\newpath
\moveto(590,850.00000262)
\lineto(590,850.00000262)
\lineto(495,1055.00000262)
\lineto(415,850.00000262)
}
}
{
\newrgbcolor{curcolor}{0 0 0}
\pscustom[linewidth=1,linecolor=curcolor,linestyle=dashed,dash=1 8]
{
\newpath
\moveto(414.99996,850.00000262)
\lineto(414.99996,230.00001262)
}
}
{
\newrgbcolor{curcolor}{0 0 0}
\pscustom[linewidth=1,linecolor=curcolor,linestyle=dashed,dash=1 8]
{
\newpath
\moveto(176.4286,850.00000262)
\lineto(176.4286,230.00001262)
}
}
{
\newrgbcolor{curcolor}{0 0 0}
\pscustom[linewidth=1,linecolor=curcolor]
{
\newpath
\moveto(0,568.00000262)
\lineto(175,768.00000262)
\lineto(415,768.00000262)
\lineto(590,568.00000262)
}
}
{
\newrgbcolor{curcolor}{0 0 0}
\pscustom[linestyle=none,fillstyle=solid,fillcolor=curcolor]
{
\newpath
\moveto(-268.06799993,799.09197805)
\curveto(-268.06799993,798.75597838)(-267.95599873,797.83197805)(-266.75199993,797.83197805)
\curveto(-266.08000061,797.83197805)(-265.71599982,798.2239783)(-265.60399993,798.47597805)
\curveto(-265.52000002,798.67197785)(-265.46399993,798.89597824)(-265.46399993,799.09197805)
\lineto(-265.46399993,805.50397805)
\curveto(-265.46399993,807.01597653)(-265.43599929,808.05197905)(-264.79199993,809.05997805)
\curveto(-264.12000061,810.12397698)(-263.08399879,810.54397805)(-261.93599993,810.54397805)
\curveto(-259.6680022,810.54397805)(-259.44399988,808.63997757)(-259.38799993,808.16397805)
\curveto(-259.33199999,807.74397847)(-259.33199993,807.49197673)(-259.33199993,806.17597805)
\lineto(-259.33199993,799.09197805)
\curveto(-259.33199993,798.69997844)(-259.2479994,798.25197779)(-258.71599993,797.99997805)
\curveto(-258.49200016,797.88797816)(-258.26799971,797.83197805)(-258.04399993,797.83197805)
\curveto(-257.90400007,797.83197805)(-257.62399968,797.85997819)(-257.37199993,797.99997805)
\curveto(-256.84000047,798.25197779)(-256.75599993,798.69997844)(-256.75599993,799.09197805)
\lineto(-256.75599993,806.03597805)
\curveto(-256.75599993,807.21197687)(-256.75599932,808.13597897)(-256.13999993,809.05997805)
\curveto(-255.69200038,809.73197737)(-254.79599842,810.54397805)(-253.28399993,810.54397805)
\curveto(-250.73600248,810.54397805)(-250.65199993,808.38797712)(-250.65199993,807.46397805)
\curveto(-250.65199993,807.01597849)(-250.62399993,806.23197721)(-250.62399993,805.39197805)
\lineto(-250.62399993,799.11997805)
\curveto(-250.62399993,798.69997847)(-250.53999937,798.25197779)(-249.97999993,797.99997805)
\curveto(-249.78400013,797.88797816)(-249.55999971,797.83197805)(-249.33599993,797.83197805)
\curveto(-248.13200114,797.83197805)(-248.01999993,798.75597841)(-248.01999993,799.11997805)
\lineto(-248.01999993,807.49197805)
\curveto(-248.01999993,808.89197665)(-248.07600089,810.17997919)(-249.02799993,811.32797805)
\curveto(-250.0639989,812.55997681)(-251.49200133,812.81197805)(-252.89199993,812.81197805)
\curveto(-255.01999781,812.81197805)(-256.33600103,811.85997673)(-257.42799993,810.54397805)
\curveto(-257.62399974,810.93597765)(-258.01600094,811.71997858)(-259.02399993,812.25197805)
\curveto(-259.75199921,812.67197763)(-260.59200077,812.81197805)(-261.43199993,812.81197805)
\curveto(-263.19599817,812.81197805)(-264.42800103,812.1959767)(-265.51999993,810.85197805)
\curveto(-265.51999993,811.80397709)(-265.52000052,812.25197833)(-266.10799993,812.53197805)
\curveto(-266.49999954,812.75597782)(-267.14400024,812.69997788)(-267.45199993,812.53197805)
\curveto(-267.9839994,812.2799783)(-268.06799993,811.85997765)(-268.06799993,811.46797805)
\lineto(-268.06799993,799.09197805)
}
}
{
\newrgbcolor{curcolor}{0 0 0}
\pscustom[linestyle=none,fillstyle=solid,fillcolor=curcolor]
{
\newpath
\moveto(-232.82956243,804.55197805)
\curveto(-232.4655628,804.55197805)(-231.56956243,804.63597922)(-231.56956243,805.81197805)
\curveto(-231.56956243,807.37997648)(-232.04556336,809.19997933)(-232.96956243,810.48797805)
\curveto(-233.83756157,811.66397687)(-234.92956369,812.27997835)(-236.18956243,812.58797805)
\curveto(-236.77756185,812.72797791)(-237.39356305,812.78397805)(-238.00956243,812.78397805)
\curveto(-242.96555748,812.78397805)(-244.86956243,808.66797452)(-244.86956243,805.13997805)
\curveto(-244.86956243,803.12398006)(-244.28156117,801.10797662)(-243.02156243,799.67997805)
\curveto(-241.59356386,798.08397964)(-239.74556061,797.71997805)(-237.92556243,797.71997805)
\curveto(-237.14156322,797.71997805)(-234.6215605,797.71998023)(-232.68956243,799.90397805)
\curveto(-232.38156274,800.23997771)(-231.90556232,800.85597858)(-231.79356243,801.38797805)
\curveto(-231.76556246,801.47197796)(-231.76556243,801.55597816)(-231.76556243,801.66797805)
\curveto(-231.76556243,802.47997723)(-232.60556305,802.84397805)(-233.22156243,802.84397805)
\curveto(-233.83756182,802.84397805)(-233.97756288,802.56397729)(-234.42556243,801.80797805)
\curveto(-234.92956193,801.02397883)(-235.71356465,799.87597805)(-237.92556243,799.87597805)
\curveto(-239.04556131,799.87597805)(-240.19356339,800.07197922)(-241.14556243,801.24797805)
\curveto(-241.90156168,802.17197712)(-242.18156249,803.29197931)(-242.23756243,804.55197805)
\lineto(-232.82956243,804.55197805)
\moveto(-242.20956243,806.65197805)
\curveto(-242.06956257,807.2959774)(-241.81756131,808.63997903)(-240.69756243,809.61997805)
\curveto(-239.71756341,810.48797718)(-238.56956185,810.54397805)(-237.98156243,810.54397805)
\curveto(-235.32156509,810.54397805)(-234.31356232,808.4999762)(-234.20156243,806.65197805)
\lineto(-242.20956243,806.65197805)
}
}
{
\newrgbcolor{curcolor}{0 0 0}
\pscustom[linestyle=none,fillstyle=solid,fillcolor=curcolor]
{
\newpath
\moveto(-228.41956243,799.09197805)
\curveto(-228.41956243,798.75597838)(-228.30756123,797.83197805)(-227.10356243,797.83197805)
\curveto(-226.43156311,797.83197805)(-226.06756232,798.2239783)(-225.95556243,798.47597805)
\curveto(-225.87156252,798.67197785)(-225.81556243,798.89597824)(-225.81556243,799.09197805)
\lineto(-225.81556243,805.50397805)
\curveto(-225.81556243,807.01597653)(-225.78756179,808.05197905)(-225.14356243,809.05997805)
\curveto(-224.47156311,810.12397698)(-223.43556129,810.54397805)(-222.28756243,810.54397805)
\curveto(-220.0195647,810.54397805)(-219.79556238,808.63997757)(-219.73956243,808.16397805)
\curveto(-219.68356249,807.74397847)(-219.68356243,807.49197673)(-219.68356243,806.17597805)
\lineto(-219.68356243,799.09197805)
\curveto(-219.68356243,798.69997844)(-219.5995619,798.25197779)(-219.06756243,797.99997805)
\curveto(-218.84356266,797.88797816)(-218.61956221,797.83197805)(-218.39556243,797.83197805)
\curveto(-218.25556257,797.83197805)(-217.97556218,797.85997819)(-217.72356243,797.99997805)
\curveto(-217.19156297,798.25197779)(-217.10756243,798.69997844)(-217.10756243,799.09197805)
\lineto(-217.10756243,806.03597805)
\curveto(-217.10756243,807.21197687)(-217.10756182,808.13597897)(-216.49156243,809.05997805)
\curveto(-216.04356288,809.73197737)(-215.14756092,810.54397805)(-213.63556243,810.54397805)
\curveto(-211.08756498,810.54397805)(-211.00356243,808.38797712)(-211.00356243,807.46397805)
\curveto(-211.00356243,807.01597849)(-210.97556243,806.23197721)(-210.97556243,805.39197805)
\lineto(-210.97556243,799.11997805)
\curveto(-210.97556243,798.69997847)(-210.89156187,798.25197779)(-210.33156243,797.99997805)
\curveto(-210.13556263,797.88797816)(-209.91156221,797.83197805)(-209.68756243,797.83197805)
\curveto(-208.48356364,797.83197805)(-208.37156243,798.75597841)(-208.37156243,799.11997805)
\lineto(-208.37156243,807.49197805)
\curveto(-208.37156243,808.89197665)(-208.42756339,810.17997919)(-209.37956243,811.32797805)
\curveto(-210.4155614,812.55997681)(-211.84356383,812.81197805)(-213.24356243,812.81197805)
\curveto(-215.37156031,812.81197805)(-216.68756353,811.85997673)(-217.77956243,810.54397805)
\curveto(-217.97556224,810.93597765)(-218.36756344,811.71997858)(-219.37556243,812.25197805)
\curveto(-220.10356171,812.67197763)(-220.94356327,812.81197805)(-221.78356243,812.81197805)
\curveto(-223.54756067,812.81197805)(-224.77956353,812.1959767)(-225.87156243,810.85197805)
\curveto(-225.87156243,811.80397709)(-225.87156302,812.25197833)(-226.45956243,812.53197805)
\curveto(-226.85156204,812.75597782)(-227.49556274,812.69997788)(-227.80356243,812.53197805)
\curveto(-228.3355619,812.2799783)(-228.41956243,811.85997765)(-228.41956243,811.46797805)
\lineto(-228.41956243,799.09197805)
}
}
{
\newrgbcolor{curcolor}{0 0 0}
\pscustom[linestyle=none,fillstyle=solid,fillcolor=curcolor]
{
\newpath
\moveto(-202.00112493,816.67597805)
\curveto(-202.00112493,817.09597763)(-202.08512547,817.51597833)(-202.61712493,817.79597805)
\curveto(-202.81312474,817.87997796)(-203.03712516,817.93597805)(-203.26112493,817.93597805)
\curveto(-203.93312426,817.93597805)(-204.26912505,817.54397782)(-204.38112493,817.31997805)
\curveto(-204.46512485,817.12397824)(-204.52112493,816.89997782)(-204.52112493,816.67597805)
\lineto(-204.52112493,799.06397805)
\curveto(-204.52112493,798.64397847)(-204.4371244,798.25197777)(-203.90512493,797.97197805)
\curveto(-203.70912513,797.85997816)(-203.48512468,797.83197805)(-203.23312493,797.83197805)
\curveto(-202.61712555,797.83197805)(-202.28112479,798.1679783)(-202.14112493,798.41997805)
\curveto(-202.00112507,798.67197779)(-202.00112493,798.92397869)(-202.00112493,799.56797805)
\curveto(-201.16112577,798.72797889)(-200.01312253,797.69197805)(-197.60512493,797.69197805)
\curveto(-193.88112866,797.69197805)(-191.36112493,800.51998275)(-191.36112493,805.22397805)
\curveto(-191.36112493,808.16397511)(-192.45312992,812.78397805)(-197.43712493,812.78397805)
\curveto(-200.37712199,812.78397805)(-201.6371253,811.15997757)(-202.00112493,810.68397805)
\lineto(-202.00112493,816.67597805)
\moveto(-194.04912493,805.19597805)
\curveto(-194.04912493,802.08798115)(-194.97312619,800.77197743)(-196.23312493,800.15597805)
\curveto(-196.7651244,799.9039783)(-197.35312555,799.79197805)(-197.96912493,799.79197805)
\curveto(-201.3011216,799.79197805)(-202.02912493,803.09598026)(-202.02912493,805.30797805)
\curveto(-202.02912493,807.23997611)(-201.49712325,809.28397897)(-199.81712493,810.20797805)
\curveto(-199.22912552,810.51597774)(-198.55712426,810.65597805)(-197.88512493,810.65597805)
\curveto(-194.07712874,810.65597805)(-194.04912493,806.37197687)(-194.04912493,805.19597805)
}
}
{
\newrgbcolor{curcolor}{0 0 0}
\pscustom[linestyle=none,fillstyle=solid,fillcolor=curcolor]
{
\newpath
\moveto(-176.82956243,804.55197805)
\curveto(-176.4655628,804.55197805)(-175.56956243,804.63597922)(-175.56956243,805.81197805)
\curveto(-175.56956243,807.37997648)(-176.04556336,809.19997933)(-176.96956243,810.48797805)
\curveto(-177.83756157,811.66397687)(-178.92956369,812.27997835)(-180.18956243,812.58797805)
\curveto(-180.77756185,812.72797791)(-181.39356305,812.78397805)(-182.00956243,812.78397805)
\curveto(-186.96555748,812.78397805)(-188.86956243,808.66797452)(-188.86956243,805.13997805)
\curveto(-188.86956243,803.12398006)(-188.28156117,801.10797662)(-187.02156243,799.67997805)
\curveto(-185.59356386,798.08397964)(-183.74556061,797.71997805)(-181.92556243,797.71997805)
\curveto(-181.14156322,797.71997805)(-178.6215605,797.71998023)(-176.68956243,799.90397805)
\curveto(-176.38156274,800.23997771)(-175.90556232,800.85597858)(-175.79356243,801.38797805)
\curveto(-175.76556246,801.47197796)(-175.76556243,801.55597816)(-175.76556243,801.66797805)
\curveto(-175.76556243,802.47997723)(-176.60556305,802.84397805)(-177.22156243,802.84397805)
\curveto(-177.83756182,802.84397805)(-177.97756288,802.56397729)(-178.42556243,801.80797805)
\curveto(-178.92956193,801.02397883)(-179.71356465,799.87597805)(-181.92556243,799.87597805)
\curveto(-183.04556131,799.87597805)(-184.19356339,800.07197922)(-185.14556243,801.24797805)
\curveto(-185.90156168,802.17197712)(-186.18156249,803.29197931)(-186.23756243,804.55197805)
\lineto(-176.82956243,804.55197805)
\moveto(-186.20956243,806.65197805)
\curveto(-186.06956257,807.2959774)(-185.81756131,808.63997903)(-184.69756243,809.61997805)
\curveto(-183.71756341,810.48797718)(-182.56956185,810.54397805)(-181.98156243,810.54397805)
\curveto(-179.32156509,810.54397805)(-178.31356232,808.4999762)(-178.20156243,806.65197805)
\lineto(-186.20956243,806.65197805)
}
}
{
\newrgbcolor{curcolor}{0 0 0}
\pscustom[linestyle=none,fillstyle=solid,fillcolor=curcolor]
{
\newpath
\moveto(-169.98356243,811.49597805)
\curveto(-169.98356243,811.83197771)(-170.09556353,812.67197805)(-171.18756243,812.67197805)
\curveto(-172.30756131,812.67197805)(-172.41956243,811.83197771)(-172.41956243,811.49597805)
\lineto(-172.41956243,799.00797805)
\curveto(-172.41956243,798.61597844)(-172.30756129,797.83197805)(-171.15956243,797.83197805)
\curveto(-169.95556364,797.83197805)(-169.87156243,798.55997849)(-169.87156243,799.00797805)
\lineto(-169.87156243,804.57997805)
\curveto(-169.87156243,806.25997637)(-169.87156059,808.21997933)(-168.02356243,809.50797805)
\curveto(-167.26756319,809.98397757)(-166.53956173,810.09597816)(-165.83956243,810.20797805)
\curveto(-165.44756283,810.26397799)(-164.66356243,810.43197903)(-164.66356243,811.41197805)
\curveto(-164.66356243,812.67197679)(-165.89556274,812.67197805)(-166.20356243,812.67197805)
\curveto(-167.85556078,812.67197805)(-169.45156297,811.63597645)(-169.98356243,810.03997805)
\lineto(-169.98356243,811.49597805)
}
}
{
\newrgbcolor{curcolor}{0 0 0}
\pscustom[linestyle=none,fillstyle=solid,fillcolor=curcolor]
{
\newpath
\moveto(-157.98512493,806.93197805)
\curveto(-159.27312365,807.32397765)(-160.39312493,807.65997933)(-160.39312493,808.94797805)
\curveto(-160.39312493,809.56397743)(-160.05712446,810.12397835)(-159.58112493,810.43197805)
\curveto(-159.04912547,810.79597768)(-158.2931246,810.82397805)(-157.95712493,810.82397805)
\curveto(-155.74512715,810.82397805)(-154.98912468,809.50797757)(-154.73712493,809.03197805)
\curveto(-154.31712535,808.21997886)(-154.14912432,807.91197805)(-153.53312493,807.91197805)
\curveto(-153.11312535,807.91197805)(-152.35712493,808.16397889)(-152.35712493,809.00397805)
\curveto(-152.35712493,809.56397749)(-153.11312967,812.78397805)(-157.84512493,812.78397805)
\curveto(-162.66112012,812.78397805)(-162.94112493,809.33997732)(-162.94112493,808.61197805)
\curveto(-162.94112493,805.92398073)(-160.56112339,805.13997754)(-159.02112493,804.63597805)
\lineto(-157.39712493,804.15997805)
\curveto(-155.96912636,803.73997847)(-154.68112493,803.37597659)(-154.68112493,801.91997805)
\curveto(-154.68112493,800.60397936)(-155.91312659,799.73597805)(-157.56512493,799.73597805)
\curveto(-158.74112376,799.73597805)(-159.91712558,800.12797903)(-160.56112493,801.10797805)
\curveto(-160.75712474,801.41597774)(-160.78512533,801.52797911)(-161.17712493,802.59197805)
\curveto(-161.31712479,802.87197777)(-161.48512558,803.20797805)(-162.12912493,803.20797805)
\curveto(-162.54912451,803.20797805)(-163.50112493,802.98397709)(-163.50112493,802.03197805)
\curveto(-163.50112493,801.63997844)(-163.24912365,800.09997695)(-161.96112493,799.00797805)
\curveto(-160.53312636,797.80397925)(-158.54512412,797.71997805)(-157.73312493,797.71997805)
\curveto(-156.44512622,797.71997805)(-154.96112387,797.97197877)(-153.89712493,798.69997805)
\curveto(-153.05712577,799.25997749)(-152.13312493,800.35197984)(-152.13312493,802.14397805)
\curveto(-152.13312493,805.19597499)(-154.84912645,805.97997849)(-156.36112493,806.42797805)
\lineto(-157.98512493,806.93197805)
}
}
{
\newrgbcolor{curcolor}{0 0 0}
\pscustom[linestyle=none,fillstyle=solid,fillcolor=curcolor]
{
\newpath
\moveto(-146.51868743,816.67597805)
\curveto(-146.51868743,817.01197771)(-146.65868861,817.93597805)(-147.83468743,817.93597805)
\curveto(-149.01068626,817.93597805)(-149.12268743,817.01197768)(-149.12268743,816.64797805)
\lineto(-149.12268743,799.09197805)
\curveto(-149.12268743,798.72797841)(-149.01068626,797.83197805)(-147.83468743,797.83197805)
\curveto(-147.13468813,797.83197805)(-146.79868732,798.2239783)(-146.68668743,798.47597805)
\curveto(-146.60268752,798.64397788)(-146.54668743,798.86797827)(-146.54668743,799.09197805)
\lineto(-146.54668743,805.69997805)
\curveto(-146.54668743,807.01597673)(-146.49068643,808.21997911)(-145.48268743,809.28397805)
\curveto(-144.83868808,809.95597737)(-143.80268623,810.45997805)(-142.59868743,810.45997805)
\curveto(-142.15068788,810.45997805)(-140.86268679,810.43197684)(-140.21868743,809.22797805)
\curveto(-139.88268777,808.55597872)(-139.82668743,807.82797631)(-139.82668743,806.09197805)
\lineto(-139.82668743,799.09197805)
\curveto(-139.82668743,798.72797841)(-139.71468626,797.83197805)(-138.53868743,797.83197805)
\curveto(-137.83868813,797.83197805)(-137.50268732,798.2239783)(-137.39068743,798.47597805)
\curveto(-137.30668752,798.64397788)(-137.25068743,798.86797827)(-137.25068743,799.09197805)
\lineto(-137.25068743,806.87597805)
\curveto(-137.25068743,808.2199767)(-137.25068833,809.84397931)(-138.14668743,811.10397805)
\curveto(-138.37068721,811.41197774)(-139.40669009,812.78397805)(-142.06668743,812.78397805)
\curveto(-143.71868578,812.78397805)(-145.51068844,812.1959767)(-146.51868743,810.85197805)
\lineto(-146.51868743,816.67597805)
}
}
{
\newrgbcolor{curcolor}{0 0 0}
\pscustom[linestyle=none,fillstyle=solid,fillcolor=curcolor]
{
\newpath
\moveto(-130.82468743,811.43997805)
\curveto(-130.82468743,811.77597771)(-130.93668861,812.67197805)(-132.11268743,812.67197805)
\curveto(-133.26068629,812.67197805)(-133.37268743,811.77597771)(-133.37268743,811.43997805)
\lineto(-133.37268743,799.09197805)
\curveto(-133.37268743,798.30797883)(-132.84068668,797.83197805)(-132.08468743,797.83197805)
\curveto(-131.41268811,797.83197805)(-131.07668732,798.2239783)(-130.96468743,798.47597805)
\curveto(-130.88068752,798.64397788)(-130.82468743,798.86797827)(-130.82468743,799.09197805)
\lineto(-130.82468743,811.43997805)
\moveto(-133.68068743,816.33997805)
\curveto(-133.68068743,815.47197891)(-133.12068643,814.79997805)(-132.11268743,814.79997805)
\curveto(-131.16068839,814.79997805)(-130.54468743,815.387979)(-130.54468743,816.33997805)
\curveto(-130.54468743,817.37597701)(-131.30068825,817.93597805)(-132.11268743,817.93597805)
\curveto(-132.98068657,817.93597805)(-133.68068743,817.34797704)(-133.68068743,816.33997805)
}
}
{
\newrgbcolor{curcolor}{0 0 0}
\pscustom[linestyle=none,fillstyle=solid,fillcolor=curcolor]
{
\newpath
\moveto(-126.97424993,793.85597805)
\curveto(-126.97424993,793.43597847)(-126.89024937,793.01597774)(-126.33024993,792.70797805)
\curveto(-126.13425013,792.62397813)(-125.91024971,792.56797805)(-125.68624993,792.56797805)
\curveto(-124.48225114,792.56797805)(-124.37024993,793.49197841)(-124.37024993,793.85597805)
\lineto(-124.37024993,799.70797805)
\curveto(-123.16625114,798.64397911)(-122.07424761,797.71997805)(-119.75024993,797.71997805)
\curveto(-115.32625436,797.71997805)(-113.81424993,801.80798152)(-113.81424993,805.27997805)
\curveto(-113.81424993,810.06797326)(-116.47425352,812.81197805)(-120.05824993,812.81197805)
\curveto(-121.85024814,812.81197805)(-123.25025105,812.1959767)(-124.37024993,810.85197805)
\curveto(-124.37024993,811.69197721)(-124.37025058,812.22397835)(-125.01424993,812.53197805)
\curveto(-125.29424965,812.67197791)(-125.57425005,812.67197805)(-125.68624993,812.67197805)
\curveto(-126.86224876,812.67197805)(-126.97424993,811.77597768)(-126.97424993,811.41197805)
\lineto(-126.97424993,793.85597805)
\moveto(-116.47424993,805.22397805)
\curveto(-116.47424993,803.90797936)(-116.64225072,802.33997698)(-117.42624993,801.27597805)
\curveto(-117.9582494,800.54797877)(-118.91025142,799.79197805)(-120.39424993,799.79197805)
\curveto(-123.08224725,799.79197805)(-124.42624993,802.17198129)(-124.42624993,805.41997805)
\curveto(-124.42624993,808.63997483)(-122.9982473,810.65597805)(-120.36624993,810.65597805)
\curveto(-116.58625371,810.65597805)(-116.47424993,806.70797656)(-116.47424993,805.22397805)
}
}
{
\newrgbcolor{curcolor}{0 0 0}
\pscustom[linestyle=none,fillstyle=solid,fillcolor=curcolor]
{
\newpath
\moveto(-255.60799993,781.67597805)
\curveto(-255.60799993,782.09597763)(-255.69200047,782.51597833)(-256.22399993,782.79597805)
\curveto(-256.50399965,782.93597791)(-256.75600005,782.93597805)(-256.86799993,782.93597805)
\curveto(-257.45599935,782.93597805)(-257.82000007,782.59997777)(-257.95999993,782.31997805)
\curveto(-258.04399985,782.12397824)(-258.09999993,781.89997782)(-258.09999993,781.67597805)
\lineto(-258.09999993,775.62797805)
\curveto(-258.46399957,776.10397757)(-259.72400293,777.78397805)(-262.71999993,777.78397805)
\curveto(-267.19999545,777.78397805)(-268.76799993,773.89197449)(-268.76799993,770.33597805)
\curveto(-268.76799993,766.38798199)(-266.97599551,762.71997805)(-262.55199993,762.71997805)
\curveto(-261.37600111,762.71997805)(-260.25599893,762.97197872)(-259.24799993,763.64397805)
\curveto(-258.63200055,764.06397763)(-258.32399968,764.42797835)(-258.07199993,764.73597805)
\curveto(-258.09999991,763.95197883)(-258.09999982,763.72797779)(-257.98799993,763.47597805)
\curveto(-257.84800007,763.19597833)(-257.51199929,762.83197805)(-256.86799993,762.83197805)
\curveto(-255.72000108,762.83197805)(-255.60799993,763.69997841)(-255.60799993,764.06397805)
\lineto(-255.60799993,781.67597805)
\moveto(-258.07199993,770.22397805)
\curveto(-258.07199993,769.55197872)(-258.07200217,766.4159769)(-260.31199993,765.26797805)
\curveto(-260.87199937,764.95997835)(-261.51600063,764.84797805)(-262.21599993,764.84797805)
\curveto(-265.49199666,764.84797805)(-266.10799993,767.78798048)(-266.10799993,770.22397805)
\curveto(-266.10799993,770.8679774)(-266.05199982,771.51197869)(-265.93999993,772.15597805)
\curveto(-265.54800033,774.11597609)(-264.45599772,775.59997805)(-262.24399993,775.59997805)
\curveto(-261.20800097,775.59997805)(-259.83599901,775.34797653)(-258.91199993,773.83597805)
\curveto(-258.21200063,772.71597917)(-258.07199993,771.3719769)(-258.07199993,770.22397805)
}
}
{
\newrgbcolor{curcolor}{0 0 0}
\pscustom[linestyle=none,fillstyle=solid,fillcolor=curcolor]
{
\newpath
\moveto(-240.37643743,769.55197805)
\curveto(-240.0124378,769.55197805)(-239.11643743,769.63597922)(-239.11643743,770.81197805)
\curveto(-239.11643743,772.37997648)(-239.59243836,774.19997933)(-240.51643743,775.48797805)
\curveto(-241.38443657,776.66397687)(-242.47643869,777.27997835)(-243.73643743,777.58797805)
\curveto(-244.32443685,777.72797791)(-244.94043805,777.78397805)(-245.55643743,777.78397805)
\curveto(-250.51243248,777.78397805)(-252.41643743,773.66797452)(-252.41643743,770.13997805)
\curveto(-252.41643743,768.12398006)(-251.82843617,766.10797662)(-250.56843743,764.67997805)
\curveto(-249.14043886,763.08397964)(-247.29243561,762.71997805)(-245.47243743,762.71997805)
\curveto(-244.68843822,762.71997805)(-242.1684355,762.71998023)(-240.23643743,764.90397805)
\curveto(-239.92843774,765.23997771)(-239.45243732,765.85597858)(-239.34043743,766.38797805)
\curveto(-239.31243746,766.47197796)(-239.31243743,766.55597816)(-239.31243743,766.66797805)
\curveto(-239.31243743,767.47997723)(-240.15243805,767.84397805)(-240.76843743,767.84397805)
\curveto(-241.38443682,767.84397805)(-241.52443788,767.56397729)(-241.97243743,766.80797805)
\curveto(-242.47643693,766.02397883)(-243.26043965,764.87597805)(-245.47243743,764.87597805)
\curveto(-246.59243631,764.87597805)(-247.74043839,765.07197922)(-248.69243743,766.24797805)
\curveto(-249.44843668,767.17197712)(-249.72843749,768.29197931)(-249.78443743,769.55197805)
\lineto(-240.37643743,769.55197805)
\moveto(-249.75643743,771.65197805)
\curveto(-249.61643757,772.2959774)(-249.36443631,773.63997903)(-248.24443743,774.61997805)
\curveto(-247.26443841,775.48797718)(-246.11643685,775.54397805)(-245.52843743,775.54397805)
\curveto(-242.86844009,775.54397805)(-241.86043732,773.4999762)(-241.74843743,771.65197805)
\lineto(-249.75643743,771.65197805)
}
}
{
\newrgbcolor{curcolor}{0 0 0}
\pscustom[linestyle=none,fillstyle=solid,fillcolor=curcolor]
{
\newpath
\moveto(-224.09443743,776.57997805)
\curveto(-224.09443743,776.88797774)(-224.1784385,777.64397805)(-225.24243743,777.64397805)
\curveto(-225.83043685,777.64397805)(-226.11043755,777.33597785)(-226.22243743,777.13997805)
\curveto(-226.33443732,776.91597827)(-226.33443746,776.77597712)(-226.36243743,775.85197805)
\curveto(-226.67043713,776.35597754)(-227.0344385,776.88797852)(-228.09843743,777.36397805)
\curveto(-228.6304369,777.58797782)(-229.35843847,777.78397805)(-230.39443743,777.78397805)
\curveto(-233.44643438,777.78397805)(-235.49043825,776.07597527)(-236.30243743,773.30397805)
\curveto(-236.55443718,772.351979)(-236.66643743,771.39997707)(-236.66643743,770.41997805)
\curveto(-236.66643743,766.72398174)(-234.90243323,763.50397805)(-230.70243743,763.50397805)
\curveto(-228.15443998,763.50397805)(-227.11843679,764.7079788)(-226.47443743,765.46397805)
\lineto(-226.50243743,764.17597805)
\curveto(-226.55843738,762.24397998)(-226.67044077,760.11597805)(-230.00243743,760.11597805)
\curveto(-230.64643679,760.11597805)(-232.04643805,760.19997875)(-232.66243743,760.89997805)
\curveto(-232.85843724,761.12397782)(-232.97043769,761.34797855)(-233.22243743,761.85197805)
\curveto(-233.44643721,762.32797757)(-233.72643827,762.91597805)(-234.56643743,762.91597805)
\curveto(-235.15443685,762.91597805)(-235.91043743,762.55197718)(-235.91043743,761.68397805)
\curveto(-235.91043743,760.9279788)(-235.23843645,759.72397735)(-234.25843743,759.02397805)
\curveto(-233.27843841,758.29597877)(-231.79443581,757.95997805)(-230.17043743,757.95997805)
\curveto(-228.23843937,757.95997805)(-226.55843645,758.43597894)(-225.57843743,759.33197805)
\curveto(-224.45843855,760.36797701)(-224.09443743,761.8519811)(-224.09443743,764.90397805)
\lineto(-224.09443743,776.57997805)
\moveto(-226.44643743,770.81197805)
\curveto(-226.44643743,770.16797869)(-226.50243819,768.34797695)(-227.25843743,767.25597805)
\curveto(-227.87443682,766.35997894)(-229.05043872,765.71597805)(-230.33843743,765.71597805)
\curveto(-231.59843617,765.71597805)(-232.57843799,766.30397883)(-233.13843743,767.08797805)
\curveto(-233.81043676,767.98397715)(-234.03443743,769.35597931)(-234.03443743,770.61597805)
\curveto(-234.03443743,771.7639769)(-233.81043676,773.07997903)(-233.13843743,774.05997805)
\curveto(-232.2704383,775.29197681)(-231.03843651,775.57197805)(-230.11443743,775.57197805)
\curveto(-226.4464411,775.57197805)(-226.44643743,771.53997732)(-226.44643743,770.81197805)
}
}
{
\newrgbcolor{curcolor}{0 0 0}
\pscustom[linestyle=none,fillstyle=solid,fillcolor=curcolor]
{
\newpath
\moveto(-217.78043743,776.49597805)
\curveto(-217.78043743,776.83197771)(-217.89243853,777.67197805)(-218.98443743,777.67197805)
\curveto(-220.10443631,777.67197805)(-220.21643743,776.83197771)(-220.21643743,776.49597805)
\lineto(-220.21643743,764.00797805)
\curveto(-220.21643743,763.61597844)(-220.10443629,762.83197805)(-218.95643743,762.83197805)
\curveto(-217.75243864,762.83197805)(-217.66843743,763.55997849)(-217.66843743,764.00797805)
\lineto(-217.66843743,769.57997805)
\curveto(-217.66843743,771.25997637)(-217.66843559,773.21997933)(-215.82043743,774.50797805)
\curveto(-215.06443819,774.98397757)(-214.33643673,775.09597816)(-213.63643743,775.20797805)
\curveto(-213.24443783,775.26397799)(-212.46043743,775.43197903)(-212.46043743,776.41197805)
\curveto(-212.46043743,777.67197679)(-213.69243774,777.67197805)(-214.00043743,777.67197805)
\curveto(-215.65243578,777.67197805)(-217.24843797,776.63597645)(-217.78043743,775.03997805)
\lineto(-217.78043743,776.49597805)
}
}
{
\newrgbcolor{curcolor}{0 0 0}
\pscustom[linestyle=none,fillstyle=solid,fillcolor=curcolor]
{
\newpath
\moveto(-198.97799993,769.55197805)
\curveto(-198.6140003,769.55197805)(-197.71799993,769.63597922)(-197.71799993,770.81197805)
\curveto(-197.71799993,772.37997648)(-198.19400086,774.19997933)(-199.11799993,775.48797805)
\curveto(-199.98599907,776.66397687)(-201.07800119,777.27997835)(-202.33799993,777.58797805)
\curveto(-202.92599935,777.72797791)(-203.54200055,777.78397805)(-204.15799993,777.78397805)
\curveto(-209.11399498,777.78397805)(-211.01799993,773.66797452)(-211.01799993,770.13997805)
\curveto(-211.01799993,768.12398006)(-210.42999867,766.10797662)(-209.16999993,764.67997805)
\curveto(-207.74200136,763.08397964)(-205.89399811,762.71997805)(-204.07399993,762.71997805)
\curveto(-203.29000072,762.71997805)(-200.769998,762.71998023)(-198.83799993,764.90397805)
\curveto(-198.53000024,765.23997771)(-198.05399982,765.85597858)(-197.94199993,766.38797805)
\curveto(-197.91399996,766.47197796)(-197.91399993,766.55597816)(-197.91399993,766.66797805)
\curveto(-197.91399993,767.47997723)(-198.75400055,767.84397805)(-199.36999993,767.84397805)
\curveto(-199.98599932,767.84397805)(-200.12600038,767.56397729)(-200.57399993,766.80797805)
\curveto(-201.07799943,766.02397883)(-201.86200215,764.87597805)(-204.07399993,764.87597805)
\curveto(-205.19399881,764.87597805)(-206.34200089,765.07197922)(-207.29399993,766.24797805)
\curveto(-208.04999918,767.17197712)(-208.32999999,768.29197931)(-208.38599993,769.55197805)
\lineto(-198.97799993,769.55197805)
\moveto(-208.35799993,771.65197805)
\curveto(-208.21800007,772.2959774)(-207.96599881,773.63997903)(-206.84599993,774.61997805)
\curveto(-205.86600091,775.48797718)(-204.71799935,775.54397805)(-204.12999993,775.54397805)
\curveto(-201.47000259,775.54397805)(-200.46199982,773.4999762)(-200.34999993,771.65197805)
\lineto(-208.35799993,771.65197805)
}
}
{
\newrgbcolor{curcolor}{0 0 0}
\pscustom[linestyle=none,fillstyle=solid,fillcolor=curcolor]
{
\newpath
\moveto(-183.22799993,769.55197805)
\curveto(-182.8640003,769.55197805)(-181.96799993,769.63597922)(-181.96799993,770.81197805)
\curveto(-181.96799993,772.37997648)(-182.44400086,774.19997933)(-183.36799993,775.48797805)
\curveto(-184.23599907,776.66397687)(-185.32800119,777.27997835)(-186.58799993,777.58797805)
\curveto(-187.17599935,777.72797791)(-187.79200055,777.78397805)(-188.40799993,777.78397805)
\curveto(-193.36399498,777.78397805)(-195.26799993,773.66797452)(-195.26799993,770.13997805)
\curveto(-195.26799993,768.12398006)(-194.67999867,766.10797662)(-193.41999993,764.67997805)
\curveto(-191.99200136,763.08397964)(-190.14399811,762.71997805)(-188.32399993,762.71997805)
\curveto(-187.54000072,762.71997805)(-185.019998,762.71998023)(-183.08799993,764.90397805)
\curveto(-182.78000024,765.23997771)(-182.30399982,765.85597858)(-182.19199993,766.38797805)
\curveto(-182.16399996,766.47197796)(-182.16399993,766.55597816)(-182.16399993,766.66797805)
\curveto(-182.16399993,767.47997723)(-183.00400055,767.84397805)(-183.61999993,767.84397805)
\curveto(-184.23599932,767.84397805)(-184.37600038,767.56397729)(-184.82399993,766.80797805)
\curveto(-185.32799943,766.02397883)(-186.11200215,764.87597805)(-188.32399993,764.87597805)
\curveto(-189.44399881,764.87597805)(-190.59200089,765.07197922)(-191.54399993,766.24797805)
\curveto(-192.29999918,767.17197712)(-192.57999999,768.29197931)(-192.63599993,769.55197805)
\lineto(-183.22799993,769.55197805)
\moveto(-192.60799993,771.65197805)
\curveto(-192.46800007,772.2959774)(-192.21599881,773.63997903)(-191.09599993,774.61997805)
\curveto(-190.11600091,775.48797718)(-188.96799935,775.54397805)(-188.37999993,775.54397805)
\curveto(-185.72000259,775.54397805)(-184.71199982,773.4999762)(-184.59999993,771.65197805)
\lineto(-192.60799993,771.65197805)
}
}
{
\newrgbcolor{curcolor}{0 0 0}
\pscustom[linestyle=none,fillstyle=solid,fillcolor=curcolor]
{
\newpath
\moveto(658.9320077,857.38797805)
\lineto(663.8320077,857.38797805)
\curveto(664.14000739,857.38797805)(664.9800077,857.49997911)(664.9800077,858.56397805)
\curveto(664.9800077,859.17997743)(664.64400744,859.51597819)(664.3920077,859.65597805)
\curveto(664.22400786,859.73997796)(664.0280075,859.76797805)(663.8320077,859.76797805)
\lineto(651.3160077,859.76797805)
\curveto(651.008008,859.76797805)(650.1680077,859.65597698)(650.1680077,858.59197805)
\curveto(650.1680077,857.49997914)(650.98000803,857.38797805)(651.3160077,857.38797805)
\lineto(656.2720077,857.38797805)
\lineto(656.2720077,841.11997805)
\curveto(656.2720077,840.72797844)(656.3840089,839.83197805)(657.5880077,839.83197805)
\curveto(658.82000646,839.83197805)(658.90400772,840.72797844)(658.9320077,841.11997805)
\lineto(658.9320077,857.38797805)
}
}
{
\newrgbcolor{curcolor}{0 0 0}
\pscustom[linestyle=none,fillstyle=solid,fillcolor=curcolor]
{
\newpath
\moveto(669.6831327,853.43997805)
\curveto(669.6831327,853.77597771)(669.57113152,854.67197805)(668.3951327,854.67197805)
\curveto(667.24713384,854.67197805)(667.1351327,853.77597771)(667.1351327,853.43997805)
\lineto(667.1351327,841.09197805)
\curveto(667.1351327,840.30797883)(667.66713345,839.83197805)(668.4231327,839.83197805)
\curveto(669.09513202,839.83197805)(669.43113281,840.2239783)(669.5431327,840.47597805)
\curveto(669.62713261,840.64397788)(669.6831327,840.86797827)(669.6831327,841.09197805)
\lineto(669.6831327,853.43997805)
\moveto(666.8271327,858.33997805)
\curveto(666.8271327,857.47197891)(667.3871337,856.79997805)(668.3951327,856.79997805)
\curveto(669.34713174,856.79997805)(669.9631327,857.387979)(669.9631327,858.33997805)
\curveto(669.9631327,859.37597701)(669.20713188,859.93597805)(668.3951327,859.93597805)
\curveto(667.52713356,859.93597805)(666.8271327,859.34797704)(666.8271327,858.33997805)
}
}
{
\newrgbcolor{curcolor}{0 0 0}
\pscustom[linestyle=none,fillstyle=solid,fillcolor=curcolor]
{
\newpath
\moveto(673.5335702,841.09197805)
\curveto(673.5335702,840.75597838)(673.6455714,839.83197805)(674.8495702,839.83197805)
\curveto(675.52156952,839.83197805)(675.88557031,840.2239783)(675.9975702,840.47597805)
\curveto(676.08157011,840.67197785)(676.1375702,840.89597824)(676.1375702,841.09197805)
\lineto(676.1375702,847.50397805)
\curveto(676.1375702,849.01597653)(676.16557084,850.05197905)(676.8095702,851.05997805)
\curveto(677.48156952,852.12397698)(678.51757134,852.54397805)(679.6655702,852.54397805)
\curveto(681.93356793,852.54397805)(682.15757025,850.63997757)(682.2135702,850.16397805)
\curveto(682.26957014,849.74397847)(682.2695702,849.49197673)(682.2695702,848.17597805)
\lineto(682.2695702,841.09197805)
\curveto(682.2695702,840.69997844)(682.35357073,840.25197779)(682.8855702,839.99997805)
\curveto(683.10956997,839.88797816)(683.33357042,839.83197805)(683.5575702,839.83197805)
\curveto(683.69757006,839.83197805)(683.97757045,839.85997819)(684.2295702,839.99997805)
\curveto(684.76156966,840.25197779)(684.8455702,840.69997844)(684.8455702,841.09197805)
\lineto(684.8455702,848.03597805)
\curveto(684.8455702,849.21197687)(684.84557081,850.13597897)(685.4615702,851.05997805)
\curveto(685.90956975,851.73197737)(686.80557171,852.54397805)(688.3175702,852.54397805)
\curveto(690.86556765,852.54397805)(690.9495702,850.38797712)(690.9495702,849.46397805)
\curveto(690.9495702,849.01597849)(690.9775702,848.23197721)(690.9775702,847.39197805)
\lineto(690.9775702,841.11997805)
\curveto(690.9775702,840.69997847)(691.06157076,840.25197779)(691.6215702,839.99997805)
\curveto(691.81757,839.88797816)(692.04157042,839.83197805)(692.2655702,839.83197805)
\curveto(693.46956899,839.83197805)(693.5815702,840.75597841)(693.5815702,841.11997805)
\lineto(693.5815702,849.49197805)
\curveto(693.5815702,850.89197665)(693.52556924,852.17997919)(692.5735702,853.32797805)
\curveto(691.53757123,854.55997681)(690.1095688,854.81197805)(688.7095702,854.81197805)
\curveto(686.58157232,854.81197805)(685.2655691,853.85997673)(684.1735702,852.54397805)
\curveto(683.97757039,852.93597765)(683.58556919,853.71997858)(682.5775702,854.25197805)
\curveto(681.84957092,854.67197763)(681.00956936,854.81197805)(680.1695702,854.81197805)
\curveto(678.40557196,854.81197805)(677.1735691,854.1959767)(676.0815702,852.85197805)
\curveto(676.0815702,853.80397709)(676.08156961,854.25197833)(675.4935702,854.53197805)
\curveto(675.10157059,854.75597782)(674.45756989,854.69997788)(674.1495702,854.53197805)
\curveto(673.61757073,854.2799783)(673.5335702,853.85997765)(673.5335702,853.46797805)
\lineto(673.5335702,841.09197805)
}
}
{
\newrgbcolor{curcolor}{0 0 0}
\pscustom[linestyle=none,fillstyle=solid,fillcolor=curcolor]
{
\newpath
\moveto(708.7720077,846.55197805)
\curveto(709.13600733,846.55197805)(710.0320077,846.63597922)(710.0320077,847.81197805)
\curveto(710.0320077,849.37997648)(709.55600677,851.19997933)(708.6320077,852.48797805)
\curveto(707.76400856,853.66397687)(706.67200644,854.27997835)(705.4120077,854.58797805)
\curveto(704.82400828,854.72797791)(704.20800708,854.78397805)(703.5920077,854.78397805)
\curveto(698.63601265,854.78397805)(696.7320077,850.66797452)(696.7320077,847.13997805)
\curveto(696.7320077,845.12398006)(697.32000896,843.10797662)(698.5800077,841.67997805)
\curveto(700.00800627,840.08397964)(701.85600952,839.71997805)(703.6760077,839.71997805)
\curveto(704.46000691,839.71997805)(706.98000963,839.71998023)(708.9120077,841.90397805)
\curveto(709.22000739,842.23997771)(709.69600781,842.85597858)(709.8080077,843.38797805)
\curveto(709.83600767,843.47197796)(709.8360077,843.55597816)(709.8360077,843.66797805)
\curveto(709.8360077,844.47997723)(708.99600708,844.84397805)(708.3800077,844.84397805)
\curveto(707.76400831,844.84397805)(707.62400725,844.56397729)(707.1760077,843.80797805)
\curveto(706.6720082,843.02397883)(705.88800548,841.87597805)(703.6760077,841.87597805)
\curveto(702.55600882,841.87597805)(701.40800674,842.07197922)(700.4560077,843.24797805)
\curveto(699.70000845,844.17197712)(699.42000764,845.29197931)(699.3640077,846.55197805)
\lineto(708.7720077,846.55197805)
\moveto(699.3920077,848.65197805)
\curveto(699.53200756,849.2959774)(699.78400882,850.63997903)(700.9040077,851.61997805)
\curveto(701.88400672,852.48797718)(703.03200828,852.54397805)(703.6200077,852.54397805)
\curveto(706.28000504,852.54397805)(707.28800781,850.4999762)(707.4000077,848.65197805)
\lineto(699.3920077,848.65197805)
}
}
{
\newrgbcolor{curcolor}{0 0 0}
\pscustom[linestyle=none,fillstyle=solid,fillcolor=curcolor]
{
\newpath
\moveto(668.9320077,585.38797805)
\lineto(673.8320077,585.38797805)
\curveto(674.14000739,585.38797805)(674.9800077,585.49997911)(674.9800077,586.56397805)
\curveto(674.9800077,587.17997743)(674.64400744,587.51597819)(674.3920077,587.65597805)
\curveto(674.22400786,587.73997796)(674.0280075,587.76797805)(673.8320077,587.76797805)
\lineto(661.3160077,587.76797805)
\curveto(661.008008,587.76797805)(660.1680077,587.65597698)(660.1680077,586.59197805)
\curveto(660.1680077,585.49997914)(660.98000803,585.38797805)(661.3160077,585.38797805)
\lineto(666.2720077,585.38797805)
\lineto(666.2720077,569.11997805)
\curveto(666.2720077,568.72797844)(666.3840089,567.83197805)(667.5880077,567.83197805)
\curveto(668.82000646,567.83197805)(668.90400772,568.72797844)(668.9320077,569.11997805)
\lineto(668.9320077,585.38797805)
}
}
{
\newrgbcolor{curcolor}{0 0 0}
\pscustom[linestyle=none,fillstyle=solid,fillcolor=curcolor]
{
\newpath
\moveto(679.6831327,581.43997805)
\curveto(679.6831327,581.77597771)(679.57113152,582.67197805)(678.3951327,582.67197805)
\curveto(677.24713384,582.67197805)(677.1351327,581.77597771)(677.1351327,581.43997805)
\lineto(677.1351327,569.09197805)
\curveto(677.1351327,568.30797883)(677.66713345,567.83197805)(678.4231327,567.83197805)
\curveto(679.09513202,567.83197805)(679.43113281,568.2239783)(679.5431327,568.47597805)
\curveto(679.62713261,568.64397788)(679.6831327,568.86797827)(679.6831327,569.09197805)
\lineto(679.6831327,581.43997805)
\moveto(676.8271327,586.33997805)
\curveto(676.8271327,585.47197891)(677.3871337,584.79997805)(678.3951327,584.79997805)
\curveto(679.34713174,584.79997805)(679.9631327,585.387979)(679.9631327,586.33997805)
\curveto(679.9631327,587.37597701)(679.20713188,587.93597805)(678.3951327,587.93597805)
\curveto(677.52713356,587.93597805)(676.8271327,587.34797704)(676.8271327,586.33997805)
}
}
{
\newrgbcolor{curcolor}{0 0 0}
\pscustom[linestyle=none,fillstyle=solid,fillcolor=curcolor]
{
\newpath
\moveto(683.5335702,569.09197805)
\curveto(683.5335702,568.75597838)(683.6455714,567.83197805)(684.8495702,567.83197805)
\curveto(685.52156952,567.83197805)(685.88557031,568.2239783)(685.9975702,568.47597805)
\curveto(686.08157011,568.67197785)(686.1375702,568.89597824)(686.1375702,569.09197805)
\lineto(686.1375702,575.50397805)
\curveto(686.1375702,577.01597653)(686.16557084,578.05197905)(686.8095702,579.05997805)
\curveto(687.48156952,580.12397698)(688.51757134,580.54397805)(689.6655702,580.54397805)
\curveto(691.93356793,580.54397805)(692.15757025,578.63997757)(692.2135702,578.16397805)
\curveto(692.26957014,577.74397847)(692.2695702,577.49197673)(692.2695702,576.17597805)
\lineto(692.2695702,569.09197805)
\curveto(692.2695702,568.69997844)(692.35357073,568.25197779)(692.8855702,567.99997805)
\curveto(693.10956997,567.88797816)(693.33357042,567.83197805)(693.5575702,567.83197805)
\curveto(693.69757006,567.83197805)(693.97757045,567.85997819)(694.2295702,567.99997805)
\curveto(694.76156966,568.25197779)(694.8455702,568.69997844)(694.8455702,569.09197805)
\lineto(694.8455702,576.03597805)
\curveto(694.8455702,577.21197687)(694.84557081,578.13597897)(695.4615702,579.05997805)
\curveto(695.90956975,579.73197737)(696.80557171,580.54397805)(698.3175702,580.54397805)
\curveto(700.86556765,580.54397805)(700.9495702,578.38797712)(700.9495702,577.46397805)
\curveto(700.9495702,577.01597849)(700.9775702,576.23197721)(700.9775702,575.39197805)
\lineto(700.9775702,569.11997805)
\curveto(700.9775702,568.69997847)(701.06157076,568.25197779)(701.6215702,567.99997805)
\curveto(701.81757,567.88797816)(702.04157042,567.83197805)(702.2655702,567.83197805)
\curveto(703.46956899,567.83197805)(703.5815702,568.75597841)(703.5815702,569.11997805)
\lineto(703.5815702,577.49197805)
\curveto(703.5815702,578.89197665)(703.52556924,580.17997919)(702.5735702,581.32797805)
\curveto(701.53757123,582.55997681)(700.1095688,582.81197805)(698.7095702,582.81197805)
\curveto(696.58157232,582.81197805)(695.2655691,581.85997673)(694.1735702,580.54397805)
\curveto(693.97757039,580.93597765)(693.58556919,581.71997858)(692.5775702,582.25197805)
\curveto(691.84957092,582.67197763)(691.00956936,582.81197805)(690.1695702,582.81197805)
\curveto(688.40557196,582.81197805)(687.1735691,582.1959767)(686.0815702,580.85197805)
\curveto(686.0815702,581.80397709)(686.08156961,582.25197833)(685.4935702,582.53197805)
\curveto(685.10157059,582.75597782)(684.45756989,582.69997788)(684.1495702,582.53197805)
\curveto(683.61757073,582.2799783)(683.5335702,581.85997765)(683.5335702,581.46797805)
\lineto(683.5335702,569.09197805)
}
}
{
\newrgbcolor{curcolor}{0 0 0}
\pscustom[linestyle=none,fillstyle=solid,fillcolor=curcolor]
{
\newpath
\moveto(718.7720077,574.55197805)
\curveto(719.13600733,574.55197805)(720.0320077,574.63597922)(720.0320077,575.81197805)
\curveto(720.0320077,577.37997648)(719.55600677,579.19997933)(718.6320077,580.48797805)
\curveto(717.76400856,581.66397687)(716.67200644,582.27997835)(715.4120077,582.58797805)
\curveto(714.82400828,582.72797791)(714.20800708,582.78397805)(713.5920077,582.78397805)
\curveto(708.63601265,582.78397805)(706.7320077,578.66797452)(706.7320077,575.13997805)
\curveto(706.7320077,573.12398006)(707.32000896,571.10797662)(708.5800077,569.67997805)
\curveto(710.00800627,568.08397964)(711.85600952,567.71997805)(713.6760077,567.71997805)
\curveto(714.46000691,567.71997805)(716.98000963,567.71998023)(718.9120077,569.90397805)
\curveto(719.22000739,570.23997771)(719.69600781,570.85597858)(719.8080077,571.38797805)
\curveto(719.83600767,571.47197796)(719.8360077,571.55597816)(719.8360077,571.66797805)
\curveto(719.8360077,572.47997723)(718.99600708,572.84397805)(718.3800077,572.84397805)
\curveto(717.76400831,572.84397805)(717.62400725,572.56397729)(717.1760077,571.80797805)
\curveto(716.6720082,571.02397883)(715.88800548,569.87597805)(713.6760077,569.87597805)
\curveto(712.55600882,569.87597805)(711.40800674,570.07197922)(710.4560077,571.24797805)
\curveto(709.70000845,572.17197712)(709.42000764,573.29197931)(709.3640077,574.55197805)
\lineto(718.7720077,574.55197805)
\moveto(709.3920077,576.65197805)
\curveto(709.53200756,577.2959774)(709.78400882,578.63997903)(710.9040077,579.61997805)
\curveto(711.88400672,580.48797718)(713.03200828,580.54397805)(713.6200077,580.54397805)
\curveto(716.28000504,580.54397805)(717.28800781,578.4999762)(717.4000077,576.65197805)
\lineto(709.3920077,576.65197805)
}
}
{
\newrgbcolor{curcolor}{0 0 0}
\pscustom[linewidth=1,linecolor=curcolor]
{
\newpath
\moveto(-81.4286,1064.28576262)
\lineto(-81.4286,1104.28576262)
}
}
{
\newrgbcolor{curcolor}{0 0 0}
\pscustom[linestyle=none,fillstyle=solid,fillcolor=curcolor]
{
\newpath
\moveto(-81.4286,1108.90176262)
\lineto(-77.4286,1101.98176262)
\lineto(-85.4286,1101.98176262)
\lineto(-81.4286,1108.90176262)
\closepath
}
}
{
\newrgbcolor{curcolor}{0 0 0}
\pscustom[linewidth=1,linecolor=curcolor]
{
\newpath
\moveto(-81.4286,1108.90176262)
\lineto(-77.4286,1101.98176262)
\lineto(-85.4286,1101.98176262)
\lineto(-81.4286,1108.90176262)
\closepath
}
}
{
\newrgbcolor{curcolor}{0 0 0}
\pscustom[linewidth=1,linecolor=curcolor]
{
\newpath
\moveto(-79.71428,785.78571262)
\lineto(-79.71428,825.78570262)
}
}
{
\newrgbcolor{curcolor}{0 0 0}
\pscustom[linestyle=none,fillstyle=solid,fillcolor=curcolor]
{
\newpath
\moveto(-79.71428,830.40170262)
\lineto(-75.71428,823.48170262)
\lineto(-83.71428,823.48170262)
\lineto(-79.71428,830.40170262)
\closepath
}
}
{
\newrgbcolor{curcolor}{0 0 0}
\pscustom[linewidth=1,linecolor=curcolor]
{
\newpath
\moveto(-79.71428,830.40170262)
\lineto(-75.71428,823.48170262)
\lineto(-83.71428,823.48170262)
\lineto(-79.71428,830.40170262)
\closepath
}
}
{
\newrgbcolor{curcolor}{0 0 0}
\pscustom[linewidth=1,linecolor=curcolor]
{
\newpath
\moveto(605,568.00001262)
\lineto(650,568.00001262)
\lineto(650,568.00001262)
}
}
{
\newrgbcolor{curcolor}{0 0 0}
\pscustom[linestyle=none,fillstyle=solid,fillcolor=curcolor]
{
\newpath
\moveto(654.616,568.00001262)
\lineto(647.696,564.00001262)
\lineto(647.696,572.00001262)
\lineto(654.616,568.00001262)
\closepath
}
}
{
\newrgbcolor{curcolor}{0 0 0}
\pscustom[linewidth=1,linecolor=curcolor]
{
\newpath
\moveto(654.616,568.00001262)
\lineto(647.696,564.00001262)
\lineto(647.696,572.00001262)
\lineto(654.616,568.00001262)
\closepath
}
}
{
\newrgbcolor{curcolor}{0 0 0}
\pscustom[linewidth=1,linecolor=curcolor]
{
\newpath
\moveto(598.39291,849.10715262)
\lineto(643.39291,849.10715262)
\lineto(643.39291,849.10715262)
}
}
{
\newrgbcolor{curcolor}{0 0 0}
\pscustom[linestyle=none,fillstyle=solid,fillcolor=curcolor]
{
\newpath
\moveto(648.00891,849.10715262)
\lineto(641.08891,845.10715262)
\lineto(641.08891,853.10715262)
\lineto(648.00891,849.10715262)
\closepath
}
}
{
\newrgbcolor{curcolor}{0 0 0}
\pscustom[linewidth=1,linecolor=curcolor]
{
\newpath
\moveto(648.00891,849.10715262)
\lineto(641.08891,845.10715262)
\lineto(641.08891,853.10715262)
\lineto(648.00891,849.10715262)
\closepath
}
}
\end{pspicture}

\caption{Trapezoidal possibility distribution $\left[\alpha,\ \beta,\ \gamma,\ \delta\right]$. }
\label{fig:trapezoidal}
\end{figure}