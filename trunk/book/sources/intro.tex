%%%%%%%%%%%%%%%%%%%%%%%%%%%%%%%%%%%%%%%%%%%%%%%%%%%%%%%%%%%%%%%%%%%%%%
%
% Introduction
%
%%%%%%%%%%%%%%%%%%%%%%%%%%%%%%%%%%%%%%%%%%%%%%%%%%%%%%%%%%%%%%%%%%%%%%
The concept of \emph{time} is easy to understand but very complex to define~\cite{klein94}~\cite{Shackle61}. 
In Information Systems \emph{IS}, several proposals have been concerned with the obtaining of theoretical models that allow the representation of time~\cite{Bolour82}~\cite{Cru97}. 
More towards temporal databases, a lot of work have been done. The first efforts were towards the representation of the historical information related to an object in the database(reference). Some works tried to extend the Entity Relationship Model \emph{ERM} ~\cite{Klopprogge:1983} but without an impact on any database standards like SQL~\cite{Sarda:1990:ESH:627277.627409}.

The ``Consensus Glossary of Temporal Database concepts'' \cite{Dyreson1994} is the first publication where the main researchers in temporal databases define the main thesaurus of temporal database concepts. This glossary defines the main types of time in a temporal database and is the basis of furthers researches ~\cite{Sarda:1990:ESH:627277.627409} \cite{Jensen94thetsql2}.

Another main issue in temporal databases is the querying. The user specifies in the query some temporal constraints, e.g. \emph{``before this year''},\emph{``after 1995''}. In order to solve these temporal comparisons, between the time specified in the query and the temporal data stored in the database, some basic relationships have been defined. Allen~\cite{Allen83} first introduced the temporal relations between time intervals and, to a lesser extent, to time points. Some authors~\cite{ohlbach2004},\cite{nagypal2003},\cite{schockaert08} have softened the temporal operators defined by Allen. Therefore it is possible to compute the relationships between two ill-known time points or intervals. In~\cite{garrido2009}, some different temporal operators are defined by a combination of regular fuzzy comparisons. %e.g. the \emph{BEFORE} operator is defined by means of the fuzzy less than (FLT) operator. 

To allow information systems to cope with uncertainty, many approaches adopt fuzzy sets~\cite{Zadeh65} for the representation of temporal information~\cite{Billiet:Pons:Matthe:DeTre:Pons:2011:BipolarFuzzy},\cite{Dubois:jucs_9_9:fuzziness_and_uncertainty_in},~\cite{devos94}. The time points representing the time when the object is valid in the reality might not be precisely known. In order to deal with this, several models have been proposed. Garrido~\cite{garrido2009} proposes a model to deal with uncertainty in the time interval by means of fuzzy intervals. Bronselaer~\cite{Pon11} proposes a consistent framework to deal and represent ill-known time values and their relationships. Rough sets~\cite{Pawlak1995} have been also used to represent time intervals~\cite{Qia09} in databases.

%temporal reasoning and fuzzy temporal reasoning.
Next to the manage of temporal information is the temporal reasoning~\cite{Allen83}. Dubois and Prade ~\cite{Dubois:jucs_9_9:fuzziness_and_uncertainty_in},\cite{Dubois89} deal with fuzziness and uncertainty in temporal reasoning, but this topic is outside the scope of this work.

%temporal commercial systems.

The work is organized as follows: 

In Section \ref{sec:time-domain} an study about the time and its properties is done. Section \ref{sec:temporal-databases} is an overview of the main problems when managing the time in a database, the different proposals for solving them and the temporal database proposals. The section concludes with a comparison among the different proposals. In Section \ref{sec:fuzzy-temporal-databases} we present a sum up with the proposals for handling the imprecision in temporal databases and the main deficiencies we have detected. Finally, there are several commercial temporal database management systems  \emph{TDMBS} like \cite{oracle2009},\cite{posgree2009},\cite{teradata2011},\cite{timedb2005}. All of them are analyzed and compared later on this work. 
