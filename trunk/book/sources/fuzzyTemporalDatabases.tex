%
% Fuzzy temporal databases
%
Some software applications deal with uncertainty related to time. e.g. a logistics company which transports packages from one destination to another. The time when the package would be at the end point may be estimated but not precisely known. These applications need to manage the uncertainty with respect to the temporal attributes of the objects stored in the database.

There are several approaches to manage uncertainty in the temporal information stored in a database. Valid-time is usually represented as an interval, and therefore it is suitable for the representation of uncertainty in the valid-time interval. Nevertheless, the values for transaction and decision time are usually precisely known.
Some models deal with probability~\cite{Dekhtyar2001} or possibility~\cite{Dubois89} distributions at the endpoints of the time interval. In this section we will explain the most interesting proposals that deal with possibility distributions in the valid-time interval. %Two are the main points developed within this section: the representation of uncertainty in the time interval and the flexible querying of ill-known time periods.

In fuzzy databases~\cite{Galindo2006}, uncertainty is managed at storage level whereas imprecision is managed at querying level. In subsection \ref{subsec:representation-time-intervals} we will explain the main approach for the storage of uncertainty in the valid-time interval. Finally, in subsection \ref{subsec:querying-time-intervals} the proposals for querying in a flexible way valid-time intervals are analyzed.

\subsection{Representation of ill-known temporal intervals}
\label{subsec:representation-time-intervals}
The uncertainty in a valid-time interval may be represented with uncertainty at one or both starting and ending points. The uncertainty is usually represented by means of a possibility distribution in one or both points. It is also possible to represent with rough sets~\cite{Pawlak1995} time intervals~\cite{Qia09}.

Several authors ~\cite{garrido2009} propose some transformations in order to optimize the storage of the valid-time interval. Further research has shown how these transformations drive to a possibilistic information lost~\cite{Pon11}. A comparison between the transformations in ~\cite{garrido2009}  and the framework in~\cite{Pon11} is done in~\cite{pon12}. 

\begin{figure}
\centering
%%Created by jPicEdt 1.4.1_03: mixed JPIC-XML/LaTeX format
%%Wed Nov 23 16:38:02 CET 2011
%%Begin JPIC-XML
%<?xml version="1.0" standalone="yes"?>
%<jpic x-min="-2.5" x-max="60" y-min="-2.5" y-max="32.5" auto-bounding="true">
%<multicurve fill-style= "none"
%	 points= "(0,0);(0,0);(55,0);(55,0)"
%	 right-arrow= "head"
%	 />
%<multicurve fill-style= "none"
%	 points= "(0,0);(0,0);(0,30);(0,30)"
%	 right-arrow= "head"
%	 />
%<text fill-style= "none"
%	 text-vert-align= "center-v"
%	 anchor-point= "(-2.5,27.5)"
%	 text-frame= "noframe"
%	 text-hor-align= "center-h"
%	 right-arrow= "head"
%	 >
%1
%</text>
%<text fill-style= "none"
%	 text-vert-align= "center-v"
%	 anchor-point= "(-2.5,0)"
%	 text-frame= "noframe"
%	 text-hor-align= "center-h"
%	 right-arrow= "head"
%	 >
%0
%</text>
%<text fill-style= "none"
%	 text-vert-align= "center-v"
%	 anchor-point= "(15,32.5)"
%	 text-rotation= "135"
%	 text-frame= "noframe"
%	 text-hor-align= "center-h"
%	 right-arrow= "head"
%	 >
%Membership Degree
%</text>
%<text fill-style= "none"
%	 text-vert-align= "center-v"
%	 anchor-point= "(10,-2.5)"
%	 text-frame= "noframe"
%	 text-hor-align= "center-h"
%	 right-arrow= "head"
%	 >
%2
%</text>
%<text fill-style= "none"
%	 text-vert-align= "center-v"
%	 anchor-point= "(15,-2.5)"
%	 text-frame= "noframe"
%	 text-hor-align= "center-h"
%	 right-arrow= "head"
%	 >
%3
%</text>
%<text fill-style= "none"
%	 text-vert-align= "center-v"
%	 anchor-point= "(20,-2.5)"
%	 text-frame= "noframe"
%	 text-hor-align= "center-h"
%	 right-arrow= "head"
%	 >
%4
%</text>
%<text fill-style= "none"
%	 text-vert-align= "center-v"
%	 anchor-point= "(25,-2.5)"
%	 text-frame= "noframe"
%	 text-hor-align= "center-h"
%	 right-arrow= "head"
%	 >
%5
%</text>
%<text fill-style= "none"
%	 text-vert-align= "center-v"
%	 anchor-point= "(30,-2.5)"
%	 text-frame= "noframe"
%	 text-hor-align= "center-h"
%	 right-arrow= "head"
%	 >
%6
%</text>
%<text fill-style= "none"
%	 text-vert-align= "center-v"
%	 anchor-point= "(35,-2.5)"
%	 text-frame= "noframe"
%	 text-hor-align= "center-h"
%	 right-arrow= "head"
%	 >
%7
%</text>
%<text fill-style= "none"
%	 text-vert-align= "center-v"
%	 anchor-point= "(40,-2.5)"
%	 text-frame= "noframe"
%	 text-hor-align= "center-h"
%	 right-arrow= "head"
%	 >
%8
%</text>
%<text fill-style= "none"
%	 text-vert-align= "center-v"
%	 anchor-point= "(45,-2.5)"
%	 text-frame= "noframe"
%	 text-hor-align= "center-h"
%	 right-arrow= "head"
%	 >
%9
%</text>
%<text fill-style= "none"
%	 text-vert-align= "center-v"
%	 anchor-point= "(50,-2.5)"
%	 text-frame= "noframe"
%	 text-hor-align= "center-h"
%	 right-arrow= "head"
%	 >
%10
%</text>
%<text fill-style= "none"
%	 text-vert-align= "center-v"
%	 anchor-point= "(5,-2.5)"
%	 text-frame= "noframe"
%	 text-hor-align= "center-h"
%	 right-arrow= "head"
%	 >
%1
%</text>
%<multicurve fill-style= "none"
%	 points= "(5,0);(5,0);(15,27.5);(15,27.5)"
%	 />
%<multicurve fill-style= "none"
%	 points= "(15,27.5);(15,27.5);(20,0);(20,0)"
%	 />
%<multicurve fill-style= "none"
%	 points= "(25,0);(25,0);(35,27.5);(35,27.5)"
%	 />
%<multicurve fill-style= "none"
%	 points= "(35,27.5);(35,27.5);(50,0);(50,0)"
%	 />
%<text fill-style= "none"
%	 text-vert-align= "center-v"
%	 anchor-point= "(10,25)"
%	 text-frame= "noframe"
%	 text-hor-align= "center-h"
%	 >
%X
%</text>
%<text fill-style= "none"
%	 text-vert-align= "center-v"
%	 anchor-point= "(40,25)"
%	 text-frame= "noframe"
%	 text-hor-align= "center-h"
%	 >
%Y
%</text>
%<text fill-style= "none"
%	 text-vert-align= "center-v"
%	 anchor-point= "(60,0)"
%	 text-frame= "noframe"
%	 text-hor-align= "center-h"
%	 >
%Time
%</text>
%</jpic>
%%End JPIC-XML
%LaTeX-picture environment using emulated lines and arcs
%You can rescale the whole picture (to 80% for instance) by using the command \def\JPicScale{0.8}
\ifx\JPicScale\undefined\def\JPicScale{1}\fi
\unitlength \JPicScale mm
\begin{picture}(60,32.5)(0,0)
\linethickness{0.3mm}
\put(0,0){\line(1,0){55}}
\put(55,0){\vector(1,0){0.12}}
\linethickness{0.3mm}
\put(0,0){\line(0,1){30}}
\put(0,30){\vector(0,1){0.12}}
\put(-2.5,27.5){\makebox(0,0)[cc]{1}}

\put(-2.5,0){\makebox(0,0)[cc]{0}}

\put(15,32.5){\makebox(0,0)[cc]{Membership Degree}}

\put(10,-2.5){\makebox(0,0)[cc]{2}}

\put(15,-2.5){\makebox(0,0)[cc]{3}}

\put(20,-2.5){\makebox(0,0)[cc]{4}}

\put(25,-2.5){\makebox(0,0)[cc]{5}}

\put(30,-2.5){\makebox(0,0)[cc]{6}}

\put(35,-2.5){\makebox(0,0)[cc]{7}}

\put(40,-2.5){\makebox(0,0)[cc]{8}}

\put(45,-2.5){\makebox(0,0)[cc]{9}}

\put(50,-2.5){\makebox(0,0)[cc]{10}}

\put(5,-2.5){\makebox(0,0)[cc]{1}}

\linethickness{0.3mm}
\multiput(5,0)(0.12,0.33){83}{\line(0,1){0.33}}
\linethickness{0.3mm}
\multiput(15,27.5)(0.12,-0.65){42}{\line(0,-1){0.65}}
\linethickness{0.3mm}
\multiput(25,0)(0.12,0.33){83}{\line(0,1){0.33}}
\linethickness{0.3mm}
\multiput(35,27.5)(0.12,-0.22){125}{\line(0,-1){0.22}}
\put(10,25){\makebox(0,0)[cc]{X}}

\put(40,25){\makebox(0,0)[cc]{Y}}

\put(60,0){\makebox(0,0)[cc]{Time}}

\end{picture}

\caption{Two ill-known values $X$ and $Y$, represented by means of triangular possibility distributions denoting the starting and ending points of an ill-known interval. }
\label{fig:interval}
\end{figure}


An ill-known temporal interval is usually represented by means of two possibility distributions one in each starting and ending points. Figure \ref{fig:interval} shows an ill-known valid-time interval. The starting and ending points are denoted by two fuzzy numbers, $X$ and $Y$, represented by means of two triangular possibility distributions (see the Appendix). For illustration, figure \ref{fig:convexhull} shows one of the transformations based in the convex hull and proposed in~\cite{garrido2009}.

\begin{figure}
\centering
%%Created by jPicEdt 1.4.1_03: mixed JPIC-XML/LaTeX format
%%Mon Nov 28 16:07:12 CET 2011
%%Begin JPIC-XML
%<?xml version="1.0" standalone="yes"?>
%<jpic x-min="-2.5" x-max="60" y-min="-2.5" y-max="72.5" auto-bounding="true">
%<multicurve fill-style= "none"
%	 points= "(0,40);(0,40);(55,40);(55,40)"
%	 right-arrow= "head"
%	 />
%<multicurve fill-style= "none"
%	 points= "(0,40);(0,40);(0,70);(0,70)"
%	 right-arrow= "head"
%	 />
%<text text-vert-align= "center-v"
%	 anchor-point= "(-2.5,67.5)"
%	 fill-style= "none"
%	 text-frame= "noframe"
%	 text-hor-align= "center-h"
%	 right-arrow= "head"
%	 >
%1
%</text>
%<text text-vert-align= "center-v"
%	 anchor-point= "(-2.5,40)"
%	 fill-style= "none"
%	 text-frame= "noframe"
%	 text-hor-align= "center-h"
%	 right-arrow= "head"
%	 >
%0
%</text>
%<text text-vert-align= "center-v"
%	 anchor-point= "(15,72.5)"
%	 fill-style= "none"
%	 text-rotation= "135"
%	 text-frame= "noframe"
%	 text-hor-align= "center-h"
%	 right-arrow= "head"
%	 >
%Membership Degree
%</text>
%<text text-vert-align= "center-v"
%	 anchor-point= "(10,37.5)"
%	 fill-style= "none"
%	 text-frame= "noframe"
%	 text-hor-align= "center-h"
%	 right-arrow= "head"
%	 >
%2
%</text>
%<text text-vert-align= "center-v"
%	 anchor-point= "(15,37.5)"
%	 fill-style= "none"
%	 text-frame= "noframe"
%	 text-hor-align= "center-h"
%	 right-arrow= "head"
%	 >
%3
%</text>
%<text text-vert-align= "center-v"
%	 anchor-point= "(20,37.5)"
%	 fill-style= "none"
%	 text-frame= "noframe"
%	 text-hor-align= "center-h"
%	 right-arrow= "head"
%	 >
%4
%</text>
%<text text-vert-align= "center-v"
%	 anchor-point= "(25,37.5)"
%	 fill-style= "none"
%	 text-frame= "noframe"
%	 text-hor-align= "center-h"
%	 right-arrow= "head"
%	 >
%5
%</text>
%<text text-vert-align= "center-v"
%	 anchor-point= "(30,37.5)"
%	 fill-style= "none"
%	 text-frame= "noframe"
%	 text-hor-align= "center-h"
%	 right-arrow= "head"
%	 >
%6
%</text>
%<text text-vert-align= "center-v"
%	 anchor-point= "(35,37.5)"
%	 fill-style= "none"
%	 text-frame= "noframe"
%	 text-hor-align= "center-h"
%	 right-arrow= "head"
%	 >
%7
%</text>
%<text text-vert-align= "center-v"
%	 anchor-point= "(40,37.5)"
%	 fill-style= "none"
%	 text-frame= "noframe"
%	 text-hor-align= "center-h"
%	 right-arrow= "head"
%	 >
%8
%</text>
%<text text-vert-align= "center-v"
%	 anchor-point= "(45,37.5)"
%	 fill-style= "none"
%	 text-frame= "noframe"
%	 text-hor-align= "center-h"
%	 right-arrow= "head"
%	 >
%9
%</text>
%<text text-vert-align= "center-v"
%	 anchor-point= "(50,37.5)"
%	 fill-style= "none"
%	 text-frame= "noframe"
%	 text-hor-align= "center-h"
%	 right-arrow= "head"
%	 >
%10
%</text>
%<text text-vert-align= "center-v"
%	 anchor-point= "(5,37.5)"
%	 fill-style= "none"
%	 text-frame= "noframe"
%	 text-hor-align= "center-h"
%	 right-arrow= "head"
%	 >
%1
%</text>
%<multicurve fill-style= "none"
%	 points= "(5,40);(5,40);(15,67.5);(15,67.5)"
%	 />
%<multicurve fill-style= "none"
%	 points= "(15,67.5);(15,67.5);(20,40);(20,40)"
%	 />
%<multicurve fill-style= "none"
%	 points= "(25,40);(25,40);(35,67.5);(35,67.5)"
%	 />
%<multicurve fill-style= "none"
%	 points= "(35,67.5);(35,67.5);(50,40);(50,40)"
%	 />
%<text text-vert-align= "center-v"
%	 anchor-point= "(10,65)"
%	 fill-style= "none"
%	 text-frame= "noframe"
%	 text-hor-align= "center-h"
%	 >
%X
%</text>
%<text text-vert-align= "center-v"
%	 anchor-point= "(40,65)"
%	 fill-style= "none"
%	 text-frame= "noframe"
%	 text-hor-align= "center-h"
%	 >
%Y
%</text>
%<text text-vert-align= "center-v"
%	 anchor-point= "(60,40)"
%	 fill-style= "none"
%	 text-frame= "noframe"
%	 text-hor-align= "center-h"
%	 >
%Time
%</text>
%<multicurve fill-style= "none"
%	 points= "(0,0);(0,0);(55,0);(55,0)"
%	 right-arrow= "head"
%	 />
%<multicurve fill-style= "none"
%	 points= "(0,0);(0,0);(0,30);(0,30)"
%	 right-arrow= "head"
%	 />
%<text text-vert-align= "center-v"
%	 anchor-point= "(-2.5,27.5)"
%	 fill-style= "none"
%	 text-frame= "noframe"
%	 text-hor-align= "center-h"
%	 right-arrow= "head"
%	 >
%1
%</text>
%<text text-vert-align= "center-v"
%	 anchor-point= "(-2.5,0)"
%	 fill-style= "none"
%	 text-frame= "noframe"
%	 text-hor-align= "center-h"
%	 right-arrow= "head"
%	 >
%0
%</text>
%<text text-vert-align= "center-v"
%	 anchor-point= "(15,32.5)"
%	 fill-style= "none"
%	 text-rotation= "135"
%	 text-frame= "noframe"
%	 text-hor-align= "center-h"
%	 right-arrow= "head"
%	 >
%Membership Degree
%</text>
%<text text-vert-align= "center-v"
%	 anchor-point= "(10,-2.5)"
%	 fill-style= "none"
%	 text-frame= "noframe"
%	 text-hor-align= "center-h"
%	 right-arrow= "head"
%	 >
%2
%</text>
%<text text-vert-align= "center-v"
%	 anchor-point= "(15,-2.5)"
%	 fill-style= "none"
%	 text-frame= "noframe"
%	 text-hor-align= "center-h"
%	 right-arrow= "head"
%	 >
%3
%</text>
%<text text-vert-align= "center-v"
%	 anchor-point= "(20,-2.5)"
%	 fill-style= "none"
%	 text-frame= "noframe"
%	 text-hor-align= "center-h"
%	 right-arrow= "head"
%	 >
%4
%</text>
%<text text-vert-align= "center-v"
%	 anchor-point= "(25,-2.5)"
%	 fill-style= "none"
%	 text-frame= "noframe"
%	 text-hor-align= "center-h"
%	 right-arrow= "head"
%	 >
%5
%</text>
%<text text-vert-align= "center-v"
%	 anchor-point= "(30,-2.5)"
%	 fill-style= "none"
%	 text-frame= "noframe"
%	 text-hor-align= "center-h"
%	 right-arrow= "head"
%	 >
%6
%</text>
%<text text-vert-align= "center-v"
%	 anchor-point= "(35,-2.5)"
%	 fill-style= "none"
%	 text-frame= "noframe"
%	 text-hor-align= "center-h"
%	 right-arrow= "head"
%	 >
%7
%</text>
%<text text-vert-align= "center-v"
%	 anchor-point= "(40,-2.5)"
%	 fill-style= "none"
%	 text-frame= "noframe"
%	 text-hor-align= "center-h"
%	 right-arrow= "head"
%	 >
%8
%</text>
%<text text-vert-align= "center-v"
%	 anchor-point= "(45,-2.5)"
%	 fill-style= "none"
%	 text-frame= "noframe"
%	 text-hor-align= "center-h"
%	 right-arrow= "head"
%	 >
%9
%</text>
%<text text-vert-align= "center-v"
%	 anchor-point= "(50,-2.5)"
%	 fill-style= "none"
%	 text-frame= "noframe"
%	 text-hor-align= "center-h"
%	 right-arrow= "head"
%	 >
%10
%</text>
%<text text-vert-align= "center-v"
%	 anchor-point= "(5,-2.5)"
%	 fill-style= "none"
%	 text-frame= "noframe"
%	 text-hor-align= "center-h"
%	 right-arrow= "head"
%	 >
%1
%</text>
%<multicurve fill-style= "none"
%	 points= "(5,0);(5,0);(15,27.5);(15,27.5)"
%	 />
%<multicurve fill-style= "none"
%	 points= "(15,27.5);(15,27.5);(35,27.5);(35,27.5)"
%	 />
%<multicurve fill-style= "none"
%	 points= "(35,27.5);(35,27.5);(50,0);(50,0)"
%	 />
%<text text-vert-align= "center-v"
%	 anchor-point= "(60,0)"
%	 fill-style= "none"
%	 text-frame= "noframe"
%	 text-hor-align= "center-h"
%	 >
%Time
%</text>
%<text text-vert-align= "center-v"
%	 stroke-dasharray= "1;1"
%	 anchor-point= "(40,25)"
%	 stroke-style= "dashed"
%	 fill-style= "none"
%	 text-frame= "noframe"
%	 text-hor-align= "center-h"
%	 >
%T
%</text>
%</jpic>
%%End JPIC-XML
%LaTeX-picture environment using emulated lines and arcs
%You can rescale the whole picture (to 80% for instance) by using the command \def\JPicScale{0.8}
\ifx\JPicScale\undefined\def\JPicScale{1}\fi
\unitlength \JPicScale mm
\begin{picture}(60,72.5)(0,0)
\linethickness{0.3mm}
\put(0,40){\line(1,0){55}}
\put(55,40){\vector(1,0){0.12}}
\linethickness{0.3mm}
\put(0,40){\line(0,1){30}}
\put(0,70){\vector(0,1){0.12}}
\put(-2.5,67.5){\makebox(0,0)[cc]{1}}

\put(-2.5,40){\makebox(0,0)[cc]{0}}

\put(15,72.5){\makebox(0,0)[cc]{Membership Degree}}

\put(10,37.5){\makebox(0,0)[cc]{2}}

\put(15,37.5){\makebox(0,0)[cc]{3}}

\put(20,37.5){\makebox(0,0)[cc]{4}}

\put(25,37.5){\makebox(0,0)[cc]{5}}

\put(30,37.5){\makebox(0,0)[cc]{6}}

\put(35,37.5){\makebox(0,0)[cc]{7}}

\put(40,37.5){\makebox(0,0)[cc]{8}}

\put(45,37.5){\makebox(0,0)[cc]{9}}

\put(50,37.5){\makebox(0,0)[cc]{10}}

\put(5,37.5){\makebox(0,0)[cc]{1}}

\linethickness{0.3mm}
\multiput(5,40)(0.12,0.33){83}{\line(0,1){0.33}}
\linethickness{0.3mm}
\multiput(15,67.5)(0.12,-0.65){42}{\line(0,-1){0.65}}
\linethickness{0.3mm}
\multiput(25,40)(0.12,0.33){83}{\line(0,1){0.33}}
\linethickness{0.3mm}
\multiput(35,67.5)(0.12,-0.22){125}{\line(0,-1){0.22}}
\put(10,65){\makebox(0,0)[cc]{X}}

\put(40,65){\makebox(0,0)[cc]{Y}}

\put(60,40){\makebox(0,0)[cc]{Time}}

\linethickness{0.3mm}
\put(0,0){\line(1,0){55}}
\put(55,0){\vector(1,0){0.12}}
\linethickness{0.3mm}
\put(0,0){\line(0,1){30}}
\put(0,30){\vector(0,1){0.12}}
\put(-2.5,27.5){\makebox(0,0)[cc]{1}}

\put(-2.5,0){\makebox(0,0)[cc]{0}}

\put(15,32.5){\makebox(0,0)[cc]{Membership Degree}}

\put(10,-2.5){\makebox(0,0)[cc]{2}}

\put(15,-2.5){\makebox(0,0)[cc]{3}}

\put(20,-2.5){\makebox(0,0)[cc]{4}}

\put(25,-2.5){\makebox(0,0)[cc]{5}}

\put(30,-2.5){\makebox(0,0)[cc]{6}}

\put(35,-2.5){\makebox(0,0)[cc]{7}}

\put(40,-2.5){\makebox(0,0)[cc]{8}}

\put(45,-2.5){\makebox(0,0)[cc]{9}}

\put(50,-2.5){\makebox(0,0)[cc]{10}}

\put(5,-2.5){\makebox(0,0)[cc]{1}}

\linethickness{0.3mm}
\multiput(5,0)(0.12,0.33){83}{\line(0,1){0.33}}
\linethickness{0.3mm}
\put(15,27.5){\line(1,0){20}}
\linethickness{0.3mm}
\multiput(35,27.5)(0.12,-0.22){125}{\line(0,-1){0.22}}
\put(60,0){\makebox(0,0)[cc]{Time}}

\put(40,25){\makebox(0,0)[cc]{T}}

\end{picture}

\caption{Transformation based in the convex hull from the two ill-known points $X$ and $Y$. }
\label{fig:convexhull}
\end{figure}

\subsection{Querying fuzzy temporal databases}
\label{subsec:querying-time-intervals}


\begin{table}[h]

\caption{Allen's relations used in the framework. Here, $I = \left[a, b\right]$ denotes a crisp time interval, $J = \left[X, Y\right]$ denotes an ill-known time interval, with $\pi_{X}$ and $\pi_{Y}$ the possibility distributions of $X$ and $Y$ respectively. The second column contains the corresponding formula to calculate the possibility that $I$ satisfies all constraints given by the Allen's relation.}

\centering
\begin{tabular}{|c|c|}
\hline
Allen Relation &  $\Pos\left(\text{I satisfies all constraints }C_{i}, i = 1,2,...\right)$ \\
\hline
I before J & $\sup_{a>w}\pi_x(w)$\\
\hline

\multirow{2}{*}
{I equal J} &  $\min ( \sup_{a \leq w}\pi_x(w),\pi_x(w),$ \\
 &  $\sup_{b \geq w}\pi_Y(w),\pi_Y(w))$\\
\hline

I meets J  & $\min (\sup_{a\geq w} \pi_X(w),$ $\pi_X(w))$ \\
\hline

\multirow{2}{*}
{I overlaps J}  & $\min ( \sup_{b>w}\pi_Y(w), $ \\
 & $\sup_{a \geq w}\pi_X(w),\sup_{a \leq w}\pi_X(w))$ \\
\hline

\multirow{2}{*}
{I during J}  & $\max ( \min ( \sup_{a<w}\pi_X(w),\sup_{b \geq w}\pi_Y(w)),$ \\
 & $\min ( \sup_{a \leq w }\pi_X(w),\sup_{b>w}\pi_Y(w)$\\
\hline

{I starts J} &  $\min( \sup_{a \leq w}\pi_X(w),$ $\pi_X(w))$\\
\hline

{I finishes J} &  $\min ( \sup_{b \geq w} \pi_Y(w),$ $\pi_Y(w))$ \\
\hline 

\end{tabular}
%
%\vspace{10pt}
%
%
%\vspace{-25pt}
\label{tab:fuzzy-allen-relations}
\end{table}


One of the main purposes of a database is to allow the information retrieval. The standard query language for databases is SQL ~\cite{Mel93}, however, there are several proposals to extend the SQL language for transaction-time databases~\cite{Sarda90}, valid-time databases~\cite{gad92} and bi-temporal ~\cite{TSQL}. Some authors have studied how to support temporal querying in SQL~\cite{Snodgrass98}.

In the querying of a fuzzy temporal database it is possible to distinguish among the following cases:

\begin{itemize}
\item
Fuzzy data stored in the database and crisp specification in the query.
\item
Crisp data stored in the database and fuzzy data in the query specification.
\item
Both data stored in the database and the query specification are fuzzy.
\end{itemize}
In this subsection we will explain the query specification for a database that stores ill-known time intervals and crisp values in the query. Afterwards, the aggregation and ranking of the temporal results is explained.

\subsubsection{Query structure}
Consider that in a regular or fuzzy relational database, the query specification for the non-temporal attributes is given by $Q$. Therefore, the query specification in a temporal or fuzzy temporal database is given by $\tilde{Q}$:

\begin{definition}
\textbf{(query specification)}
A query $\tilde{Q}$ is given by:
\begin{equation}
\label{eq:query-specification}
\tilde{Q} = \left( Q^{time}, Q \right)
\end{equation}
Here $Q = \left \lbrace q_1, \cdots, q_n \right \rbrace$ are the possibly fuzzy non-temporal constraints and $Q^{time}$ are the temporal constraints specified in the query:
\end{definition}

%%specify better this:
%\begin{svgraybox}
%\begin{definition}
%\label{def:query-constraint}
%\textbf{(query constraint)}
%$Q$ is the 
%A query constraint $q_a$ for an attribute $a$ with attribute domain $\mathcal{D}$, is a restriction in the subset of the values in the attribute domain:
%\begin{equation}
%\label{eq:query-constraint}
%q_a = \left \lbrace d \mid d \subseteq \mathcal{D} \right \rbrace
%\end{equation}
%\end{definition}


\begin{definition}
\textbf{(temporal query specification)}
$Q^{time}$ is defined by:
\begin{equation}
Q^{time} = \left( I,AR \right)
\end{equation}
Where $I$ is a crisp time interval and $AR$ is one of the Allen's relations, see table \ref{tab:fuzzy-allen-relations}. The interpretation is that given an ill-known time interval $J$, the user request the set of records in which $I$ AR $J$ hold.
\end{definition}

\subsubsection{Query evaluation}
\label{subsubsec:query-evaluation}
Query satisfaction in a (fuzzy) relational database is a matter of degree. Typically, each constraint provided by the user is evaluated and results in a \emph{satisfaction degree} $s \in \left[ 0,1,\right]$. The value 0 denotes a total dissatisfaction while the value 1 denotes a total satisfaction.

\begin{definition}
\label{def:evaluation-function}
\textbf{(evaluation function)}
An evaluation function $e_{Q} \left( r \right)$ is a mapping from the attribute value from the row $r$ in the database with respect to the constraints $Q = \left \lbrace q_1, \cdots, q_n \right \rbrace$ to the unit interval $\left[ 0,1 \right]$. 
\begin{equation}
\label{eq:evaluation-function} 
e_{Q} = t \mid t \in \left[ 0,1 \right]
\end{equation}
\end{definition}

Each element in $\tilde{Q}$ is evaluated independently:
\begin{itemize}
\item
The non-temporal part, $Q$ is evaluated resulting in a satisfaction degree noted as $e_Q(r)$ for each record $r$.
\item
The temporal part, $Q^{time}$ is evaluated depending on $AR$. A specific set of ill-known constraints is considered depending on the Allen relation in $AR$. Then both possibility and necessity measures are computed and aggregated.
\end{itemize}

\subsubsection{Ranking and aggregation}
For each temporal attribute $r_{time}$ for the record $r$, both possibility $\Pos \left( Q_{time}\right)$ and necessity $\Nec \left( Q_{time} \right)$ measures are obtained. Then, the measures are combined in order to obtain an evaluation score $e_{Q_{time} \left( r \right)}$. The evaluation function is computed by:

\begin{equation}
\label{eq:temporal-eval}
e_{Q_{time} \left( r \right)} = \left \lbrace t \mid t = \left( \frac{\Pos \left( Q_{time}\right) + \Nec \left( Q_{time} \right)}{2} \right)   \in \left[ 0,1\right] \right \rbrace
\end{equation}

This measure provides a natural score because of the following property of the necessity measure: 
\begin{equation}
\label{eq:necessity} %iff or iif?
\Nec \left( Q_{time} \right) > 0  \Longleftrightarrow  \Pos \left( Q_{time}\right) = 1
\end{equation}

The final ranking is given by a convex combination of both temporal and non-temporal evaluations:

\begin{equation}
\label{eq:convex-combination}
e_{final} \left( r \right) = \left \lbrace t \mid t = \omega \ast e_{Q} \left( r \right) + \left( 1- \omega \right) \ast e_{Q^{time}}, t,\omega \in \left[0,1\right] \right \rbrace
\end{equation}




\subsection{Bipolarity in temporal databases}
\label{subsubsec:bipolarity}
Humans express their preferences in both positive and negative statements. This property is interesting in querying of databases, because sometimes, the user does not know exactly his / her preferences but what he / she does not want. \emph{Bipolarity} is therefore a technique to model both positive and negative preferences. Sometimes positive and negative information are clearly symmetric and therefore, it is possible to derive one information from another e.g. a person may define the concept of tall as 1.80 meters or higher. The negative would be the opposite, not tall is therefore, being less than 1.80 meters.  But in some cases, positive information can not be obtained directly from the negative information and vice-verse. E.g. a person wants to buy a blue motorbike, this does not mean that the person would reject a black motorbike. This heterogeneous semantics are given by the \emph{heterogeneous bipolarity}~\cite{Dubois2006},~\cite{Dubois2008}.\\

Some previous work have been done in ~\cite{Lacroix87}. There, instead of positive and negative criteria, desired and mandatory query conditions are explained. This can be seen as a way to specify positive and negative information: the inverse of a mandatory condition is what should be rejected (negative information), whereas desired conditions can be seen as positive information.\\


This introduces the necessity of a bipolar way of querying databases. Therefore, the bipolar querying of databases~\cite{DeTre2009} is a querying technique which allows the user to express both positive and negative criteria in the database query.\\

Bipolarity is handled by means of different tools:

\begin{itemize}
\item
The intuitionistic fuzzy sets  \emph{AFS} by Atannasov ~\cite{Atanassov1986}. The approach re-define the concept of inverse. Instead of considering the inverse of a set just the complementary, a AFS is defined by means of a membership function and a non-membership function.
\item
The interval valued fuzzy sets \emph{IVFS} were introduced independently by Zadeh~\cite{Zadeh75a}, Grattan-Guiness~\cite{Grattan76}, Janh~\cite{Jahn75}, Sambuc~\cite{Sambuc75} in the seventies~\cite{Dubois05}. %look for the reference!
\item
The two fold fuzzy sets \emph{(TFS)} by Dubois and Prade~\cite{Dubois02}. This approach defines desired and mandatory query conditions.
\end{itemize}

From a theoretical point of view, in a database, bipolarity may be found at:
\begin{description}
\item[\textbf{The query}] It is possible to distinguish between:
	\begin{itemize}
	\item
	Inside criterion bipolarity: Each individual criteria may be specified with both positive and negative conditions. E.g. the user wants a black car, but definitively not a red neither a blue one.
	\item
	Outside criterion bipolarity: The query is specified with a global positive and negative conditions. E.g. The user wants a black car with a fuel consumption of 5 liters or less but definitively not a white car with a fuel consumption of 6 liters or more. 
	\end{itemize}
\item[\textbf{The data storage}] In the database some attributes or even at tuple level it should be possible to specify them with bipolarity. Nevertheless, there is not so many research on this topic.
%	\begin{itemize}
%	\item
%	Attribute bipolarity: Each attribute may be specified  
%	\item
%	Tuple bipolarity:
%	\end{itemize}
\end{description}

A proposal for the bipolar querying of temporal databases has been made by Billiet~\cite{Billiet:Pons:Matthe:DeTre:Pons:2011:BipolarFuzzy}. First of all, the query structure is shown. Then, both evaluation and ranking methods are defined. The model deals with a fuzzy valid-time specification in~\cite{garrido2009}.

%\paragraph{Query structure}
%The approach in this work follows the query specification given in \eqref{eq:query-specification}. The query has a global time demand, $Q^{time}$ and here, the regular query constraints specified by $Q$ are splitted into possitive and negative preferences:
%
%\begin{equation}
%\label{eq:bipolar-specification}
%\tilde{Q} = \left \lbrace Q^{time}, \left( Q^{pos}, Q^{neg} \right) \right \rbrace
%\end{equation}
%
%Here $Q^{pos}$ and $Q^{neg}$ represent the positive and the negative criteria, respectively. The temporal demand in $Q^{time}$ may be specified as in ~\cite{garrido2009} or as in ~ \cite{Pon11}, see section \ref{subsec:representation-time-intervals}.

%\paragraph{Query evaluation}
%As explained previously in the query evaluation section (\ref{subsubsec:query-evaluation}), each element in $\tilde{Q}$ is evaluated independently:
%
%\begin{itemize}
%\item
%The query $Q$ has now two elements: $Q^{pos}$ and $Q^{neg}$. Each element is again evaluated independently and the result is a tuple $\left(s,d \right)$ in which $s$ is the satisfaction degree for $Q^{pos}$. $d$ is called dissatisfaction degree for $Q^{neg}$. The tuple $\left(s,d \right) s,d \in \left[0,1 \right]$ is called \emph{Bipolar Satisfaction Degree}.
%\item
%The temporal specification is evaluated as explained in \ref{subsubsec:query-evaluation} and the result is a value in the unit interval.
%\end{itemize}
%
%\paragraph{Ranking}
%In order to present the results to the user, it is necessary to design a combination function. This function allows the classification of the results in the unit interval the preffered way is to made a convex combination as  presented in equation \eqref{eq:convex-combination}.
%

