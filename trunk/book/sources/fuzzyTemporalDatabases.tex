%
% Fuzzy temporal databases
%
Consider a logistics company which transports packages. At the moment a package leaves, the time when it will arrive at its destination may be estimated, but will typically not be known precisely. For such companies and in many other situations, information systems able to handle imperfection with respect to certain temporal aspects of the objects modelled by the system are necessary.

\subsection{Handling Data Imperfections in Temporal Databases}

%The time when a package would arrive at its destination may be estimated and can eventually be researched, but it is typically not always precisely known. For such companies and in many other situations, information systems able to handle imperfection corresponding to certain temporal aspects of the objects modelled by the system are necessary. The transition between granularities~\cite{Lin97} is also sometimes considered as a source of imprecision. Therefore, some proposals consider granularity as the base of the temporal model~\cite{Cru97}.
%Some software applications deal with uncertainty related to time. e.g. a logistics company which transports packages from one destination to another. The time when the package would be at the end point may be estimated but not precisely known. These applications need to manage the uncertainty with respect to the temporal attributes of the objects stored in the database.

Data imperfections and techniques to represent them correctly in databases and queries are usually the focus of research in fuzzy databases. In fuzzy databases~\cite{Galindo2006}, uncertainty is usually expected to appear at storage level whereas imprecision is usually expected to appear in querying. Concerning temporal databases, there are several approaches to handle uncertainty in temporal data stored in a database.

Valid-time notions usually take the form of an interval. Some models consider probability~\cite{Dekhtyar2001} or possibility~\cite{Dubois89} distributions to describe the endpoints of a time interval. In this section, we will present a novel and currently researched technique to describe valid-time interval start- and endpoints using possibility distributions. In subsection \ref{subsec:representation-time-intervals} we will present an approach to modelling uncertainty in valid-time intervals. Finally, in subsection \ref{subsec:querying-time-intervals} the proposals for querying valid-time intervals in a flexible way are presented and discussed.%In this section we will explain the most interesting proposals that deal with possibility distributions in the valid-time interval. %Two are the main points developed within this section: the representation of uncertainty in the time interval and the flexible querying of ill-known time periods.

%In fuzzy databases~\cite{Galindo2006}, uncertainty is managed at storage level whereas imprecision is managed at querying level. In subsection \ref{subsec:representation-time-intervals} we will explain the main approach for the storage of uncertainty in the valid-time interval. Finally, in subsection \ref{subsec:querying-time-intervals} the proposals for querying in a flexible way valid-time intervals are analyzed.
\subsection{Handling Uncertainty in a Valid-time Relation}

\subsubsection{Ill-known Time Intervals}
\label{subsec:representation-time-intervals}
Uncertainty in a valid-time interval may be modelled by modelling the uncertainty concerning the exact start, respectively end point of the valid-time interval. The uncertainty concerning these points can by modelled by means of possibility distributions. It is of course also possible to use rough sets~\cite{Pawlak1995} for this modelling~\cite{Qia09}. The work presented in this section relies on possibilistic variables, which rely on possibility theory \cite{Dubois:Prade:1988:PossibilityTheory}.

A \emph{possibilistic variable} is defined as follows \cite{Pon11}.

\begin{svgraybox}
\vspace{-10pt}
\begin{definition}
A possibilistic variable $X$ over a universe $U$ is defined as a variable taking exactly one value in $U$, but for which this value is (partially) unknown. Its possibility distribution $\pi_X$ on $U$ models the available knowledge about the value that $X$ takes: for each $u\in U$, $\pi_X(u)$ represents the possibility that $X$ takes the value $u$. In this work, this possibility is interpreted as a measure of how plausible it is that $X$ takes the value $u$, given (partial) knowledge about the value $X$ takes.
\end{definition}
\vspace{-10pt}
\end{svgraybox}

The exact value a possibilistic variable takes, which is (partially) unknown, is called an \emph{ill-known value} in this work \cite{Dubois88b}.

When a possibilistic variable is defined on the powerset $\Pow(R)$ of some universe $R$, the unique value the variable takes will be a crisp set and its possibility distribution on the powerset $\Pow(R)$ will describe the possibility of each crisp subset of $R$ to be the value the variable takes. This exact value (a crisp set) the variable takes, is now called an \emph{ill-known set} \cite{Dubois88b}.

This work will deal with ill-known time intervals. These are ill-known intervals in time. Ill-known intervals are ill-known sets, defined and represented via a start and end point, which will be ill-known values. The elements of the set are the values between the start and end point. A closed ill-known interval with start point defined by possibilistic variable $X$ and end point by possibilistic variable $Y$ is noted here $\left[X, Y\right]$. The correspondences and transitions between the representations of ill-known sets, between the representations of ill-known intervals and between the representations of an ill-known set and an ill-known interval are part of the authors current research. Figure \ref{fig:interval} illustrates this.

%Uncertainty in a valid-time interval may be modelled with uncertainty at one or both starting and ending points. The uncertainty is usually represented by means of a possibility distribution in one or both points. It is also possible to represent with rough sets~\cite{Pawlak1995} time intervals~\cite{Qia09}.

Several authors ~\cite{garrido2009} propose transformations in order to optimize the storage of such ill-known valid-time intervals, though recent research might seem to indicate some minor issues with respect to some of these transformations ~\cite{Pon11}. A comparison between the transformations in ~\cite{garrido2009} and the framework in~\cite{Pon11} is presented in~\cite{pon12}. Figure \ref{fig:convexhull} illustrates a transformation based on the `convex hull' approach REF.



%An ill-known time interval is usually represented by means of two possibility distributions one in each starting and ending points. Figure \ref{fig:interval} shows an ill-known valid-time interval. The starting and ending points are denoted by two fuzzy numbers, $X$ and $Y$, represented by means of two triangular possibility distributions (see the Appendix). For illustration, figure \ref{fig:convexhull} shows one of the transformations based in the convex hull and proposed in~\cite{garrido2009}.

\begin{figure}
\centering
%%Created by jPicEdt 1.4.1_03: mixed JPIC-XML/LaTeX format
%%Wed Nov 23 16:38:02 CET 2011
%%Begin JPIC-XML
%<?xml version="1.0" standalone="yes"?>
%<jpic x-min="-2.5" x-max="60" y-min="-2.5" y-max="32.5" auto-bounding="true">
%<multicurve fill-style= "none"
%	 points= "(0,0);(0,0);(55,0);(55,0)"
%	 right-arrow= "head"
%	 />
%<multicurve fill-style= "none"
%	 points= "(0,0);(0,0);(0,30);(0,30)"
%	 right-arrow= "head"
%	 />
%<text fill-style= "none"
%	 text-vert-align= "center-v"
%	 anchor-point= "(-2.5,27.5)"
%	 text-frame= "noframe"
%	 text-hor-align= "center-h"
%	 right-arrow= "head"
%	 >
%1
%</text>
%<text fill-style= "none"
%	 text-vert-align= "center-v"
%	 anchor-point= "(-2.5,0)"
%	 text-frame= "noframe"
%	 text-hor-align= "center-h"
%	 right-arrow= "head"
%	 >
%0
%</text>
%<text fill-style= "none"
%	 text-vert-align= "center-v"
%	 anchor-point= "(15,32.5)"
%	 text-rotation= "135"
%	 text-frame= "noframe"
%	 text-hor-align= "center-h"
%	 right-arrow= "head"
%	 >
%Membership Degree
%</text>
%<text fill-style= "none"
%	 text-vert-align= "center-v"
%	 anchor-point= "(10,-2.5)"
%	 text-frame= "noframe"
%	 text-hor-align= "center-h"
%	 right-arrow= "head"
%	 >
%2
%</text>
%<text fill-style= "none"
%	 text-vert-align= "center-v"
%	 anchor-point= "(15,-2.5)"
%	 text-frame= "noframe"
%	 text-hor-align= "center-h"
%	 right-arrow= "head"
%	 >
%3
%</text>
%<text fill-style= "none"
%	 text-vert-align= "center-v"
%	 anchor-point= "(20,-2.5)"
%	 text-frame= "noframe"
%	 text-hor-align= "center-h"
%	 right-arrow= "head"
%	 >
%4
%</text>
%<text fill-style= "none"
%	 text-vert-align= "center-v"
%	 anchor-point= "(25,-2.5)"
%	 text-frame= "noframe"
%	 text-hor-align= "center-h"
%	 right-arrow= "head"
%	 >
%5
%</text>
%<text fill-style= "none"
%	 text-vert-align= "center-v"
%	 anchor-point= "(30,-2.5)"
%	 text-frame= "noframe"
%	 text-hor-align= "center-h"
%	 right-arrow= "head"
%	 >
%6
%</text>
%<text fill-style= "none"
%	 text-vert-align= "center-v"
%	 anchor-point= "(35,-2.5)"
%	 text-frame= "noframe"
%	 text-hor-align= "center-h"
%	 right-arrow= "head"
%	 >
%7
%</text>
%<text fill-style= "none"
%	 text-vert-align= "center-v"
%	 anchor-point= "(40,-2.5)"
%	 text-frame= "noframe"
%	 text-hor-align= "center-h"
%	 right-arrow= "head"
%	 >
%8
%</text>
%<text fill-style= "none"
%	 text-vert-align= "center-v"
%	 anchor-point= "(45,-2.5)"
%	 text-frame= "noframe"
%	 text-hor-align= "center-h"
%	 right-arrow= "head"
%	 >
%9
%</text>
%<text fill-style= "none"
%	 text-vert-align= "center-v"
%	 anchor-point= "(50,-2.5)"
%	 text-frame= "noframe"
%	 text-hor-align= "center-h"
%	 right-arrow= "head"
%	 >
%10
%</text>
%<text fill-style= "none"
%	 text-vert-align= "center-v"
%	 anchor-point= "(5,-2.5)"
%	 text-frame= "noframe"
%	 text-hor-align= "center-h"
%	 right-arrow= "head"
%	 >
%1
%</text>
%<multicurve fill-style= "none"
%	 points= "(5,0);(5,0);(15,27.5);(15,27.5)"
%	 />
%<multicurve fill-style= "none"
%	 points= "(15,27.5);(15,27.5);(20,0);(20,0)"
%	 />
%<multicurve fill-style= "none"
%	 points= "(25,0);(25,0);(35,27.5);(35,27.5)"
%	 />
%<multicurve fill-style= "none"
%	 points= "(35,27.5);(35,27.5);(50,0);(50,0)"
%	 />
%<text fill-style= "none"
%	 text-vert-align= "center-v"
%	 anchor-point= "(10,25)"
%	 text-frame= "noframe"
%	 text-hor-align= "center-h"
%	 >
%X
%</text>
%<text fill-style= "none"
%	 text-vert-align= "center-v"
%	 anchor-point= "(40,25)"
%	 text-frame= "noframe"
%	 text-hor-align= "center-h"
%	 >
%Y
%</text>
%<text fill-style= "none"
%	 text-vert-align= "center-v"
%	 anchor-point= "(60,0)"
%	 text-frame= "noframe"
%	 text-hor-align= "center-h"
%	 >
%Time
%</text>
%</jpic>
%%End JPIC-XML
%LaTeX-picture environment using emulated lines and arcs
%You can rescale the whole picture (to 80% for instance) by using the command \def\JPicScale{0.8}
\ifx\JPicScale\undefined\def\JPicScale{1}\fi
\unitlength \JPicScale mm
\begin{picture}(60,32.5)(0,0)
\linethickness{0.3mm}
\put(0,0){\line(1,0){55}}
\put(55,0){\vector(1,0){0.12}}
\linethickness{0.3mm}
\put(0,0){\line(0,1){30}}
\put(0,30){\vector(0,1){0.12}}
\put(-2.5,27.5){\makebox(0,0)[cc]{1}}

\put(-2.5,0){\makebox(0,0)[cc]{0}}

\put(15,32.5){\makebox(0,0)[cc]{Membership Degree}}

\put(10,-2.5){\makebox(0,0)[cc]{2}}

\put(15,-2.5){\makebox(0,0)[cc]{3}}

\put(20,-2.5){\makebox(0,0)[cc]{4}}

\put(25,-2.5){\makebox(0,0)[cc]{5}}

\put(30,-2.5){\makebox(0,0)[cc]{6}}

\put(35,-2.5){\makebox(0,0)[cc]{7}}

\put(40,-2.5){\makebox(0,0)[cc]{8}}

\put(45,-2.5){\makebox(0,0)[cc]{9}}

\put(50,-2.5){\makebox(0,0)[cc]{10}}

\put(5,-2.5){\makebox(0,0)[cc]{1}}

\linethickness{0.3mm}
\multiput(5,0)(0.12,0.33){83}{\line(0,1){0.33}}
\linethickness{0.3mm}
\multiput(15,27.5)(0.12,-0.65){42}{\line(0,-1){0.65}}
\linethickness{0.3mm}
\multiput(25,0)(0.12,0.33){83}{\line(0,1){0.33}}
\linethickness{0.3mm}
\multiput(35,27.5)(0.12,-0.22){125}{\line(0,-1){0.22}}
\put(10,25){\makebox(0,0)[cc]{X}}

\put(40,25){\makebox(0,0)[cc]{Y}}

\put(60,0){\makebox(0,0)[cc]{Time}}

\end{picture}

\caption{A closed ill-known time interval $\left[X, Y\right]$, where triangular possibility distributions describe the ill-known values defining the start and end points.}
\label{fig:interval}
\end{figure}

\begin{figure}
\centering
%%Created by jPicEdt 1.4.1_03: mixed JPIC-XML/LaTeX format
%%Mon Nov 28 16:07:12 CET 2011
%%Begin JPIC-XML
%<?xml version="1.0" standalone="yes"?>
%<jpic x-min="-2.5" x-max="60" y-min="-2.5" y-max="72.5" auto-bounding="true">
%<multicurve fill-style= "none"
%	 points= "(0,40);(0,40);(55,40);(55,40)"
%	 right-arrow= "head"
%	 />
%<multicurve fill-style= "none"
%	 points= "(0,40);(0,40);(0,70);(0,70)"
%	 right-arrow= "head"
%	 />
%<text text-vert-align= "center-v"
%	 anchor-point= "(-2.5,67.5)"
%	 fill-style= "none"
%	 text-frame= "noframe"
%	 text-hor-align= "center-h"
%	 right-arrow= "head"
%	 >
%1
%</text>
%<text text-vert-align= "center-v"
%	 anchor-point= "(-2.5,40)"
%	 fill-style= "none"
%	 text-frame= "noframe"
%	 text-hor-align= "center-h"
%	 right-arrow= "head"
%	 >
%0
%</text>
%<text text-vert-align= "center-v"
%	 anchor-point= "(15,72.5)"
%	 fill-style= "none"
%	 text-rotation= "135"
%	 text-frame= "noframe"
%	 text-hor-align= "center-h"
%	 right-arrow= "head"
%	 >
%Membership Degree
%</text>
%<text text-vert-align= "center-v"
%	 anchor-point= "(10,37.5)"
%	 fill-style= "none"
%	 text-frame= "noframe"
%	 text-hor-align= "center-h"
%	 right-arrow= "head"
%	 >
%2
%</text>
%<text text-vert-align= "center-v"
%	 anchor-point= "(15,37.5)"
%	 fill-style= "none"
%	 text-frame= "noframe"
%	 text-hor-align= "center-h"
%	 right-arrow= "head"
%	 >
%3
%</text>
%<text text-vert-align= "center-v"
%	 anchor-point= "(20,37.5)"
%	 fill-style= "none"
%	 text-frame= "noframe"
%	 text-hor-align= "center-h"
%	 right-arrow= "head"
%	 >
%4
%</text>
%<text text-vert-align= "center-v"
%	 anchor-point= "(25,37.5)"
%	 fill-style= "none"
%	 text-frame= "noframe"
%	 text-hor-align= "center-h"
%	 right-arrow= "head"
%	 >
%5
%</text>
%<text text-vert-align= "center-v"
%	 anchor-point= "(30,37.5)"
%	 fill-style= "none"
%	 text-frame= "noframe"
%	 text-hor-align= "center-h"
%	 right-arrow= "head"
%	 >
%6
%</text>
%<text text-vert-align= "center-v"
%	 anchor-point= "(35,37.5)"
%	 fill-style= "none"
%	 text-frame= "noframe"
%	 text-hor-align= "center-h"
%	 right-arrow= "head"
%	 >
%7
%</text>
%<text text-vert-align= "center-v"
%	 anchor-point= "(40,37.5)"
%	 fill-style= "none"
%	 text-frame= "noframe"
%	 text-hor-align= "center-h"
%	 right-arrow= "head"
%	 >
%8
%</text>
%<text text-vert-align= "center-v"
%	 anchor-point= "(45,37.5)"
%	 fill-style= "none"
%	 text-frame= "noframe"
%	 text-hor-align= "center-h"
%	 right-arrow= "head"
%	 >
%9
%</text>
%<text text-vert-align= "center-v"
%	 anchor-point= "(50,37.5)"
%	 fill-style= "none"
%	 text-frame= "noframe"
%	 text-hor-align= "center-h"
%	 right-arrow= "head"
%	 >
%10
%</text>
%<text text-vert-align= "center-v"
%	 anchor-point= "(5,37.5)"
%	 fill-style= "none"
%	 text-frame= "noframe"
%	 text-hor-align= "center-h"
%	 right-arrow= "head"
%	 >
%1
%</text>
%<multicurve fill-style= "none"
%	 points= "(5,40);(5,40);(15,67.5);(15,67.5)"
%	 />
%<multicurve fill-style= "none"
%	 points= "(15,67.5);(15,67.5);(20,40);(20,40)"
%	 />
%<multicurve fill-style= "none"
%	 points= "(25,40);(25,40);(35,67.5);(35,67.5)"
%	 />
%<multicurve fill-style= "none"
%	 points= "(35,67.5);(35,67.5);(50,40);(50,40)"
%	 />
%<text text-vert-align= "center-v"
%	 anchor-point= "(10,65)"
%	 fill-style= "none"
%	 text-frame= "noframe"
%	 text-hor-align= "center-h"
%	 >
%X
%</text>
%<text text-vert-align= "center-v"
%	 anchor-point= "(40,65)"
%	 fill-style= "none"
%	 text-frame= "noframe"
%	 text-hor-align= "center-h"
%	 >
%Y
%</text>
%<text text-vert-align= "center-v"
%	 anchor-point= "(60,40)"
%	 fill-style= "none"
%	 text-frame= "noframe"
%	 text-hor-align= "center-h"
%	 >
%Time
%</text>
%<multicurve fill-style= "none"
%	 points= "(0,0);(0,0);(55,0);(55,0)"
%	 right-arrow= "head"
%	 />
%<multicurve fill-style= "none"
%	 points= "(0,0);(0,0);(0,30);(0,30)"
%	 right-arrow= "head"
%	 />
%<text text-vert-align= "center-v"
%	 anchor-point= "(-2.5,27.5)"
%	 fill-style= "none"
%	 text-frame= "noframe"
%	 text-hor-align= "center-h"
%	 right-arrow= "head"
%	 >
%1
%</text>
%<text text-vert-align= "center-v"
%	 anchor-point= "(-2.5,0)"
%	 fill-style= "none"
%	 text-frame= "noframe"
%	 text-hor-align= "center-h"
%	 right-arrow= "head"
%	 >
%0
%</text>
%<text text-vert-align= "center-v"
%	 anchor-point= "(15,32.5)"
%	 fill-style= "none"
%	 text-rotation= "135"
%	 text-frame= "noframe"
%	 text-hor-align= "center-h"
%	 right-arrow= "head"
%	 >
%Membership Degree
%</text>
%<text text-vert-align= "center-v"
%	 anchor-point= "(10,-2.5)"
%	 fill-style= "none"
%	 text-frame= "noframe"
%	 text-hor-align= "center-h"
%	 right-arrow= "head"
%	 >
%2
%</text>
%<text text-vert-align= "center-v"
%	 anchor-point= "(15,-2.5)"
%	 fill-style= "none"
%	 text-frame= "noframe"
%	 text-hor-align= "center-h"
%	 right-arrow= "head"
%	 >
%3
%</text>
%<text text-vert-align= "center-v"
%	 anchor-point= "(20,-2.5)"
%	 fill-style= "none"
%	 text-frame= "noframe"
%	 text-hor-align= "center-h"
%	 right-arrow= "head"
%	 >
%4
%</text>
%<text text-vert-align= "center-v"
%	 anchor-point= "(25,-2.5)"
%	 fill-style= "none"
%	 text-frame= "noframe"
%	 text-hor-align= "center-h"
%	 right-arrow= "head"
%	 >
%5
%</text>
%<text text-vert-align= "center-v"
%	 anchor-point= "(30,-2.5)"
%	 fill-style= "none"
%	 text-frame= "noframe"
%	 text-hor-align= "center-h"
%	 right-arrow= "head"
%	 >
%6
%</text>
%<text text-vert-align= "center-v"
%	 anchor-point= "(35,-2.5)"
%	 fill-style= "none"
%	 text-frame= "noframe"
%	 text-hor-align= "center-h"
%	 right-arrow= "head"
%	 >
%7
%</text>
%<text text-vert-align= "center-v"
%	 anchor-point= "(40,-2.5)"
%	 fill-style= "none"
%	 text-frame= "noframe"
%	 text-hor-align= "center-h"
%	 right-arrow= "head"
%	 >
%8
%</text>
%<text text-vert-align= "center-v"
%	 anchor-point= "(45,-2.5)"
%	 fill-style= "none"
%	 text-frame= "noframe"
%	 text-hor-align= "center-h"
%	 right-arrow= "head"
%	 >
%9
%</text>
%<text text-vert-align= "center-v"
%	 anchor-point= "(50,-2.5)"
%	 fill-style= "none"
%	 text-frame= "noframe"
%	 text-hor-align= "center-h"
%	 right-arrow= "head"
%	 >
%10
%</text>
%<text text-vert-align= "center-v"
%	 anchor-point= "(5,-2.5)"
%	 fill-style= "none"
%	 text-frame= "noframe"
%	 text-hor-align= "center-h"
%	 right-arrow= "head"
%	 >
%1
%</text>
%<multicurve fill-style= "none"
%	 points= "(5,0);(5,0);(15,27.5);(15,27.5)"
%	 />
%<multicurve fill-style= "none"
%	 points= "(15,27.5);(15,27.5);(35,27.5);(35,27.5)"
%	 />
%<multicurve fill-style= "none"
%	 points= "(35,27.5);(35,27.5);(50,0);(50,0)"
%	 />
%<text text-vert-align= "center-v"
%	 anchor-point= "(60,0)"
%	 fill-style= "none"
%	 text-frame= "noframe"
%	 text-hor-align= "center-h"
%	 >
%Time
%</text>
%<text text-vert-align= "center-v"
%	 stroke-dasharray= "1;1"
%	 anchor-point= "(40,25)"
%	 stroke-style= "dashed"
%	 fill-style= "none"
%	 text-frame= "noframe"
%	 text-hor-align= "center-h"
%	 >
%T
%</text>
%</jpic>
%%End JPIC-XML
%LaTeX-picture environment using emulated lines and arcs
%You can rescale the whole picture (to 80% for instance) by using the command \def\JPicScale{0.8}
\ifx\JPicScale\undefined\def\JPicScale{1}\fi
\unitlength \JPicScale mm
\begin{picture}(60,72.5)(0,0)
\linethickness{0.3mm}
\put(0,40){\line(1,0){55}}
\put(55,40){\vector(1,0){0.12}}
\linethickness{0.3mm}
\put(0,40){\line(0,1){30}}
\put(0,70){\vector(0,1){0.12}}
\put(-2.5,67.5){\makebox(0,0)[cc]{1}}

\put(-2.5,40){\makebox(0,0)[cc]{0}}

\put(15,72.5){\makebox(0,0)[cc]{Membership Degree}}

\put(10,37.5){\makebox(0,0)[cc]{2}}

\put(15,37.5){\makebox(0,0)[cc]{3}}

\put(20,37.5){\makebox(0,0)[cc]{4}}

\put(25,37.5){\makebox(0,0)[cc]{5}}

\put(30,37.5){\makebox(0,0)[cc]{6}}

\put(35,37.5){\makebox(0,0)[cc]{7}}

\put(40,37.5){\makebox(0,0)[cc]{8}}

\put(45,37.5){\makebox(0,0)[cc]{9}}

\put(50,37.5){\makebox(0,0)[cc]{10}}

\put(5,37.5){\makebox(0,0)[cc]{1}}

\linethickness{0.3mm}
\multiput(5,40)(0.12,0.33){83}{\line(0,1){0.33}}
\linethickness{0.3mm}
\multiput(15,67.5)(0.12,-0.65){42}{\line(0,-1){0.65}}
\linethickness{0.3mm}
\multiput(25,40)(0.12,0.33){83}{\line(0,1){0.33}}
\linethickness{0.3mm}
\multiput(35,67.5)(0.12,-0.22){125}{\line(0,-1){0.22}}
\put(10,65){\makebox(0,0)[cc]{X}}

\put(40,65){\makebox(0,0)[cc]{Y}}

\put(60,40){\makebox(0,0)[cc]{Time}}

\linethickness{0.3mm}
\put(0,0){\line(1,0){55}}
\put(55,0){\vector(1,0){0.12}}
\linethickness{0.3mm}
\put(0,0){\line(0,1){30}}
\put(0,30){\vector(0,1){0.12}}
\put(-2.5,27.5){\makebox(0,0)[cc]{1}}

\put(-2.5,0){\makebox(0,0)[cc]{0}}

\put(15,32.5){\makebox(0,0)[cc]{Membership Degree}}

\put(10,-2.5){\makebox(0,0)[cc]{2}}

\put(15,-2.5){\makebox(0,0)[cc]{3}}

\put(20,-2.5){\makebox(0,0)[cc]{4}}

\put(25,-2.5){\makebox(0,0)[cc]{5}}

\put(30,-2.5){\makebox(0,0)[cc]{6}}

\put(35,-2.5){\makebox(0,0)[cc]{7}}

\put(40,-2.5){\makebox(0,0)[cc]{8}}

\put(45,-2.5){\makebox(0,0)[cc]{9}}

\put(50,-2.5){\makebox(0,0)[cc]{10}}

\put(5,-2.5){\makebox(0,0)[cc]{1}}

\linethickness{0.3mm}
\multiput(5,0)(0.12,0.33){83}{\line(0,1){0.33}}
\linethickness{0.3mm}
\put(15,27.5){\line(1,0){20}}
\linethickness{0.3mm}
\multiput(35,27.5)(0.12,-0.22){125}{\line(0,-1){0.22}}
\put(60,0){\makebox(0,0)[cc]{Time}}

\put(40,25){\makebox(0,0)[cc]{T}}

\end{picture}

\caption{Transformation based in the convex hull from the two ill-known points $X$ and $Y$. }
\label{fig:convexhull}
\end{figure}

In section \ref{subsec:querying-time-intervals}, a technique allowing the querying of fuzzy valid-time databases which contain ill-known valid-time intervals is presented and briefly explained. The technique relies on the framework for set evaluation presented in \cite{Pon11}.



\subsubsection{Querying Fuzzy Valid-Time Databases with Ill-Known Valid-Time Intervals}
\label{subsec:querying-time-intervals}

One of the main purposes of the existence of databases is to allow information retrieval. The standard query language for databases is SQL ~\cite{Mel93}. Several proposals to extend the SQL language for transaction-time databases~\cite{Sarda90}, valid-time databases~\cite{gad92} and bitemporal databases ~\cite{TSQL} exist and some authors have studied how to support temporal querying in standard SQL~\cite{Snodgrass98}.

In the following subsections, the query structure, the query evaluation method and the technique to meaningfully rank and aggregate the query results, proposed in this work, are presented. 

%In the querying of a fuzzy temporal database it is possible to distinguish among the following cases:

%\begin{itemize}
%\item
%Fuzzy data stored in the database and crisp specification in the query.
%\item
%Crisp data stored in the database and fuzzy data in the query specification.
%\item
%Both data stored in the database and the query specification are fuzzy.
%\end{itemize}
%In this subsection we will explain the query specification for a database that stores ill-known time intervals and crisp values in the query. Afterwards, the aggregation and ranking of the temporal results is explained.




\subsubsection{Query Structure} TO BE ANOTHER PART
In this work, a query $\tilde{Q}$ is made up of two separate constructs of user query demands.

%Consider that in a regular or fuzzy relational database, the query specification for the non-temporal attributes is given by $Q$. Therefore, the query specification in a temporal or fuzzy temporal database is given by $\tilde{Q}$:

\begin{svgraybox}
\vspace{-10pt}
\begin{definition}
\textbf{Query}
A query $\tilde{Q}$ in this work is given by:
\begin{equation}
\label{eq:query-specification}
\tilde{Q} = \left( Q^{time}, Q \right)
\end{equation}
Here $Q = \left \lbrace q_1,\ \cdots,\ q_n \right \rbrace$ contains all (possibly fuzzy) non-temporal user-defined query demands. These comprise all constraints unrelated to the valid-time indications in the queried relations. $Q^{time}$ denotes the temporal constraint specified by the user.
\end{definition}
\vspace{-10pt}
\end{svgraybox}
%%specify better this:
%\begin{svgraybox}
%\begin{definition}
%\label{def:query-constraint}
%\textbf{(query constraint)}
%$Q$ is the 
%A query constraint $q_a$ for an attribute $a$ with attribute domain $\mathcal{D}$, is a restriction in the subset of the values in the attribute domain:
%\begin{equation}
%\label{eq:query-constraint}
%q_a = \left \lbrace d \mid d \subseteq \mathcal{D} \right \rbrace
%\end{equation}
%\end{definition}

In this work, the query structure allows the user to specify a single temporal constraint.

\begin{svgraybox}
\vspace{-10pt}
\begin{definition}
\textbf{Temporal Constraint $Q^{time}$}
In this work, $Q^{time}$ is defined by:
\begin{equation}
Q^{time} = \left( I,AR \right)
\end{equation}
where $I$ denotes a crisp time interval and $AR$ is one of the Allen relations.
\end{definition}
\vspace{-10pt}
\end{svgraybox}

The interpretation is that, for a record with an ill-known valid-time interval $J$, the user demands that $I$ AR $J$ hold.

\subsubsection{Query evaluation} TO BE ANOTHER PART
\label{subsubsec:query-evaluation}
Query satisfaction in a fuzzy relational database is a matter of degree. Typically, the query evaluation results in a \emph{satisfaction degree} $s$, which is typically in the unit interval. $s \in \left[ 0,1,\right]$. Here, a satisfaction degree of $0$ denotes a total dissatisfaction while a degree of $1$ denotes a total satisfaction.

%\begin{svgraybox}
%\begin{definition}
%\label{def:evaluation-function}
%\textbf{Evaluation function}
%An evaluation function $e_{Q} \left( r \right)$ is a mapping from the attribute value from the row $r$ in the database with respect to the constraints $Q = \left \lbrace q_1, \cdots, q_n \right \rbrace$ to the unit interval $\left[ 0,1 \right]$. 
%\begin{equation}
%\label{eq:evaluation-function} 
%e_{Q} = t \mid t \in \left[ 0,1 \right]
%\end{equation}
%\end{definition}
%\end{svgraybox}


In the presented approach, for every record $r$, each part of $\tilde{Q}$ is evaluated independently:
\begin{itemize}
\item
The user preferences expressed in the non-temporal part, $Q$, are evaluated, resulting in a satisfaction degree denoted as $e_Q(r)$. The presented approach accepts any valid, sound method of calculating this evaluation, as long as the method is well-founded and $e_Q(r) \in \left[0,1\right]$.
\item
The evaluation of the temporal demand expressed in the temporal part, $Q^{time}$, depends on $AR$. A specific set of ill-known constraints\cite{Pon11} is considered depending on the Allen relation denoted by $AR$. The exact set chosen can be found in table \ref{tab:fuzzy-allen-relations}, for every possible value of $AR$. Then, using these formulas given in table \ref{tab:fuzzy-allen-relations}, both the possibility degree $\Pos_{Q^{time}}(r)$ and the necessity degree $\Nec_{Q^{time}}(r)$ are computed.
\end{itemize}

\textcolor{red}{TODO: maybe part on what these and mean? Maybe part on where the formulas in table \ref{tab:fuzzy-allen-relations} come from?}

\begin{table}[h]

\caption{Allen's relations used in the framework. Here, $I = \left[a, b\right]$ denotes a crisp time interval, $J = \left[X, Y\right]$ denotes an ill-known time interval, with $\pi_{X}$ and $\pi_{Y}$ the possibility distributions of $X$ and $Y$ respectively. The second column contains the corresponding formula to calculate the possibility that $I$ satisfies all constraints given by the Allen's relation.}

\centering
\begin{tabular}{|c|c|c|}
\hline
Allen Relation &  $\Pos\left(\text{I satisfies all constraints }C_{i}, i = 1,2,...\right)$ & $\Nec\left(C_{i}, i = 1,2,... \right)$ \\
\hline
I before J & $\sup_{a>w}\pi_X(w)$ & $\inf_{a \leq w} 1- \pi_X(w)$\\
\hline

\multirow{4}{*}
{I equal J} &  $\min ( \sup_{a \leq w}\pi_X(w),$  & $\min ( \inf_{a>w} 1-\pi_X(w),$\\
& $\pi_X(w),$ & $\inf_{w \in \left[a,b\right]} \pi_X(w), $ \\
 &  $\sup_{b \geq w}\pi_Y(w),$ & $\inf_{b>w} 1-\pi_Y(w),$\\
 & $\pi_Y(w))$ & $\inf_{w \in \left[a,b\right]} \pi_Y(w) )$ \\
\hline

\multirow{2}{*}
I meets J  & $\min (\sup_{a\geq w} \pi_X(w),$ & $\min (\inf_{a<w} 1 - \pi_X(w),$\\
& $\pi_X(w))$  & $ \inf_{w \in \left[a,b\right]} \pi_X(w) )$ \\
\hline

\multirow{3}{*}
{I overlaps J}  & $\min ( \sup_{b>w}\pi_Y(w), $ & $\min ( \inf_{b \geq w} 1 - \pi_Y(w),$\\
 & $\sup_{a \geq w}\pi_X(w),$& $1- \inf_{a>w}1-\pi_X(w),$\\
 & $\sup_{a \leq w}\pi_X(w))$ & $1- \inf_{a<w}1-\pi_X(w) )$ \\
\hline

\multirow{4}{*}
{I during J}  & $\max ( \min ( \sup_{a<w}\pi_X(w),,$ & $\max (\min (\inf_{a \leq w}1-\pi_X(w), $\\
 & $\sup_{b \geq w}\pi_Y(w)) $ & $ \inf_{b>w} 1- \pi_Y(w)),$ \\
 & $\min ( \sup_{a \leq w }\pi_X(w),$ & $\min ( \inf_{a<w}1-\pi_X(w),$\\
 & $\sup_{b>w}\pi_Y(w))$ & $\inf_{b\geq w}1-\pi_Y(w)))$ \\
\hline
\multirow{2}{*}
{I starts J} &  $\min( \sup_{a \leq w}\pi_X(w),$  & $\min (\inf_{a<w}1-\pi_X(w),$\\
& $\pi_X(w))$ & $inf_{w \in \left[a,b\right]} 1- \pi_X(w))$ \\
\hline
\multirow{2}{*}
{I finishes J} &  $\min ( \sup_{b \geq w} \pi_Y(w),$  & $\min(\inf_{b>w}1-\pi_Y(w),$\\
& $\pi_Y(w))$ & $\inf_{w \in [a,b]} 1- \pi_Y(w))$ \\
\hline 

\end{tabular}
%
%\vspace{10pt}
%
%
%\vspace{-25pt}
\label{tab:fuzzy-allen-relations}
\end{table}

\subsubsection{Aggregation and Ranking} TO BE ANOTHER PART
To be able to present the most appropriate results to the user, for every record $r$, an aggregation method is used to aggregate $\Pos_{Q^{time}}(r)$ and $\Nec_{Q^{time}}(r)$ into a temporal record rank $e_{Q^{time}}(r)$ and after this, a convex combination combining $e_{Q^{time}}(r)$ and $e_{Q}(r)$ will provide the final record rank $e_{final}(r)$.

%For each temporal attribute $r_{time}$ for the record $r$, both possibility $\Pos \left( Q_{time}\right)$ and necessity $\Nec \left( Q_{time} \right)$ measures are obtained. Then, the measures are combined in order to obtain an evaluation score $e_{Q_{time} \left( r \right)}$. The evaluation function is computed by:

To calculate $e_{Q^{time}}(r)$, an a simple and crude method is used:

\begin{equation}
\label{eq:temporal-eval}
e_{Q^{time}}(r) = \left(\frac{\Pos_{Q^{time}}(r)+\Nec_{Q^{time}}(r)}{2}\right)
%e_{Q_{time} \left( r \right)} = \left \lbrace t \mid t = \left( \frac{\Pos \left( Q_{time}\right) + \Nec \left( Q_{time} \right)}{2} \right)   \in \left[ 0,1\right] \right \rbrace
\end{equation}

This method aims to provide the result records with a natural ranking based on the users temporal constraint. $e_{Q^{time}}(r)$ will of course be a value in $\left[0,1\right]$, as both $\Pos_{Q^{time}}(r) \in \left[0,1\right]$ and $\Nec_{Q^{time}}(r) \in \left[0,1\right]$. The purpose is that records which fit the users temporal demand better get a higher score than records fitting the temporal demand worse. Here, this aim is reached because the necessity degree $\Nec_{Q^{time}}(r)$ cannot exceed $0$ unless the possibility degree $\Pos_{Q^{time}}(r)$ equals $1$.

%This measure provides a natural score because of the following property of the necessity measure: 
%\begin{equation}
%\label{eq:necessity} %iff or iif?
%\Nec \left( Q_{time} \right) > 0  \Longleftrightarrow  \Pos \left( Q_{time}\right) = 1
%\end{equation}

The final ranking $e_{final}(r)$ for a record $r$ is now given by a convex combination of both temporal and non-temporal evaluation scores.

\begin{equation}
\label{eq:convex-combination}
e_{final} \left( r \right) = \omega \ast e_{Q} \left( r \right) + \left( 1- \omega \right) \ast e_{Q^{time}}
\end{equation}

A convex combination is used mainly for 2 reasons:
\begin{itemize}
	\item The use of this convex combination allows a record to make up for a low temporal evaluation score with a high non-temporal evaluation score and vice versa.
	\item The exact value of $\omega$ can now be modified to ascribe more value to either the fulfillment of the user's temporal demands or the fulfillment of the user's non-temporal constraints.
\end{itemize}

\subsection{Bipolarity in Temporal Databases}
\label{subsubsec:bipolarity}
Humans express their preferences using both positive and negative statements, where positive statements express what is desired or acceptable and negative statements express what is undesired or unacceptable \cite{Billiet:Pons:Matthe:DeTre:Pons:2011:BipolarFuzzy}. This realization is interesting with regard to database querying, because sometimes a user does not exactly know his or her preferences or can't express them in only positive statements, but prefers to use negative statements to express what he or she dislikes or doesn't need. This introduces the need for \emph{bipolar querying}, a technique to model both positive and negative user preferences in a database query. Sometimes positive and negative preferences are clearly symmetric, making it possible to derive one from the other. For example, a person may define the concept of `tall' as `1.80 meters or higher'. The negative would then be the opposite: not tall would be `anyhing less than 1.80 meters'. However, in some cases, positive preferences can not be directly obtained from negative preferences or vice versa. E.g. when a person prefers to buy a black motorbike, this does not necessarily mean the person would totally reject a very dark blue motorbike. This phenomenon is called \emph{heterogeneous bipolarity}~\cite{Dubois2006},~\cite{Dubois2008}.

The use of imprecise query preference formulation in bipolar querying is well discussed in existing literature \cite{DeTre2009}, \cite{Dubois2008}, \cite{Lacroix87}. In \cite{Lacroix87}, desired and mandatory query conditions are used, instead of positive and negative preferences. However, the inverse of a mandatory preference expresses what should be rejected and this could be seen as negative information, whereas desired query conditions can be seen as positive preferences. However, the combination of bipolar querying and the use of imprecise query preferences in the context of temporal databases is not so well discussed in existing literature. A proposal for the bipolar querying of valid-time databases has been made by Billiet et al.~\cite{Billiet:Pons:Matthe:DeTre:Pons:2011:BipolarFuzzy}. The model presented there deals with a fuzzy valid-time specification based on~\cite{garrido2009}.

%Some previous work have been done in ~\cite{Lacroix87}. There, instead of positive and negative criteria, desired and mandatory query conditions are explained. This can be seen as a way to specify positive and negative information: the inverse of a mandatory condition is what should be rejected (negative information), whereas desired conditions can be seen as positive information.\\


%This introduces the necessity of a bipolar way of querying databases. Therefore, the bipolar querying of databases~\cite{DeTre2009} is a querying technique which allows the user to express both positive and negative criteria in the database query.\\

Bipolarity can be handled using different concepts, such as intuitionistic fuzzy sets~\cite{Atanassov1986}, interval valued fuzzy sets~\cite{Zadeh75a} Grattan-Guiness~\cite{Grattan76}, Janh~\cite{Jahn75}, Sambuc~\cite{Sambuc75},~\cite{Dubois05} or two fold fuzzy sets \cite{Dubois02}.

%\begin{itemize}
%\item
%Intuitionistic Fuzzy Sets~\cite{Atanassov1986} are sometimes used. The approach re-define the concept of inverse. Instead of considering the inverse of a set just the complementary, a AFS is defined by means of a membership function and a non-membership function.
%\item
%The interval valued fuzzy sets \emph{IVFS} were introduced independently by Zadeh~\cite{Zadeh75a}, Grattan-Guiness~\cite{Grattan76}, Janh~\cite{Jahn75}, Sambuc~\cite{Sambuc75} in the seventies~\cite{Dubois05}. %look for the reference!
%\item
%The two fold fuzzy sets \emph{(TFS)} by Dubois and Prade~\cite{Dubois02}. This approach defines desired and mandatory query conditions.
%\end{itemize}

From a theoretical point of view, bipolarity might be found either in the queries presented to a database system or in a database managed by a database system.

When bipolarity is found in queries, it is possible to distinguish between:
\begin{itemize}
	\item
	Bipolarity inside query criteria: each individual query criterion may be specified using both positive and negative preferences. For example when querying a car database, the user can express that he or she wants a black car, but definitely not a red neither a blue one. Bipolarity resides here within the car color criterion.
	\item
	Bipolarity outside query criteria: the query is specified using a global positive and a global negative preference part. For example when querying a car database, the user can express that he or she wants a black car, but definitely not a car with a fuel consumption of 6 liters or more.
	\end{itemize}

Concerning bipolarity inside a database, it should be possible to specify both positive and negative real world object or concept aspects, even at record level. Nevertheless, not so much research exists concerning bipolarity in databases.

%some attributes or even at tuple level it should be possible to specify them with bipolarity. Nevertheless, there is not so many research on this topic.
%	\begin{itemize}
%	\item
%	Attribute bipolarity: Each attribute may be specified  
%	\item
%	Tuple bipolarity:
%	\end{itemize}




%\paragraph{Query structure}
%The approach in this work follows the query specification given in \eqref{eq:query-specification}. The query has a global time demand, $Q^{time}$ and here, the regular query constraints specified by $Q$ are splitted into possitive and negative preferences:
%
%\begin{equation}
%\label{eq:bipolar-specification}
%\tilde{Q} = \left \lbrace Q^{time}, \left( Q^{pos}, Q^{neg} \right) \right \rbrace
%\end{equation}
%
%Here $Q^{pos}$ and $Q^{neg}$ represent the positive and the negative criteria, respectively. The temporal demand in $Q^{time}$ may be specified as in ~\cite{garrido2009} or as in ~ \cite{Pon11}, see section \ref{subsec:representation-time-intervals}.

%\paragraph{Query evaluation}
%As explained previously in the query evaluation section (\ref{subsubsec:query-evaluation}), each element in $\tilde{Q}$ is evaluated independently:
%
%\begin{itemize}
%\item
%The query $Q$ has now two elements: $Q^{pos}$ and $Q^{neg}$. Each element is again evaluated independently and the result is a tuple $\left(s,d \right)$ in which $s$ is the satisfaction degree for $Q^{pos}$. $d$ is called dissatisfaction degree for $Q^{neg}$. The tuple $\left(s,d \right) s,d \in \left[0,1 \right]$ is called \emph{Bipolar Satisfaction Degree}.
%\item
%The temporal specification is evaluated as explained in \ref{subsubsec:query-evaluation} and the result is a value in the unit interval.
%\end{itemize}
%
%\paragraph{Ranking}
%In order to present the results to the user, it is necessary to design a combination function. This function allows the classification of the results in the unit interval the preffered way is to made a convex combination as  presented in equation \eqref{eq:convex-combination}.
%

