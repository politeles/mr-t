%%Created by jPicEdt 1.4.1_03: mixed JPIC-XML/LaTeX format
%%Thu Jun 30 10:32:52 CEST 2011
%%Begin JPIC-XML
%<?xml version="1.0" standalone="yes"?>
%<jpic x-min="5" x-max="125" y-min="5" y-max="65" auto-bounding="true">
%<multicurve fill-style= "none"
%	 points= "(10,10);(10,10);(10,60);(10,60)"
%	 />
%<multicurve fill-style= "none"
%	 points= "(10,10);(10,10);(110,10);(110,10)"
%	 />
%<multicurve fill-style= "none"
%	 points= "(30,10);(30,10);(60,60);(60,60)"
%	 />
%<multicurve fill-style= "none"
%	 stroke-color= "#ccccff"
%	 points= "(60,60);(60,60);(90,10);(90,10)"
%	 />
%<multicurve stroke-width= "0.9"
%	 fill-style= "none"
%	 stroke-color= "#ccccff"
%	 points= "(80,10);(80,10);(100,60);(100,60)"
%	 />
%<multicurve stroke-width= "0.9"
%	 fill-style= "none"
%	 stroke-color= "#ff00cc"
%	 points= "(100,60);(100,60);(110,10);(110,10)"
%	 />
%<text stroke-width= "0.95"
%	 text-vert-align= "center-v"
%	 anchor-point= "(60,65)"
%	 fill-style= "none"
%	 stroke-color= "#ff0033"
%	 text-frame= "noframe"
%	 text-hor-align= "center-h"
%	 >
%X
%</text>
%<text stroke-width= "0.95"
%	 text-vert-align= "center-v"
%	 anchor-point= "(100,65)"
%	 fill-style= "none"
%	 stroke-color= "#ff0033"
%	 text-frame= "noframe"
%	 text-hor-align= "center-h"
%	 >
%Y
%</text>
%<text stroke-width= "0.95"
%	 text-vert-align= "center-v"
%	 anchor-point= "(125,40)"
%	 fill-style= "none"
%	 stroke-color= "#ff0033"
%	 text-frame= "noframe"
%	 text-hor-align= "center-h"
%	 >
%
%</text>
%<text stroke-width= "0.95"
%	 text-vert-align= "center-v"
%	 anchor-point= "(20,5)"
%	 fill-style= "none"
%	 stroke-color= "#ff0033"
%	 text-frame= "noframe"
%	 text-hor-align= "center-h"
%	 >
%1
%</text>
%<text stroke-width= "0.95"
%	 text-vert-align= "center-v"
%	 anchor-point= "(30,5)"
%	 fill-style= "none"
%	 stroke-color= "#ff0033"
%	 text-frame= "noframe"
%	 text-hor-align= "center-h"
%	 >
%2
%</text>
%<text stroke-width= "0.95"
%	 text-vert-align= "center-v"
%	 anchor-point= "(40,5)"
%	 fill-style= "none"
%	 stroke-color= "#ff0033"
%	 text-frame= "noframe"
%	 text-hor-align= "center-h"
%	 >
%3
%</text>
%<text stroke-width= "0.95"
%	 text-vert-align= "center-v"
%	 anchor-point= "(50,5)"
%	 fill-style= "none"
%	 stroke-color= "#ff0033"
%	 text-frame= "noframe"
%	 text-hor-align= "center-h"
%	 >
%4
%</text>
%<text stroke-width= "0.95"
%	 text-vert-align= "center-v"
%	 anchor-point= "(60,5)"
%	 fill-style= "none"
%	 stroke-color= "#ff0033"
%	 text-frame= "noframe"
%	 text-hor-align= "center-h"
%	 >
%5
%</text>
%<text stroke-width= "0.95"
%	 text-vert-align= "center-v"
%	 anchor-point= "(70,5)"
%	 fill-style= "none"
%	 stroke-color= "#ff0033"
%	 text-frame= "noframe"
%	 text-hor-align= "center-h"
%	 >
%6
%</text>
%<text stroke-width= "0.95"
%	 text-vert-align= "center-v"
%	 anchor-point= "(80,5)"
%	 fill-style= "none"
%	 stroke-color= "#ff0033"
%	 text-frame= "noframe"
%	 text-hor-align= "center-h"
%	 >
%7
%</text>
%<text stroke-width= "0.95"
%	 text-vert-align= "center-v"
%	 anchor-point= "(90,5)"
%	 fill-style= "none"
%	 stroke-color= "#ff0033"
%	 text-frame= "noframe"
%	 text-hor-align= "center-h"
%	 >
%8
%</text>
%<text stroke-width= "0.95"
%	 text-vert-align= "center-v"
%	 anchor-point= "(100,5)"
%	 fill-style= "none"
%	 stroke-color= "#ff0033"
%	 text-frame= "noframe"
%	 text-hor-align= "center-h"
%	 >
%9
%</text>
%<text stroke-width= "0.95"
%	 text-vert-align= "center-v"
%	 anchor-point= "(110,5)"
%	 fill-style= "none"
%	 stroke-color= "#ff0033"
%	 text-frame= "noframe"
%	 text-hor-align= "center-h"
%	 >
%10
%</text>
%<text stroke-width= "0.95"
%	 text-vert-align= "center-v"
%	 anchor-point= "(10,5)"
%	 fill-style= "none"
%	 stroke-color= "#ff0033"
%	 text-frame= "noframe"
%	 text-hor-align= "center-h"
%	 >
%0
%</text>
%<text stroke-width= "0.95"
%	 text-vert-align= "center-v"
%	 anchor-point= "(5,60)"
%	 fill-style= "none"
%	 stroke-color= "#ff0033"
%	 text-frame= "noframe"
%	 text-hor-align= "center-h"
%	 >
%1
%</text>
%</jpic>
%%End JPIC-XML
%LaTeX-picture environment using emulated lines and arcs
%You can rescale the whole picture (to 80% for instance) by using the command \def\JPicScale{0.8}
\ifx\JPicScale\undefined\def\JPicScale{1}\fi
\unitlength \JPicScale mm
\begin{picture}(125,65)(0,0)
\linethickness{0.3mm}
\put(10,10){\line(0,1){50}}
\linethickness{0.3mm}
\put(10,10){\line(1,0){100}}
\linethickness{0.3mm}
\multiput(30,10)(0.12,0.2){250}{\line(0,1){0.2}}
\linethickness{0.3mm}
\multiput(60,60)(0.12,-0.2){250}{\line(0,-1){0.2}}
\linethickness{0.9mm}
\multiput(80,10)(0.12,0.3){167}{\line(0,1){0.3}}
\linethickness{0.9mm}
\multiput(100,60)(0.12,-0.6){83}{\line(0,-1){0.6}}
\put(60,65){\makebox(0,0)[cc]{X}}

\put(100,65){\makebox(0,0)[cc]{Y}}

\put(125,40){\makebox(0,0)[cc]{}}

\put(20,5){\makebox(0,0)[cc]{1}}

\put(30,5){\makebox(0,0)[cc]{2}}

\put(40,5){\makebox(0,0)[cc]{3}}

\put(50,5){\makebox(0,0)[cc]{4}}

\put(60,5){\makebox(0,0)[cc]{5}}

\put(70,5){\makebox(0,0)[cc]{6}}

\put(80,5){\makebox(0,0)[cc]{7}}

\put(90,5){\makebox(0,0)[cc]{8}}

\put(100,5){\makebox(0,0)[cc]{9}}

\put(110,5){\makebox(0,0)[cc]{10}}

\put(10,5){\makebox(0,0)[cc]{0}}

\put(5,60){\makebox(0,0)[cc]{1}}

\end{picture}
