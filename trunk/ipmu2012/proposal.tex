The proposal consist on a possibilistic valid-time model. The representation and the querying are explained in the following subsections.

\subsection{Representation of Ill-known Valid-time Intervals}
\label{subsec:representation-ill-known}
Valid time is usually represented as an interval. The interval has a starting and an ending points. An ill-known valid-time interval is an interval in witch one or both points are ill-known. 

\begin{definition}
A Possibilistic Valid-Time Period \textbf{PVP} is a possibilistic interval defined by means of two ill-known points, namely $\left[ X,\ Y \right]$
\begin{equation}
PVP = \left[X,\ Y \right] 
\end{equation}
$X$ and $Y$ are ill-known values in the set of the real numbers $\mathbb{R}$. The uncertainty about the values taken by $X$ and $Y$ are given by the possibility distributions $\pi_X$ and $\pi_Y$.
\end{definition}

For convenience, the possibility distributions $\pi_X$ and $\pi_Y$ are given in the way of a triangular distribution, as explained in subsection \ref{subsec:fuzzy-numbers}. This representation allows overlapping (Fig. \ref{fig:pvp}). Note that while the two ill-known values $X,Y \in \mathbb{R}$, the fuzzy interval  $[X,Y] \in \mathbb{R}^2$.


\begin{figure}[h!]
  \centering
  %%Created by jPicEdt 1.4.1_03: mixed JPIC-XML/LaTeX format
%%Thu Jun 30 10:32:52 CEST 2011
%%Begin JPIC-XML
%<?xml version="1.0" standalone="yes"?>
%<jpic x-min="5" x-max="125" y-min="5" y-max="65" auto-bounding="true">
%<multicurve fill-style= "none"
%	 points= "(10,10);(10,10);(10,60);(10,60)"
%	 />
%<multicurve fill-style= "none"
%	 points= "(10,10);(10,10);(110,10);(110,10)"
%	 />
%<multicurve fill-style= "none"
%	 points= "(30,10);(30,10);(60,60);(60,60)"
%	 />
%<multicurve fill-style= "none"
%	 stroke-color= "#ccccff"
%	 points= "(60,60);(60,60);(90,10);(90,10)"
%	 />
%<multicurve stroke-width= "0.9"
%	 fill-style= "none"
%	 stroke-color= "#ccccff"
%	 points= "(80,10);(80,10);(100,60);(100,60)"
%	 />
%<multicurve stroke-width= "0.9"
%	 fill-style= "none"
%	 stroke-color= "#ff00cc"
%	 points= "(100,60);(100,60);(110,10);(110,10)"
%	 />
%<text stroke-width= "0.95"
%	 text-vert-align= "center-v"
%	 anchor-point= "(60,65)"
%	 fill-style= "none"
%	 stroke-color= "#ff0033"
%	 text-frame= "noframe"
%	 text-hor-align= "center-h"
%	 >
%X
%</text>
%<text stroke-width= "0.95"
%	 text-vert-align= "center-v"
%	 anchor-point= "(100,65)"
%	 fill-style= "none"
%	 stroke-color= "#ff0033"
%	 text-frame= "noframe"
%	 text-hor-align= "center-h"
%	 >
%Y
%</text>
%<text stroke-width= "0.95"
%	 text-vert-align= "center-v"
%	 anchor-point= "(125,40)"
%	 fill-style= "none"
%	 stroke-color= "#ff0033"
%	 text-frame= "noframe"
%	 text-hor-align= "center-h"
%	 >
%
%</text>
%<text stroke-width= "0.95"
%	 text-vert-align= "center-v"
%	 anchor-point= "(20,5)"
%	 fill-style= "none"
%	 stroke-color= "#ff0033"
%	 text-frame= "noframe"
%	 text-hor-align= "center-h"
%	 >
%1
%</text>
%<text stroke-width= "0.95"
%	 text-vert-align= "center-v"
%	 anchor-point= "(30,5)"
%	 fill-style= "none"
%	 stroke-color= "#ff0033"
%	 text-frame= "noframe"
%	 text-hor-align= "center-h"
%	 >
%2
%</text>
%<text stroke-width= "0.95"
%	 text-vert-align= "center-v"
%	 anchor-point= "(40,5)"
%	 fill-style= "none"
%	 stroke-color= "#ff0033"
%	 text-frame= "noframe"
%	 text-hor-align= "center-h"
%	 >
%3
%</text>
%<text stroke-width= "0.95"
%	 text-vert-align= "center-v"
%	 anchor-point= "(50,5)"
%	 fill-style= "none"
%	 stroke-color= "#ff0033"
%	 text-frame= "noframe"
%	 text-hor-align= "center-h"
%	 >
%4
%</text>
%<text stroke-width= "0.95"
%	 text-vert-align= "center-v"
%	 anchor-point= "(60,5)"
%	 fill-style= "none"
%	 stroke-color= "#ff0033"
%	 text-frame= "noframe"
%	 text-hor-align= "center-h"
%	 >
%5
%</text>
%<text stroke-width= "0.95"
%	 text-vert-align= "center-v"
%	 anchor-point= "(70,5)"
%	 fill-style= "none"
%	 stroke-color= "#ff0033"
%	 text-frame= "noframe"
%	 text-hor-align= "center-h"
%	 >
%6
%</text>
%<text stroke-width= "0.95"
%	 text-vert-align= "center-v"
%	 anchor-point= "(80,5)"
%	 fill-style= "none"
%	 stroke-color= "#ff0033"
%	 text-frame= "noframe"
%	 text-hor-align= "center-h"
%	 >
%7
%</text>
%<text stroke-width= "0.95"
%	 text-vert-align= "center-v"
%	 anchor-point= "(90,5)"
%	 fill-style= "none"
%	 stroke-color= "#ff0033"
%	 text-frame= "noframe"
%	 text-hor-align= "center-h"
%	 >
%8
%</text>
%<text stroke-width= "0.95"
%	 text-vert-align= "center-v"
%	 anchor-point= "(100,5)"
%	 fill-style= "none"
%	 stroke-color= "#ff0033"
%	 text-frame= "noframe"
%	 text-hor-align= "center-h"
%	 >
%9
%</text>
%<text stroke-width= "0.95"
%	 text-vert-align= "center-v"
%	 anchor-point= "(110,5)"
%	 fill-style= "none"
%	 stroke-color= "#ff0033"
%	 text-frame= "noframe"
%	 text-hor-align= "center-h"
%	 >
%10
%</text>
%<text stroke-width= "0.95"
%	 text-vert-align= "center-v"
%	 anchor-point= "(10,5)"
%	 fill-style= "none"
%	 stroke-color= "#ff0033"
%	 text-frame= "noframe"
%	 text-hor-align= "center-h"
%	 >
%0
%</text>
%<text stroke-width= "0.95"
%	 text-vert-align= "center-v"
%	 anchor-point= "(5,60)"
%	 fill-style= "none"
%	 stroke-color= "#ff0033"
%	 text-frame= "noframe"
%	 text-hor-align= "center-h"
%	 >
%1
%</text>
%</jpic>
%%End JPIC-XML
%LaTeX-picture environment using emulated lines and arcs
%You can rescale the whole picture (to 80% for instance) by using the command \def\JPicScale{0.8}
\ifx\JPicScale\undefined\def\JPicScale{1}\fi
\unitlength \JPicScale mm
\begin{picture}(125,65)(0,0)
\linethickness{0.3mm}
\put(10,10){\line(0,1){50}}
\linethickness{0.3mm}
\put(10,10){\line(1,0){100}}
\linethickness{0.3mm}
\multiput(30,10)(0.12,0.2){250}{\line(0,1){0.2}}
\linethickness{0.3mm}
\multiput(60,60)(0.12,-0.2){250}{\line(0,-1){0.2}}
\linethickness{0.9mm}
\multiput(80,10)(0.12,0.3){167}{\line(0,1){0.3}}
\linethickness{0.9mm}
\multiput(100,60)(0.12,-0.6){83}{\line(0,-1){0.6}}
\put(60,65){\makebox(0,0)[cc]{X}}

\put(100,65){\makebox(0,0)[cc]{Y}}

\put(125,40){\makebox(0,0)[cc]{}}

\put(20,5){\makebox(0,0)[cc]{1}}

\put(30,5){\makebox(0,0)[cc]{2}}

\put(40,5){\makebox(0,0)[cc]{3}}

\put(50,5){\makebox(0,0)[cc]{4}}

\put(60,5){\makebox(0,0)[cc]{5}}

\put(70,5){\makebox(0,0)[cc]{6}}

\put(80,5){\makebox(0,0)[cc]{7}}

\put(90,5){\makebox(0,0)[cc]{8}}

\put(100,5){\makebox(0,0)[cc]{9}}

\put(110,5){\makebox(0,0)[cc]{10}}

\put(10,5){\makebox(0,0)[cc]{0}}

\put(5,60){\makebox(0,0)[cc]{1}}

\end{picture}

  \caption{Two fuzzy numbers $X$ and $Y$ denoting a Possibilistic Valid-Time Period \emph{PVP}.}
  \label{fig:pvp}
\end{figure}

\subsection{Storage of Valid-time Intervals}
\label{subsec:storage}
Each database row containing a \emph{PVP} stores it as two triangular possibility distributions. In our approach we propose the representation of that as proposed in the  fuzzy interface for relational databases \emph{FIRST}~\cite{Medina94gefred.a,Gal98}. In this representation it is also possible to represent not only fuzzy numbers but fuzzy constants (see table \ref{table:relational-representation-pvp}). Note that the field \emph{FT} denotes the Fuzzy Type (\emph{NULL}, \emph{UNKNOWN}, \emph{UNDEFINED} or $\left[D,\ a,\ b \right]$) and the fields F1 to F3 store the value. Note also that while \emph{NULL} denotes the fuzzy constant, \emph{N} denotes a null value in the database.

%\begin{itemize}
%\item
%\emph{NULL}: This constant refers to a completely ignorance about the value. The possibility distribution for a given fuzzy number $X$ is not defined, therefore, any comparison between a fuzzy number and the \emph{NULL} constant always returns $0$.
%\item
%\emph{UNKNOWN}: % The possibility distribution for a given fuzzy number $X$ is $\pi_X=1$
%\item
%\emph{UNDEFINED}: The point does not have a value. %The possibility distribution for a given fuzzy number $X$ is $\pi_X=0$
%\end{itemize}
%

\begin{table}
\caption{Relational representation for an ill-known time point.}%  A \emph{PVP} is represented by two ill-known points.}
\centering
\begin{tabular}{c c c c c c c p{2cm}}
\hline
Value & FT & F1 & F2 & F3 & $\mu(x)$ & Description \\ \hline
UNKNOWN & 0 & N & N & N  & $1$ & The point has a value but it is unknown. \\ 
UNDEFINED & 1 & N & N & N & $0$ & The point does not have a value. \\ 
NULL & 2 & N & N & N &not defined. & Completely ignorance aboute the value. \\ 
$\left[D,\ a,\ b \right]$ & 3 & $D$ & $a$ & $b$ & $\mu(\left[D,\ a,\ b \right])$ & triangular poss. distr. \\ 
\hline
\end{tabular}
\label{table:relational-representation-pvp}
\end{table}

\subsection{Querying Ill-known Valid-time Intervals}
In order to provide a complete model, we will provide the tools for querying. This allows the user to specify both the preferences and an ill-known valid-time interval in the query. It is important to notice that the possibilistic / fuzzy data stored in the database has a \emph{disjunctive interpretation} (it is said that we have \emph{uncertainty}: the valid-time interval has only one value but, for some reason the value is ill-known). In the query specification, the user is allowed to express a crisp time interval. 
In this subsection we will define the query specification, then the evaluation of the query and finally the ranking for the query.

\subsubsection{Query specification}
A query in our framework has two different parts: the first one is the temporal specification. The second part is the query specification for regular attributes.

\begin{definition}
A query $\tilde Q$ is specified by:
\begin{equation}
\label{eq:query-definition}
\tilde Q = \left( Q^{time}, Q \right)
\end{equation}
\end{definition}
Where $Q$ are the (possibly fuzzy) preferences of the user  and $Q^{time}$ is the temporal part specified by a crisp interval.
\begin{definition}
 $Q^{time}$ is composed by:
\begin{equation}
Q^{time} = \left( \left[a, \ b \right] , Ar \right), a,b \in \mathbb{R}
\end{equation}
Where $ \left[a, \ b \right] $ is the specification for the crisp interval and $Ar$ is one of the Allen's relations (table \ref{tab:allen-relations}).
\end{definition}

\begin{table}[h]
\label{tab:allen-relations}
\centering
\begin{tabular}{|c|c|c|c|}
\hline
Allen Relation & Constraints & $\bool(p_1,...,p_n)$ & $\Pos\left(\lambda \left[a,b \right] \right)$ \\
\hline
I before J & $C_1\stackrel{\triangle}{=} \left(<,X\right)$ & $p_1$ & $\sup_{a>w}\pi_x(w)$\\
\hline
\multirow{2}{*}
{I equal J} & $C_1\stackrel{\triangle}{=} \left(\geq,X\right)$,$C_2\stackrel{\triangle}{=} \left(\neq,X\right)$ & \text{$p_1\wedge\neg p_2\wedge p_3\wedge\neg p_4$} & $\min ( \sup_{a \leq w}\pi_x(w),\pi_x(w),$\\
 & $C_3\stackrel{\triangle}{=} \left(\leq,Y\right)$,$C_4\stackrel{\triangle}{=} \left(\neq,Y\right)$ & & $\sup_{b \geq w}\pi_Y(w),\pi_Y(w))$\\
% &  & &  \\
% &  & &  \\
\hline
%\multirow{2}{*}
{I meets J} & $C_1\stackrel{\triangle}{=} \left(\leq,X\right)$ ,$C_2\stackrel{\triangle}{=} \left(\neq,X\right)$& $p_1\wedge\neg p_2$ & $\min (\sup_{a\geq w} \pi_X(w),$ $\pi_X(w))$ \\
% &  &  & \\
\hline
\multirow{3}{*}
{I overlaps J} & $C_1\stackrel{\triangle}{=} \left(<,Y\right)$,$C_2\stackrel{\triangle}{=} \left(\leq,X\right)$ & $p_1\wedge\neg p_2\wedge\neg p_3$ & $\min ( \sup_{b>w}\pi_Y(w), $ \\
 & $C_3\stackrel{\triangle}{=} \left(\geq,X\right)$ & & $\sup_{a \geq w}\pi_X(w),$ \\
 &  & & $\sup_{a \leq w}\pi_X(w))$ \\
\hline
\multirow{4}{*}
{I during J} & $C_1\stackrel{\triangle}{=} \left(>,X\right)$, $C_2\stackrel{\triangle}{=} \left(\leq,Y\right)$ & $(p_1\wedge p_2)\vee(p_3\wedge p_4)$ & $max ( min ( \sup_{a<w}\pi_X(w),$ \\
 & $C_3\stackrel{\triangle}{=} \left(\geq,X\right)$ ,$C_4\stackrel{\triangle}{=} \left(<,Y\right)$ & & $\sup_{b \geq w}\pi_Y(w)),$\\
 & & & $\min ( \sup_{a \leq w }\pi_X(w),$ \\
 &  & & $\sup_{b>w}\pi_Y(w)$\\
\hline
%\multirow{2}{*}
{I starts J} & $C_1\stackrel{\triangle}{=} \left(\geq,X\right)$,$C_2\stackrel{\triangle}{=} \left(\neq,X\right)$  & $p_1\wedge\neg p_2$ & $\min( \sup_{a \leq w}\pi_X(w),$ $\pi_X(w))$\\
% &  & & \\
\hline
%\multirow{2}{*}
{I finishes J} & $C_1\stackrel{\triangle}{=} \left(\leq,Y\right)$, $C_2\stackrel{\triangle}{=} \left(\neq,Y\right)$  & $p_1\wedge\neg p_2$ & $\min ( \sup_{b \geq w} \pi_Y(w),$ $\pi_Y(w))$\\
% && & \\
\hline
\end{tabular}

\caption{Allen's relations represented in the framework.}
\end{table}




\subsubsection{Query evaluation}
In fuzzy querying of regular (relational) databases, the query satisfaction modelling is a matter of degree. Usually, criteria satisfaction is modelled by means of a satisfaction degree $s \in \left[ 0, 1\right]$. In the model, every record $r$ contains a \emph{PVP}, $V_r$, to model the valid-time.The query evaluation method is the following:
\begin{itemize}
\item
For each record $r$ in the database, the preferences expressed in $Q$ are evaluated in the unit interval. Thus, $e(Q) \in \left[0, 1\right]$ is the evaluation function for the criteria in $Q$.
\item
For each record $r$ in the database, the interval $ \left[a, \ b \right]$ is compared to the $PVP$ value by means of the corresponding Allen relation specified by $ar$. Here, $e(Q^{time}) \in \left[0, 1 \right]$ is the evaluation function for the criteria $Q^{time}$.
\end{itemize}


\subsubsection{Ranking and aggregation}
In order to present the results to the user, a ranking method must be used. Both $Q$ and $Q^{time}$ are evaluated independently and also both are evaluated in the unit interval.
First of all, the possibility and the necessity are ranked with the convex combination in \eqref{eq:convex-comb} with $\omega=0.5$. Finally this value is aggregated with the rest of the criteria with the same combination. %In this case the value for the aggregation is also $\omega = 0.5$

\begin{equation}
\label{eq:convex-comb}
e(u,v)\ =\ \omega*v\ +\ (1-\omega)*v, \omega \in \left[0, 1 \right]
\end{equation}

 By increasing the value for the parameter $\omega$, the preferences expressed in the query $Q$ can be given more importance. By lowering the value for $\omega$, the temporal criteria is emphasized.
%The following example illustrates the querying process.


\begin{example} 

\textbf{The database}
%%%%%%%%%%%%%%%%%%%%%%%%%%%%%%%%%%%%%%%%%%%%%%%%%%
% Sample table for the car models. 
%%%%%%%%%%%%%%%%%%%%%%%%%%%%%%%%%%%%%%%%%%%%%%%%%%%
\begin{table}[ht]
\caption{Sample database containing the car models.}
\centering
\begin{tabular}{c c c c c c c}
\hline
ID & IID & Segment & Manufac. & Name & Start & End  \\ [0.5ex]
\hline
001 & 1 & B & Peugeot & 205 & [1985,2,3] & [1997,2,1] \\
002 & 1 & C & Peugeot & 305 & [1977,2,2] & [1989,2,3] \\
003 & 1 & B & Citroen & C2 & [2002,2,2] & [2006,1,1] \\
001 & 2 & B & Peugeot & 206 & [2000,1,2] & [2011,2,1] \\
001 & 3 & B & Peugeot & 207 & [2006,1,1] & [2011,1,1]\\
\hline
\end{tabular}
\label{tb:car-models}
\end{table}


Consider a database containing car models. There are several general attributes (car model's name, car manufacturer, car segment) and one temporal attribute (which is ill-known) containing the approximate date while the car model was sold. The temporal data is stored as explained in subsection \ref{subsec:storage}. The value for $D$ is stored in \emph{yyyy} format and $a$ and $b$ are represented by an integer, for the convenience of the representation. Therefore the chronons in our example will be years. The ID field identifies a car model while the field Instance ID (IID) identifies the instance for a car model. \\

\textbf{The query}
Consider the following query:

\emph{The user wants to obtain a list of models in the segment B for the manufacturer Peugeot before the year interval 2001-2005.}

The query is translated to the following expression, using equation \eqref{eq:query-definition}. The evaluation for the criteria are presented in the result table \ref{tb:results}.

\begin{center}
$\tilde{Q} = \left(  c^{time}, c_{segment} \right)$
\end{center}

Where:
\begin{itemize}
\item
$c^{time}\ = \ ( \left[ 2001,\ 2005 \right],\ $  before $)$.
\item
$c_{segment}\ = \ $ Peugeot.
\end{itemize}



\begin{table}[ht]
\caption{Result table and ranking}
\centering
\begin{tabular}{c c c c c c c}
\hline
ID & IID &  $\Pos_{c^{time}}$ &$\Nec_{c^{time}}$ & $e(C^{time})$ & $c_{segment}$ & rank ($\omega=0.5$) \\ [0.5ex]
\hline
001 & 1 & 1 &  1 & 1 & 1 & 1 \\
002 & 1 & 1 & 1 & 1 & 0.5 & 0.75 \\
003 & 1 & 1 & 0.5 & 0.75 &0 & 0.375\\
001 & 2 & 1 & 0 & 0.5 &1 & 0.75 \\
001 & 3 & 0 & 0 & 0 &1 & 0.5\\
\hline
\end{tabular}
\label{tb:results}
\end{table}
\end{example}




