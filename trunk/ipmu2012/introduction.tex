Dealing with time in any information system~\cite{Bolour82} is usually a difficult task. First of all, the concept of \emph{time} itself is a complex notion. Klein~\cite{klein94} studied the concept of time in language whereas Devos~\cite{devos94} modelled vague temporal expressions by means of fuzzy sets~\cite{zadeh65}.  Several research works~\cite{cruyssen97,devos98,Cru97} have been done in obtaining theoretical models that allow the representation of time in an information system. Although the main tool for the representation of temporal information are fuzzy sets~\cite{Virant199639,35337,343607,nagypal03,Dubois:jucs_9_9:fuzziness_and_uncertainty_in}, some other tools, like rough sets~\cite{pawlak95} have been used~\cite{Qia09}.
%granularity
Granularity is the abstract time levels that the people use. The change among granularities~\cite{Lin97} is considered also as a source of imprecision. Therefore, some proposals consider the granularity as the base of the temporal model~\cite{Cru97}.

The temporal relations between time points and intervals are studied first by Allen~\cite{Allen83} and recently fuzzified by several authors~\cite{ohlbach04,nagypal03,schockaert08}.

In the other hand, the addition of time to a relational database~\cite{Dyreson1994} leads to several problems: the modelling of a time-variant object has an impact on the database consistency. Different models have been proposed \cite{Jensen94,TSQL,Sarda90,Jensen91,Snodgrass84,Nascimento95} but all of them consider the time as a crisp value. 
%There exist a glossary with the common terms in the temporal database community.
Whereas the concept of time is intrinsically imprecise, the first temporal database models were crisp.
In recent works, some authors consider the necessity of allowing imprecission and vagueness in a temporal model~\cite{Cru97,624013,fuzz2009,gal01}. 

There are several proposals for transforming two fuzzy numbers which represent a fuzzy interval into a unified representation as a single fuzzy number. These transformations (the convex hull and the transformation that preserves the imprecision)~\cite{fuzz2009} are misleading. 

In this work we will propose a possibilistic valid time model for relational databases. The basis of the model is possibilistic framework presented in~\cite{Pon11}. The framework deals with the inference of uncertainty about sets from the uncertainty about values. 

The paper is organized as follows. In Section \ref{sec:preliminaries}, some preliminaries and concepts about the possibilistic evaluation framework and the temporal databases are presented. 
Section \ref{sec:proposal} presents the possibilistic model for valid-time databases. In this section is introduced both the representation and the querying.
Finally, Section \ref{sec:futher-research} presents the main conclusions and some lines for the future work in this research.