\begin{definition}
\label{def:generalized-fuzzy-temporal-domain}
\emph{Generalized fuzzy temporal domain.}
Consider $\T$ to be the temporal domain, and let $\tilde \Pow\left( \T\right)$ be the set of all \emph{normalized} possibility distributions (see Section \ref{subsec:possibility-theory}, equation \eqref{NormalizationProperty}) defined on $\T$.
The Generalized Fuzzy Temporal Domain, $\T_G$ is
\begin{equation}
\T_G \subseteq \left \lbrace \tilde \Pow\left( \T\right) \cup \text{NULL} \right \rbrace
\end{equation}
\end{definition}

Note that $\T_G \subseteq D_G$. The datatypes for this domain have been studied previously in section \ref{sec:time-rep} and are shown in tables \ref{tbl:time-point-types} and \ref{tbl:time-interval-types}.

A generalized fuzzy relation is defined in \cite{Medina1994}. Here, we will extend the definition to a generalized fuzzy temporal relation.

\begin{definition}
\emph{Generalized fuzzy temporal relation.}
\label{def:fuzzy-temporal-relation}
Consider the elements in definition \ref{def:valid-time-relation}. Some of them will be extended for the fuzzy case.

\begin{itemize}
%  \item A set of non-temporal fuzzy  or crisp attributes.
% 	\begin{equation}
% 	\label{eq:fuzzy-attribute-set}
% 	A = \left \lbrace A_1, A_2, \ldots, A_n \right \rbrace
% 	\end{equation}
%       The domain for each attribute $A_1, \ldots, A_n$ is $D_1, \ldots, D_n$ respectively. 
% \item The primary key $A_K$ is a subset of $A$.
%       \begin{equation}
%        \label{eq:fuzzy-primary-key-a}
%       A_K \subset A
%       \end{equation}
% A formal definition of primary key for fuzzy relational databases will be given later in Definition \ref{def:generalized-primary-key}.
% \item A set of two attributes; $S$  and $E$ the attributes for the starting and ending ill-known points respectively. $I$ is a possibilistic validity period \emph{PVP} as explained in Section \ref{subsec:ill-known-interval}.
% \begin{equation}
%  \label{eq:fuzzy-attribute-time-interval}
% I = \left \lbrace S, E \right \rbrace
% \end{equation}
\item An attribute called version identifier, $V_{ID}$, will be added to the schema. This attribute is a counter for each different version of the entities. 
% \begin{equation}
%  \label{eq:fuzzy-version-identifier}
% V_{ID} \subset \N
% \end{equation}



\item Then $R_{FTG}$, the schema for the fuzzy valid-time relation is:
\begin{equation}
 \label{eq:fuzzy-valid-time-relation}
R_{FTG} = A \cup V_{ID} \cup  I
\end{equation}
\item The primary key for the fuzzy valid-time relation $R_{FTG}$ is:
\begin{equation}
 \label{eq:fuzzy-valid-time-temporal-pk}
K_{GT} = A_K \cup V_{ID}
\end{equation}
A formal definition of the primary key for fuzzy valid-time relations will be given later in Definition \ref{def:generalized-fuzzy-temporal-key}.


\item We will note by $r$ any valid instance of $R_{FTG}$. 
      \begin{equation}
       \label{eq:fuzzy-valid-time-instance}
      r \subseteq D_1\ x\ \ldots\ x\ D_n
      \end{equation}

% \item Let $t$ be a tuple in the instance $r$; $t \in r$:
%       \begin{itemize}
%       \item We will note the values for the starting and the ending  points,$s$ and $e$  respectively. The value for the time interval is given by $i$.
%       \begin{align}
%        \label{eq:fuzzy-starting-point}
%       s = t\left[S \right]\\
%       e = t\left[E \right]\\
%       i = \left(s, e\right)
%       \end{align}
% 
% 
%       \item Let $A_k \subset A$ be the set of the non-temporal attributes that are part of the primary key (equation \eqref{eq:fuzzy-primary-key-a}). Then, $a_k$ denotes the values for the non-temporal attributes of the primary key.
% 	    \begin{equation}
% 	     \label{eq:fuzzy-pk-attribute} 
% 	      a_k = t\left[A_K \right]
% 	    \end{equation}
% 
    \item Let $K_{GT}$ be the primary key for the valid-time relation as given in equation \eqref{eq:fuzzy-valid-time-temporal-pk}. Then, $k$ denotes the values for the attributes in the primary key.
	  \begin{equation}
	   \label{eq:fuzzy-value-pk}
	  k = t\left[K_{GT} \right]
	  \end{equation}
% 
% 
%       \end{itemize}
% \item $Vt$ is the set with all the versions for a given tuple $t$.
% 
% \begin{equation}
%  \label{eq:fuzzy-all-the-versions}
% Vt = r\left(t\left[A_K\right] \right)
% \end{equation}
% If a tuple $t$ contains more than one version, the result of $Vt$ is a set. We will use the index $k$ to address the elements of the set. E.g. $Vt = \left \lbrace t_1, t_2, \ldots, t_n \right \rbrace$. Then $t_k , k \in \left \lbrace 1, \ldots, n \right \rbrace$ is an element of $Vt$.
\end{itemize}
%\end{definition}

We will illustrate the definitions with an example.

\begin{example}
Consider the set of attributes $A = \left \lbrace A_1, A_2, A_3 \right \rbrace$. The primary key for these attributes is given by $A_K = \left \lbrace A_1, A_2 \right \rbrace$. Let $I = \left \lbrace S, E \right \rbrace$ be the set of temporal attributes that define the possibilistic validity period of the data. $R_{FTG} = A \cup V_{ID} \cup I$ is the valid-time relation and $r$ is an instance of the relation. The instance $r$ is given by the following elements. $r = \left \lbrace \left(a_{11}, a_{12}, a_{13}, 001, s_1 ,e_1 \right) \right.$,  $\left(a_{21}, a_{22}, a_{23}, 001, s_2, e_2 \right)$ , $\left. \left(a_{11}, a_{12}, a_{31}, 002, s_3, e_3 \right) \right \rbrace$. The instance $r$ is illustrated in Table \ref{tbl:fuzzy-sample-definitions}. 
Consider the tuple $t = \left(a_{11}, a_{12}, a_{13}, s_1 ,e_1 \right)$. Then,
\begin{align}
 \nonumber
s_1 &= t\left[S \right]\\
 \nonumber
e_1 &= t[E]\\
 \nonumber
i\ \ &= \left(s_1, e_1\right)\\
 \nonumber
a_k &= t\left[A_K\right] = \left(a_{11}, a_{12} \right) \\
 \nonumber
k &= t\left[PK\right] =\left(a_{11}, a_{12}, s_1, e_1\right)\\
 \nonumber
Vt &= r\left(t\left[A_K\right] \right) = \left \lbrace t_1, t_3 \right \rbrace
\end{align}



\end{example}

\begin{table}[h]
\caption[Example of fuzzy valid-time relation.]{\label{tbl:fuzzy-sample-definitions}Sample database containing the instance $r$ of the fuzzy valid-time relation $R_{FTG}$.}
\centering
\begin{tabular}{c c c c c c c}\\ \hline
& \textbf{A$_1$}  & \textbf{A$_2$}  & $A_3$ & $V_{ID}$ & $S$ & $E$ \\
\hline
$t_1$&$a_{11}$ & $a_{12}$ & $a_{13}$ & $001$ & $s_1$ & $e_1$ \\
$t_2$ & $a_{21}$ & $a_{22}$ & $a_{23}$& $001$ &  $s_2$ & $e_2$ \\
$t_3$ & $a_{11}$ & $a_{12}$ & $a_{31}$& $002$ & $s_3$ & $e_3$\\
\hline\\
\end{tabular}
\end{table}



% 
% 
% Consider $\A$ to be the set of all the entities, and let $A=\left(A_1, \ldots, A_n \right), A \in \A$ be the set of (fuzzy) attributes that define an entity. Let $\I_{PVP}$ be the set of all the ill-known time intervals and let $I = \left(X, Y \right)$ be a PVP, $I \in \I_{PVP}$.  The pair $\left(A, I\right)$ expresses that the data regarding the entity A are valid during the ill-known time interval I. Let $R_{FTG} \subseteq \A \  x\  \I_{PVP}$ be a valid-time relation. Then, the following equation indicates that the pair $\left(A, I\right)$ is in the relation R:
% 
% \begin{equation}
% \label{eq:fuzzy-rel-def}
% \left( R_{FTG}, \left(A , I\right) \right)
% \end{equation}
% 
 A generalized fuzzy temporal relation $R_{FTG}$ can be noted also by:
\label{def:generalized-fuzzy-temporal-relation}
\begin{equation}
\label{eq:generalized-fuzzy-temporal-relation}
R_{FTG} = \left(\Head, \Body \right)
\end{equation}
Where $\Head$ is the Head of the relation and consist on a fixed set of triplets attribute- domain - compatibility with an optional the valid-time attribute:

\begin{align}
\label{eq:head-valid-time}
\Head = \big \lbrace \left(A_{G1}:D_{G1}\left[,C_{A_{G1}} \right] \right),\\
\nonumber
 \ldots,\\
 \nonumber
  \left(A_{Gn}:D_{Gn}\left[,C_{A_{Gn}} \right] \right),\\
  \nonumber
  \Big[  \left( \text{PVP}, D_{\text{PVP}}\left[,C_{A_{\text{PVP}}} \right] \right) \Big] \big \rbrace
\end{align}
Note that $D_{Gj}$ ($j = 1, \ldots, n$) is the domain for the attribute $A_{Gj}$. $C_{A_{Gj}}$ is the compatibility attribute in the unit interval $\left[0, 1 \right]$.

$\Body$ is the body of the relation and it consists on a set of tuples. Each tuple is a triplet attribute- value- degree with an optional valid-time attribute:

\begin{align}
\label{eq:body-valid-time}
\Body = \big \lbrace \left(A_{G1}:\tilde{d}_{i1}\left[,c_{i1} \right] \right),\\
\nonumber
 \ldots,\\
 \nonumber
  \left(A_{Gn}:\tilde{d}_{in}\left[,c_{in} \right] \right),\\
  \nonumber
   \Big[  \left( \text{PVP}, \tilde{d}_{\text{PVP}} \left[,C_{A_{\text{PVP}}} \right] \right)  \Big] \big \rbrace
\end{align}

\end{definition}


The definition in \cite{Medina1994} for $R_{FTG}$ shows that classical relations are a particular case of this model. 
