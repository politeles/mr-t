\documentclass{llncs}
%
%
%espero que no se enfaden por incluir este paquete:
\usepackage{amsmath}
\usepackage{amssymb}
\usepackage[dvips]{graphicx}
\usepackage{makeidx}  % allows for indexgeneration
\usepackage{color}

\hyphenation{pro-per-ties}
\hyphenation{ge-ne-ral-ly}
\hyphenation{pre-fe-ren-ces}
\hyphenation{u-sing}
\hyphenation{pu-nish-ment}
\newcommand{\Pow}{\mathcal{P}}
\newcommand{\N}{\operatorname{N}}
\newcommand{\bool}{\operatorname*{\mathcal{B}}}
\newcommand{\Pos}{\operatorname{Pos}}
\newcommand{\Nec}{\operatorname{Nec}}
\newcommand{\Open}{\operatorname{Open}}
\newcommand{\Rp}{\operatorname{Rp}}
\newcommand{\Rn}{\operatorname{Rn}}
\newcommand{\FB}{\operatorname{FB}}
\newcommand{\UC}{\operatorname{UC}}
\newcommand{\T}{\mathcal{T}}
\newcommand{\Rel}{\mathcal{R}}
\newcommand{\C}{\mathcal{C}}
\newcommand{\Nat}{\mathbb{N}}
\newcommand{\Boolean}{\mathbb{B}}
\newcommand{\Q}{\mathbb{Q}}
\newcommand{\R}{\mathbb{R}}
\newcommand{\Z}{\mathbb{Z}}
\newcommand{\Head}{\mathcal{H}}
\newcommand{\Body}{\mathcal{B}}
\newcommand{\Dom}{\mathcal{D}}
\newcommand{\INS}{\mbox{\textbf{insert}}}
\newcommand{\DEL}{\mbox{\textbf{delete}}}
\newcommand{\MOD}{\mbox{\textbf{modify}}}
\newcommand{\REV}{\mbox{\textbf{revise}}}
\newcommand{\Overlap}{\mbox{\textbf{Overlap}}}
\newcommand{\A}{\mathcal{A}}
\newcommand{\I}{\mathcal{I}}
\begin{document}

\mainmatter              % start of the contributions
%
\title{Data Manipulation Language for fuzzy temporal databases.}
%
%\titlerunning{A Fuzzy Valid-Time Model}  % abbreviated title (for running head)
%                                     also used for the TOC unless
%                                     \toctitle is used
%
\author{Jos\'e Enrique Pons \and Olga Pons Capote}
%Jeffrey Dean \and David Grove \and Craig Chambers \and Kim~B.~Bruce \and
%Elsa Bertino}
%
\authorrunning{Jos\'e Enrique Pons et al.} % abbreviated author list (for running head)
%
%%%% list of authors for the TOC (use if author list has to be modified)
%\tocauthor{Ivar Ekeland, Roger Temam, Jeffrey Dean, David Grove,
%Craig Chambers, Kim B. Bruce, and Elisa Bertino}
%
\institute{
Department of Computer Science and Artificial Intelligence \\
Universidad de Granada \\
Escuela T\'ecnica Superior de Ingenier\'ia Inform\'atica \\
C/Periodista Daniel Saucedo Aranda s/n \\
E-18071 (Granada-Spain)\\
\email{jpons,opc@decsai.ugr.es}\\ 
%WWW home page:
%\texttt{http://users/\homedir iekeland/web/welcome.html}
%\and
%Universit\'{e} de Paris-Sud,
%Laboratoire d'Analyse Num\'{e}rique, B\^{a}timent 425,\\
%F-91405 Orsay Cedex, France
}

\maketitle              % typeset the title of the contribution

\begin{abstract}

\end{abstract}


\section{\label{sec:selection}Selection}
The selection operator $\sigma$ obtain a subset of tuples that fulfill a set of constraints from a given relation $R$. The set of constraints is usually a boolean combination of atomic constraints. The selection operator is noted as follows:
\begin{equation}
 \label{eq:selection}
\sigma_{S} \left( R \right)
\end{equation}

Where $R$ is the relation, and $S$ is the selection formula. The selection formula is a set of two elements:
% \begin{itemize}
% \item relation.attribute boolean relation relation.attribute
% \item relation.temporal-attribute allen-relation relation.temporal-attribute 
% \end{itemize}
\begin{equation}
 \label{eq:selection_formula}
S = \left \lbrace Q, Q^{t}\right \rbrace
\end{equation}

Where $q$ is a boolean combination of (possibly) fuzzy (and bipolar) non-temporal constraints and $q^{t}$ is also a boolean combination of fuzzy temporal constraints.

\begin{equation}
 \label{eq:non-temporal-constraints}
Q = \left \lbrace q^{\left \lbrace \mbox{pos,neg} \right \rbrace}_{\mbox{attribute}_1}  \theta \mbox{ val }, \ldots, q^{\left \lbrace \mbox{pos,neg} \right \rbrace}_{\mbox{attribute}_n}  \theta \mbox{ val } \right \rbrace
\end{equation}
Where $\theta$ is an artihmetic operator; one of $\left \lbrace =, \neq, <, \leq, >, \geq \right \rbrace$ and val is a value in the domain of the attribute. 

Then, consider a boolean function $\bool:\Boolean^{n}  \rightarrow \Boolean$. The evaluation function for the constraints in $Q$  is given by:

\begin{equation}
 \label{eq:evaluation-function}
\lambda : Q \rightarrow \Boolean : \lambda (Q) \rightarrow \bool \left(q^{\left \lbrace \mbox{pos,neg} \right \rbrace}_{\mbox{attribute}_1}  \theta \mbox{ val }, \ldots, q^{\left \lbrace \mbox{pos,neg} \right \rbrace}_{\mbox{attribute}_n}  \theta \mbox{ val } \right)
\end{equation}

Now, the uncertainty about the evaluation of $Q$ and a boolean function $\bool:\Boolean^{n}  \rightarrow \Boolean$ is given by:

\begin{equation}
 \label{eq:evaluation-lambda-function}
\pi_{\lambda \left(Q \right)} = \tilde{\bool} \left(\pi_{q^{\left \lbrace \mbox{pos,neg} \right \rbrace}_{\mbox{attribute}_1}  \theta \mbox{ val }}, \ldots, \pi_{q^{\left \lbrace \mbox{pos,neg} \right \rbrace}_{\mbox{attribute}_n}  \theta \mbox{ val }} \right)
\end{equation}


The temporal constraints, $Q^t$ are provided in a similar way. The main difference is that, instead of comparations like $\left \lbrace =, \neq, <, \leq, >, \geq \right \rbrace$, we use the allen's relations. Hence, the representation of the temporal constraints is expressed as follows:

\begin{equation}
\label{eq:temporal-constratins}
Q^{t} = \left \lbrace q_{\mbox{attribute}_1}  ar \mbox{ val }, \ldots, q_{\mbox{attribute}_n}  ar \mbox{ val } \right \rbrace
\end{equation}
Here, $ar$ is one of the thirdteen Allen's relations (equal, before, overlaps, starts, finishes, meets, during) and their respective reverses.


Now the combination of temporal and non-temporal results should be explicited here.


\section{\label{sec:join}Join}
The join operator $\Join$ builds a new relation from two given relations, namely $R$ and $S$. This new relation is a set with all the possible combinations of tuples in both $R$ and $S$. It is usually noted as $R \Join_{r \theta s} S$, where $r,s$ are attributes from $R$ and $S$ respectively and $\theta$ an arithmetic operator.

The following example illustrates the behaviour of the join operator in a crisp temporal database.

\begin{example}
Consider the employees database shown in Table \ref{}.  
\end{example}



\section*{Acknowledgements}
%
The researchers are supported by the grant BES-2009-013805 within the research project TIN2008-02066: \emph{Fuzzy Temporal Information treatment in relational DBMS}.

\bibliographystyle{splncs03}
\bibliography{biblio}




\end{document}