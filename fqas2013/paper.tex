%%FQAS 2013 Contribution


\documentclass[runningheads,a4paper]{llncs}

\usepackage{amssymb}
\setcounter{tocdepth}{3}
\usepackage{graphicx}

\begin{document}

\mainmatter  %start of the contribution

\title{Bipolar Querying of Valid-time Intervals Subject to Uncertainty}

%Short form of title for running heads
\titlerunning{Bipolar Valid-time Querying}

\author{Christophe Billiet\inst{1} \and Jos\'{e} Enrique Pons\inst{2} \and Olga Pons\inst{2} \and Guy De Tr\'{e}\inst{1}}

\authorrunning{Christophe Billiet et al.}

\institute{
Department of Telecommunications and Information Processing, Ghent University,\\
Sint-Pietersnieuwstraat 41, B-9000, Ghent, Belgium\\
Christophe.Billiet@UGent.be, Guy.DeTr\'{e}@UGent.be\\
\and
Department of Computer Science and Artificial Intelligence, University of Granada,\\
C/Periodista Daniel Saucedo Aranda, S/N, E-18071, Granada, Spain\\
jpons@decsai.ugr.es, opc@decsai.ugr.es
}

\maketitle


\begin{abstract}
Databases model parts of reality by containing data representing properties of real-world objects or concepts. Often, some of these properties are time-related. Thus, databases often contain data representing temporal values. However, as they may be produced by humans, such data may contain imperfections like uncertainties. Now, an important purpose of databases is to allow their data to be queried to allow access to the information these data represent. In queries, users describe their preferences concerning the data they are interested in. Because users may have both positive and negative preferences, they may want to query databases in a bipolar way. Such preferences may also have a temporal nature. In this paper, a novel technique is presented to query a valid-time relation containing uncertain valid-time data in a bipolar way, which allows the query to have a single bipolar temporal query constraint.

\keywords{Bipolar Querying, Valid-time Relation, Uncertainty, Temporal Databases, Ill-known Time Intervals}
\end{abstract}


\section{Introduction}
The general purpose of database systems (DS) is to model (a part of) reality. Therefore, databases contain data representing properties of real-world objects or concepts~\cite{Pons2012ijcis},~\cite{Billiet2012ipmu},~\cite{Pons2012ipmu}. Some essential properties of real-world objects or concepts are time-related. Thus, databases often contain data representing temporal values which describe such properties~\cite{Pons2012ijcis},~\cite{Bolour1982acm},~\cite{Jensen1999ieeetkde},~\cite{Galindo2001}. These temporal values are usually either time intervals~\cite{Bohlen1998lncs} or instants~\cite{Bohlen1998lncs}, which may informally be seen as infinitesimally short `points' in time. Based on their interpretation and purpose, temporal values can be classified into several categories, but the presented work will only consider valid-time notions, which indicate when database data is a true representation of the reality modelled by the database~\cite{Bohlen1998lncs},~\cite{Pons2012ijcis},~\cite{Billiet2012ipmu},~\cite{Pons2012ipmu}.

A lot of data in DS are produced by humans, but human-made data are prone to imperfections: some data may be vague, imprecise, contradictory, incomplete or uncertain~\cite{Billiet2012ipmu},~\cite{Pons2012ipmu},~\cite{Medina1994is},~\cite{Bosc2010ijufkbs},~\cite{Bosc2009sum}. Of course, such imperfections may arise in data representing temporal notions too~\cite{Billiet2012ipmu},~\cite{Pons2012ipmu},~\cite{Pons2012ijcis},~\cite{Dyreson1998acm}. The presented work will consider DS containing data representing time intervals subject to uncertainty and will assume all non-temporal data in these DS to contain no imperfections.

One of the most important purposes of a database is allowing its data to be queried, which allows the information or knowledge represented by the data to be retrieved. A database may be queried in a `regular' way: the user describes the data which he or she finds desired or satisfactory and thus wants to retrieve, by perfectly describing the preferred values of this data. A database may also be queried in a `fuzzy' way: the user describes the data which he or she finds desired or satisfactory by imperfectly describing the preferred values of this data ~\cite{Zadrozny2008hrfipd},~\cite{DeCaluwe2007ijis}. These imperfect descriptions may contain vagueness or imprecision, often through the use of linguistic terms ~\cite{Devos1998jql},~\cite{Kacprzyk2001is}. A database may also be queried in a `bipolar' way. Generally, two main approaches to this exist. One is for the user to describe the data which he or she finds acceptable and to describe the data among this acceptable data, which he or she finds really desired (wished-for), both by describing the preferred values of this data ~\cite{Dubois2008hrfipd},~\cite{Dubois2002lnai}. The other is for the user to describe both the data which he or she finds desired or satisfactory and the data which he or she finds undesired or unsatisfactory, both by describing the preferred values of this data~\cite{DeTre2010ieeetfs},~\cite{Matthe2011ijis}. The descriptions used in bipolar querying might or might not contain imperfections. The presented work will only consider the latter approach to bipolar querying and will allow a simple form of imprecision in non-temporal elementary query conditions.

In querying, temporal preferences are usually expressed specifically. Because of their temporal nature and interpretation, users usually prefer using specific temporal operators to express temporal preferences. Hence, several proposals have considered specific sets of temporal operators~\cite{Galindo2001},~\cite{Pons2012ijcis},~\cite{Schockaert2008ieeetfs}, often based on the possible temporal relationships between two time indications~\cite{Galindo2001},~\cite{Pons2012ijcis},~\cite{Schockaert2008ieeetfs}. Such temporal relationships define semantically meaningful relationships with a temporal nature, between two time indications. In~\cite{Allen1983cacm}, a groundbreaking collection of temporal relationships between two time intervals (and as a special case instants) is introduced. Of course, to query temporal data in a fuzzy way, fuzzy variants of temporal operators are necessary and several proposals have thus introduced such variants~\cite{Galindo2001},~\cite{Pons2012ipmu}, often based on fuzzy variants of temporal relationships~\cite{Schockaert2008ieeetfs},~\cite{Pons2013ijufkbs}.

Techniques for the regular or fuzzy querying of valid-time databases containing uncertain valid-time data are considered by several existing proposals~\cite{Pons2012ijcis},~\cite{Pons2012ipmu},~\cite{Pons2013ijufkbs}. However, to the knowledge of the authors, few proposals have considered the bipolar querying of valid-time databases~\cite{Billiet2011fqas} and even fewer the bipolar querying of valid-time databases containing temporal data subject to uncertainty. This paper presents a novel technique to query a valid-time relation containing uncertain valid-time data in a bipolar way, which allows the query to have a single bipolar temporal query constraint. This paper is structured as follows: in section \ref{sec:preliminaries}, some necessary preliminary concepts and techniques are described, in section \ref{sec:proposal} the novel technique which is the contribution of the work presented in this paper is presented and in section \ref{sec:conclusions}, the conclusions of this paper and some directions for future research are given.


\section{Preliminaries\label{sec:preliminaries}}

\subsection{Time in Relational Databases}
TODO: explain everything about time in relational databases (categories of time indications, mostly, and such basic things)

\subsection{Valid-time Relations}
TODO: 'formally' explain what we will call a relation
TODO: 'formally' explain what we will cal a valid-time relation

\subsection{Uncertainty in Valid-time Intervals}
TODO: explain all theory for IKTI
TODO: explain that we will require the poss. dist. to be convex and what this means

\subsection{Evaluation of Temporal Relationships}
TODO: explain how temporal relationships boil down to combinations of Allen relationships and what the crisp Allen relationships are
TODO: explain all theory for IKC
TODO: explain 

\subsection{Bipolar Querying}

\section{A Novel Querying Approach\label{sec:proposal}}

\subsection{Valid-time Relations Subject to Uncertainty}
TODO: mention that we will consider valid-time relations subject to uncertainty and how they are composed (refer to earlier work)
TODO: present the example table

\subsection{Querying Using Bipolar Valid-time Constraints}
TODO: present the used query structure

\subsection{Evaluation}
TODO: explain how temporal query evaluation requires the use of Allen Relationships, so in this case a poss. equivalent has to be found
TODO: explain how this equivalent is provided by the IKC framework.
TODO: present the used evaluation technique

\subsection{Aggregation and Ranking}
TODO: figure correct interpretation and subsequent aggregation out


\section{Conclusions and Future Work\label{sec:conclusions}}


\bibliographystyle{splncs}
\bibliography{sources}

\end{document}
