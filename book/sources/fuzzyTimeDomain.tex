%%%%%%%%%%%%%%%%%%%%%%%%%%%%%%%%%%%%%%%%%%%%%%%%%%%%%%%%%%%%%%%%%%%%%%%%%%%
%
% Fuzzy time domain
%
%%%%%%%%%%%%%%%%%%%%%%%%%%%%%%%%%%%%%%%%%%%%%%%%%%%%%%%%%%%%%%%%%%%%%%%%%%%

%INTRO
As explained in the introduction, humans handle temporal information using temporal notions like time intervals or time points \cite{Dyreson1994}. While the used temporal notions may contain imperfections~\cite{Dev98},~\cite{Dubois:jucs_9_9:fuzziness_and_uncertainty_in},~\cite{nagypal2003},~\cite{Dubois89}, humans often gracefully deal with these, as their inherent interpretation capability accounts for a lot of them. This phenomenon has been studied a.o. in the field of artificial intelligence~\cite{Tre97},~\cite{5151} and language understanding~\cite{DeCaluwe:1997:FTI:285506.285516},~\cite{nagypal2003},~\cite{Dev98}. An information system, however, cannot appeal to a similar interpretation functionality. Thus, many proposals have been concerned with the combination of time and imperfections in the context of information systems~\cite{nagypal2003}. In this section, some main concepts and issues concerning this combination are presented.

\subsection{Types of Imperfections in Temporal Modelling}
Generally, in temporal modelling, a distinction is made between the following types of imperfections~\cite{nagypal2003}.

\begin{itemize}
	\item \emph{Uncertainty}: Temporal information or data may contain uncertainty. This means that the exact temporal value is (partially) unknown, however, generally some knowledge is present anyhow, possibly describing the value~\cite{Dubois:jucs_9_9:fuzziness_and_uncertainty_in},~\cite{nagypal2003},~\cite{Dubois89},~\cite{343607}. E.g., the temporal notion described in a sentence like `The Belfry of Bruges was finished on a day somewhere between 01/01/1201 A.D. and 31/12/1300 A.D.' contains uncertainty: it is known that the belfry of Bruges was finished on a single day and that this day lays somewhere between 01/01/1201 A.D. and 31/12/1300 A.D., but it is not known exactly which day it is.
	\item \emph{Vagueness}: Temporal information or data might contain inherent vagueness, as a precise instant or time interval may be intended, but the description of it is certainly vague~\cite{schockaert08},~\cite{nagypal2003},~\cite{Dev98}. E.g., the temporal notion described in a sentence like `It happened during summer.' contains vagueness, as even the boundaries of the mentioned temporal notion are not clearly expressed.
	\item \emph{Subjectivity or ambiguity}: Temporal notions might be subject to subjectivity or ambiguity. In certain cases, the temporal notion concerns a historical period like `late romanticism' or `the early middle ages' and thus contains subjectivity~\cite{nagypal2003}. In other cases, the interpretation of the temporal notion depends on extra factors. E.g., consider a person saying to another person `Let's meet each other at six.' The person hearing these words doesn't now if 6 a.m. or 6 p.m. is intended, though the person saying the words does.
\end{itemize}

As to the sources of the imperfections in temporal information, most proposals consider no specific source~\cite{5151},~\cite{schockaert08},~\cite{nagypal2003},~\cite{Dubois89},~\cite{343607},~\cite{ohlbach2004},~\cite{Virant199639}. Some proposals, however, deal with the imperfections specifically resulting from aspects of language~\cite{Dev98} and other proposals consider transitions between different granularities to be the source of imperfection in temporal information~\cite{Lin97}. Therefore, some proposals consider granularity as the base of the temporal model~\cite{Cru97}.


\subsection{Imperfections in Temporal Relationships}
As the existence of temporal relationships allows to compare temporal notions, many approaches have been concerned with finding similar temporal relationships, able to support imperfections in the temporal information which is described by temporal notions or even by the temporal relationships themselves~\cite{ohlbach2004},~\cite{nagypal2003},~\cite{schockaert08},~\cite{Dubois:jucs_9_9:fuzziness_and_uncertainty_in}. These approaches are often based on Allen's operators~\cite{Allen83}.

%%%TODO

%Humans usually deals with time in a imprecise way. e.g. `A few days ago we received this package'. This issue has been studied in the field of Artificial Intelligence ~\cite{Tre97} and language understanding~\cite{DeCaluwe:1997:FTI:285506.285516}. Some proposals ~\cite{knight1993},~\cite{Cru97},~\cite{nagypal2003} conclude that the best representation for incomplete temporal knowledge are time intervals instead of time points. This means that, as Allen proprosed in \cite{Allen83} the primitive units (the chronons) in the temporal system are intervals.

%When modelling temporal knowledge, the following types of information might be found~\cite{nagypal2003}:

%\begin{description}
%\item[\emph{Uncertainty.}] The temporal specification of the event is uncertain. This is usually found in historical documents when some documents state contradictory facts about some event.


%\item[\emph{Vagueness.}] Some of the temporal specifications are defined in a vague way. e.g. `during summer', `late night'. Note also that temporal specifications become vague when the granularity is changed ~\cite{Cru97},~\cite{nagypal2003} . e.g. `The book was written by July 2012' is usefull if we are interested in the month and the year but it turns into a incomplete expression when we try to obtain the exact day. It is possible to set the following classification~\cite{Dev98}:

%\begin{itemize}
%\item
%Non-numerical indications. e.g. \emph{often, usually}.
%\item
%Approximative time indications. e.g. \emph{around 10 a.m.}
%\item
%Indications on temporal relations. e.g. \emph{after 12, during summer}.
%\end{itemize}


%\item[\emph{Subjectivity} or \emph{Ambiguity}.] The specification of a temporal fact may be affected by subjectivity.  It is possible to distinguish between:
%\begin{itemize}
%\item Historical periods~\cite{nagypal2003} like `late romanticism' or `early middle ages'. 
%\item The interpretation of the temporal event~\cite{Dev98} depends on the side of the hearer. e.g. A speaker says `We will meet each other at six', the hearer has to choose between 6 p.m. or 6 a.m. The speaker, however knows exactly which one is meant. 
%\end{itemize}
%\end{description}




%Another important task in the modelling of incomplete knowledge is when dealing with transitions among the different granularities.







%%%%%%%%%%%%%%%%%%%%%%%%%%%%%%%%%%%%%%%%%%%%%%%%%%%%%%%%%%%%%%%%%%%%%%%%%%%
%
% End fuzzy time domain
%
%%%%%%%%%%%%%%%%%%%%%%%%%%%%%%%%%%%%%%%%%%%%%%%%%%%%%%%%%%%%%%%%%%%%%%%%%%%%%