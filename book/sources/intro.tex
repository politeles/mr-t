%%%%%%%%%%%%%%%%%%%%%%%%%%%%%%%%%%%%%%%%%%%%%%%%%%%%%%%%%%%%%%%%%%%%%%
%
% Introduction
%
%%%%%%%%%%%%%%%%%%%%%%%%%%%%%%%%%%%%%%%%%%%%%%%%%%%%%%%%%%%%%%%%%%%%%%

The concept of time itself is very complex to handle and interpret~\cite{klein94},~\cite{Shackle61}, though it is very natural and omnipresent. As information systems often attempt the modelling or the representation of natural objects, concepts or processes, they often require modelling or representing temporal information. Thus, several proposals have been concerned with the obtaining of theoretical models that allow the modelling or representation of time~\cite{Bolour82},~\cite{Cru97}.

A very specific type of information systems are database systems, which are computer systems designed to manage databases. A database contains data representing real objects or concepts. Every one of these data is a measurement or description of a property of a real object or concept. In reality, some aspects or properties of objects or concepts are time-variant or time-related. E.g. the moment of a bank transaction is traditionally a moment in time and thus a time-related notion, the function of an employee in a company can change through recorded history and is thus time-variant. A temporal database schema is a database schema that models real objects or concepts with time-related or time-variant properties. However, the modelling of temporal aspects has a direct impact on the consistency of the temporal database, because the temporal nature of these aspects imposes extra integrity constraints. An example. Consider a relation in a relational library database, modelling the presence of books in the library. Every physical book is represented by a unique identifier. Every record in the relation contains such an identifier, a date on which the corresponding book was loaned and a date on which it was subsequently returned (if it was returned). Without further precautions, a library employee could add several records with the same book identifier, different `loaned'-dates and no `returned'-dates. This would represent the same physical book being loaned several times on different dates and never returned, which is of course impossible. A temporal database model will typically constrain record insertion and prevent similar modelling inconsistencies.

A lot of research concerns these temporal database models and their approaches to the modelling of time. The first efforts were towards the representation of historical information related to objects represented by records in a database~\cite{Clifford:1985:AHR:971699.318922}. Some proposals tried to extend the Entity Relationship Model~\cite{Klopprogge:1983}, without impact on any database standards like SQL~\cite{Sarda:1990:ESH:627277.627409}.

Notably, in 1994, ``A Consensus Glossary of Temporal Database Concepts'' was published~\cite{Dyreson1994}. For this publication, 44 temporal database researchers, among which some of the main researchers in this field, cooperated to reach a consensus on the nature and definitions of some of the main temporal database concepts and terminology. This glossary was subsequently updated in 1998~\cite{Jensen:Dyreson:1998:TheConsensusGlossary}.

An interesting issue in temporal modelling concerns relationships between temporal notions. Notably, Allen~\cite{Allen83} studied temporal relationships between time intervals (and as a special case time points). Among others, the querying of temporal databases has greatly profited from these temporal relationships, because they allowed for richer and complexer user-specified temporal query demands, by allowing to express more complex relationships between the temporal notions in the temporal expressions in the query and the temporal indications in the database.

Humans handle temporal information using certain temporal notions like time intervals or time points~\cite{Dyreson1994}. Humans are often not hindered by imperfections like imprecisions, vaguenesses, uncertainties or inconsistencies possibly contained in the descriptions of these temporal notions. Among many others, these possible imperfections in descriptions of temporal notions determine an important issue in temporal modelling. E.g. the temporal notion in a sentence like `The Belfry of Bruges was finished on a day somewhere between 1/01/1201 A.D. and 31/12/1300 A.D.' contains imperfection because of the uncertainty in the used time-related expression. It is known that the building was finished on a single day, but it is not known precisely which day this was.

To allow information systems to cope with these and similar imperfections, many approaches adopt fuzzy sets~\cite{Zadeh65} for the representation of temporal information~\cite{343607},~\cite{nagypal2003},~\cite{Billiet:Pons:Matthe:DeTre:Pons:2011:BipolarFuzzy},~\cite{Dubois:jucs_9_9:fuzziness_and_uncertainty_in}. The temporal relationships studied by Allen were fuzzified by several authors~\cite{ohlbach2004},~\cite{nagypal2003},~\cite{schockaert08}. Garrido et al. ~\cite{garrido2009} present different temporal operators, defined by a combination of regular fuzzy comparisons. Both~\cite{garrido2009} and~\cite{Pon11} deal with uncertainty in temporal expressions concerning time intervals. Other approaches, like~\cite{Qia09}, use rough sets~\cite{Pawlak1995} to represent time intervals.

Next to temporal modelling, some attention has been spent on temporal reasoning~\cite{Allen83}. Although temporal reasoning is not discussed in this chapter, it should be noted that, among others, Dubois and Prade~\cite{Dubois:jucs_9_9:fuzziness_and_uncertainty_in},~\cite{Dubois89} have dealt with fuzziness and uncertainty in temporal reasoning.

The rest of this chapter is structured as follows. Section \ref{sec:time-domain} presents some basic concepts and terminology about temporal modelling and discusses some of its important aspects and issues. In section \ref{sec:temporal-databases}, some basic concepts and terminology about temporal databases are presented, followed by an overview of some interesting issues concerning temporal databases and a survey of some commercial temporal database systems. \textcolor{red}{TODO: REFERENCE SECTION 4!}

%The concept of \emph{time} is easy to understand but very complex to define~\cite{klein94}~\cite{Shackle61}. 
%In Information Systems \emph{IS}, several proposals have been concerned with the obtaining of theoretical models that allow the representation of time~\cite{Bolour82}~\cite{Cru97}. 
%More towards temporal databases, a lot of work have been done. The first efforts were towards the representation of the historical information related to an object in the database(reference). Some works tried to extend the Entity Relationship Model \emph{ERM} ~\cite{Klopprogge:1983} but without an impact on any database standards like SQL~\cite{Sarda:1990:ESH:627277.627409}.

%The ``Consensus Glossary of Temporal Database concepts'' \cite{Dyreson1994} is the first publication where the main researchers in temporal databases define the main thesaurus of temporal database concepts. This glossary defines the main types of time in a temporal database and is the basis of furthers researches ~\cite{Sarda:1990:ESH:627277.627409} \cite{Jensen94thetsql2}.

%Another main issue in temporal databases is the querying. The user specifies in the query some temporal constraints, e.g. \emph{``before this year''},\emph{``after 1995''}. In order to solve these temporal comparisons, between the time specified in the query and the temporal data stored in the database, some basic relationships have been defined. Allen~\cite{Allen83} first introduced the temporal relations between time intervals and, to a lesser extent, to time points. Some authors~\cite{ohlbach2004},\cite{nagypal2003},\cite{schockaert08} have softened the temporal operators defined by Allen. Therefore it is possible to compute the relationships between two ill-known time points or intervals. In~\cite{garrido2009}, some different temporal operators are defined by a combination of regular fuzzy comparisons. %e.g. the \emph{BEFORE} operator is defined by means of the fuzzy less than (FLT) operator. 

%To allow information systems to cope with uncertainty, many approaches adopt fuzzy sets~\cite{Zadeh65} for the representation of temporal information~\cite{Billiet:Pons:Matthe:DeTre:Pons:2011:BipolarFuzzy},\cite{Dubois:jucs_9_9:fuzziness_and_uncertainty_in},~\cite{devos94}. The time points representing the time when the object is valid in the reality might not be precisely known. In order to deal with this, several models have been proposed. Garrido~\cite{garrido2009} proposes a model to deal with uncertainty in the time interval by means of fuzzy intervals. Bronselaer~\cite{Pon11} proposes a consistent framework to deal and represent ill-known time values and their relationships. Rough sets~\cite{Pawlak1995} have been also used to represent time intervals~\cite{Qia09} in databases.

%temporal reasoning and fuzzy temporal reasoning.
%Next to the manage of temporal information is the temporal reasoning~\cite{Allen83}. Dubois and Prade ~\cite{Dubois:jucs_9_9:fuzziness_and_uncertainty_in},\cite{Dubois89} deal with fuzziness and uncertainty in temporal reasoning, but this topic is outside the scope of this work.

%temporal commercial systems.

%The work is organized as follows: 

%In Section \ref{sec:time-domain} an study about the time and its properties is done. Section \ref{sec:temporal-databases} is an overview of the main problems when managing the time in a database, the different proposals for solving them and the temporal database proposals. The section concludes with a comparison among the different proposals. In Section \ref{sec:fuzzy-temporal-databases} we present a sum up with the proposals for handling the imprecision in temporal databases and the main deficiencies we have detected. Finally, there are several commercial temporal database management systems  \emph{TDMBS} like \cite{oracle2009},\cite{posgree2009},\cite{teradata2011},\cite{timedb2005}. All of them are analyzed and compared later on this work. 
