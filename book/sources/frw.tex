In this chapter, some of the main concepts concerning temporal modelling and temporal modelling in information systems and the terminology corresponding with these concepts are introduced and explained and some of the main properties of and issues with these concepts are presented and discussed. An overview of some commercial temporal DBMS is briefly introduced. Finally, a novel technique for querying valid-time relations using imperfect query specifications is presented.


 %we have presented the main properties of the time in an information system. An overview of the commercial temporal DBMS has also been introduced. Although, to the best of our knowledge, no commercial fuzzy temporal DBMS have been found.  Finally the managing of imprecision in temporal databases is explained. A brief introduction of bipolar querying in temporal databases is done at the end of the section. 

Further research work could follow several general directions. First of all, a theoretical model for dealing with uncertainty in both the database and the query at the same time could be researched and defined. Next, implementations including both DDL and DML could be proposed and constructed.

%Further research work should be done in several ways: first of all, a theoretical model for both dealing with uncertainty in the database as well as in the querying should be defined. Then, another interesting work is to propose an implementation with both DDL (Data Definition Language) and DML (Data Manipulation Language).