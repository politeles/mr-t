In this work we have presented the main properties of the time in an information system. An overview of the commercial temporal DBMS has also been introduced. Although, to the best of our knowledge, no commercial fuzzy temporal DBMS have been found.  Finally the managing of imprecision in temporal databases is explained. A brief introduction of bipolar querying in temporal databases is done at the end of the section. 

Further research work should be done in several ways: first of all, a theoretical model for both dealing with uncertainty in the database as well as in the querying should be defined. Then, another interesting work is to propose an implementation with both DDL (Data Definition Language) and DML (Data Manipulation Language).