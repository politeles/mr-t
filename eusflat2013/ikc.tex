In~\cite{Pons2011}, the ill-known constraint framework is introduced. In this section, its main aspects concerning interval representation and Allen relationship evaluation are briefly presented.

\subsection{\label{subsec:ikc-preliminaries}Preliminary Concepts}
The IKC framework relies on the concepts of possibilistic variables, ill-known values and ill-known intervals~\cite{JoseEnriquePons2012},~\cite{Billiet2012},~\cite{Dubois1983},~\cite{Pons2011}. These are introduced below.

\begin{definition}
\label{def:poss-variable}
A \emph{possibilistic variable} $X$ on a universe $U$ is defined as a variable taking exactly one value in $U$, but for which this value is (partially) unknown. The variable $X$'s possibility distribution $\pi_X$ on $U$ models the available knowledge about the value that $X$ takes: for each $u \in U$, $\pi_X(u)$ represents the possibility that $X$ takes the value $u$.
\end{definition}

As mentioned above, possibility is interpreted as plausibility in the presented work and thus the possibility that possibilistic variable $X$ on $U$ takes the value $u \in U$ is interpreted as a measure of how plausible it is that $X$ takes the value $u$, given (partial) knowledge about the value $X$ takes.

Now consider a set $U$ containing single values (and not collections of values). When a possibilistic variable $X_{v}$ is defined on such a set $U$, the unique value $X_{v}$ takes, which is (partially) unknown, will be a single value in $U$ and is called an \emph{ill-known value} (IKV) in $U$ in this work~\cite{Billiet2012},~\cite{Pons2011},~\cite{Dubois1988}.

In this paper, possibilistic variables will be denoted using uppercase letters and IKV using lower-case letters.

\subsection{\label{subsec:ikc-interval}Representation of Time Intervals Subject to Uncertainty}
The IKC framework allows uncertainty in intervals by supporting a kind of ill-known intervals~\cite{JoseEnriquePons2012},~\cite{Billiet2012},~\cite{Dubois1983},~\cite{Pons2011}:

\begin{definition}
Consider an ordered set $U$. An \emph{ill-known interval} (IKI) in $U$ is an interval in $U$ of which both boundary values are ill-known values in $U$.
\end{definition}

Specifically with regards to time, an IKI in an ordered set of instants representing time is called an \emph{ill-known time interval} (IKTI) in the presented work. In this paper, an IKTI with starting instant $s$ and ending instant $e$ will be noted $\left[s, e\right]$.

\subsection{\label{subsec:ikc-evaluation}Evaluation of Allen Relationships}
At the base of the framework's approach to evaluating Allen relationships are ill-known constraints~\cite{Pons2011}:

\begin{definition}
Given an ordered set $U$, an \emph{ill-known constraint} (IKC) $C$ on $U$ is specified by means of a binary relation $R \subseteq U^{2}$ and a fixed IKV $v$ in $U$ defined by its possibilistic variable $V$ on $U$, i.e.:
\begin{align}
C \triangleq (R,v) \nonumber
\end{align}
Any set $A \subseteq U$ now satisfies IKC $C = (R,v)$ if and only if:
\begin{align}
\forall a \in A : (a,v) \in R \nonumber
\end{align}
\end{definition}

The satisfaction of an IKC $C$ by a set $A$ will be noted $C(A)$ in this paper. 

Consider an ordered set $U$, an IKC $C \triangleq (R,v)$ on $U$ and a set $A \subseteq U$. Due to the uncertainty about the exact value of $v$, it is uncertain whether $A$ satisfies $C$ or not. Confidence about the plausibility that $A$ satisfies $C$ can be expressed using a possibility degree $\Pos(C(A))$ and a necessity degree $\Nec(C(A))$, which are shown to be calculated as follows:

\begin{align}
\Pos(C(A)) & = \min_{a \in A}\left(\sup_{(a,w) \in R}\pi_{V}(w)\right) \label{ill-known-pos}\\
\Nec(C(A)) & = \min_{a \in A}\left(\inf_{(a,w) \notin R} 1-\pi_{V}(w)\right) \label{ill-known-nec}
\end{align}

Given an ordered set $U$, possibility and necessity degrees expressing confidence about the plausibility that a set $A \subseteq U$ satisfies a boolean aggregation of IKC on $U$ can be found by using the possibilistic extensions of boolean operators `and' ($\wedge$), `or' ($\vee$) and `not' ($\neg$) and can be calculated using the following expressions:

\begin{align}
\Pos\left(\left(C_1(A)\right) \wedge \ldots \wedge \left(C_n(A)\right)\right) = \min_{i=1 \ldots n} \left(\Pos\left(C_i(A)\right)\right) \nonumber \\
\Nec\left(\left(C_1(A)\right) \wedge \ldots \wedge \left(C_n(A)\right)\right) = \min_{i=1 \ldots n} \left(\Nec\left(C_i(A)\right)\right) \nonumber \\
\Pos\left(\left(C_1(A)\right) \vee \ldots \vee \left(C_n(A)\right)\right) = \max_{i=1 \ldots n} \left(\Pos\left(C_i(A)\right)\right) \nonumber \\
\Nec\left(\left(C_1(A)\right) \vee \ldots \vee \left(C_n(A)\right)\right) = \max_{i=1 \ldots n} \left(\Nec\left(C_i(A)\right)\right) \nonumber \\
\Pos\left(\neg C(A)\right) = 1 - \Nec\left(C\left(A\right)\right) \nonumber \\
\Nec\left(\neg C(A)\right) = 1 - \Pos\left(C\left(A\right)\right) \nonumber
\end{align}

Here, $C$ and all $C_i$ for which $i \in \mathbb{N}_0$ and $1 \leq i \leq n$ denote IKC on $U$.

The IKC framework now allows evaluating a given Allen relationship $AR$ between a given CTI $I$ and a given IKTI $J = \left[s, e\right]$ by allowing to calculate the possibility and necessity that $I$ $AR$ $J$ holds. For this, the combination of $AR$ and $J$ is translated to a specific boolean aggregation of specific IKC. These translations are shown in table \ref{tab:allen-relations}. Every row of this table corresponds to an Allen relationship. The Allen relationships the rows correspond to, are shown in the `Allen Relationship' column, the collections of specific IKC for given Allen relationships are shown in the `Constraints' column (every $C_i, i \in \{1, 2, 3, 4\}$ denotes an IKC) and the specific aggregation of these IKC used for evaluation of the Allen relationships are shown in column `Aggregation'. Finally, the possibility and necessity degrees expressing confidence about the plausibility that $I$ $AR$ $J$ holds are then the possibility and necessity expressing confidence about the plausibility that $I$ satisfies the specific aggregation of specific IKC found as translation of the combination of $AR$ and $J$.


Using the formulas shown above, the requested possibility and necessity degrees can be calculated from these.

\begin{table}[h]
\centering
\begin{tabular}{|c|c|c|}
\hline
Allen Relationship & Constraints & Aggregation \\
\hline
I before J & $C_1\stackrel{\triangle}{=} \left(<,s\right)$ & $C_1(I)$ \\
\hline
\multirow{4}{*}
{I equal J} & $C_1\stackrel{\triangle}{=} \left(\geq,s\right)$ & $C_1(I)\wedge$\\
 & $C_2\stackrel{\triangle}{=} \left(\neq,s\right)$ & $\neg C_2(I)\wedge$\\
 & $C_3\stackrel{\triangle}{=} \left(\leq,e\right)$ &  $C_3(I)\wedge$ \\
 & $C_4\stackrel{\triangle}{=} \left(\neq,e\right)$ & $\neg C_4(I)$\\
\hline
\multirow{2}{*}
{I meets J} & $C_1\stackrel{\triangle}{=} \left(\leq,s\right)$ & $C_1(I)\wedge$\\
 & $C_2\stackrel{\triangle}{=} \left(\neq,s\right)$ & $\neg C_2(I)$\\
\hline
\multirow{3}{*}
{I overlaps J} & $C_1\stackrel{\triangle}{=} \left(<,e\right)$ & $C_1(I)\wedge$\\
 & $C_2\stackrel{\triangle}{=} \left(\leq,s\right)$ & $\neg C_2(I)\wedge$\\
 & $C_3\stackrel{\triangle}{=} \left(\geq,s\right)$ & $\neg C_3(I)$\\
\hline
\multirow{4}{*}
{I during J} & $C_1\stackrel{\triangle}{=} \left(>,s\right)$ & $\big(C_1(I)\wedge$\\
 & $C_2\stackrel{\triangle}{=} \left(\leq,e\right)$ & $ C_2(I)\big)\vee$ \\
 & $C_3\stackrel{\triangle}{=} \left(\geq,s\right)$ & $\big(C_3(I)\wedge$ \\
 & $C_4\stackrel{\triangle}{=} \left(<,e\right)$ & $C_4(I)\big)$\\
\hline
\multirow{2}{*}
{I starts J} & $C_1\stackrel{\triangle}{=} \left(\geq,s\right)$ & $C_1(I)\wedge$\\
 &  $C_2\stackrel{\triangle}{=} \left(\neq,s\right)$ & $\neg C_2(I)$\\
\hline
\multirow{2}{*}
{I finishes J} & $C_1\stackrel{\triangle}{=} \left(\leq,e\right)$ & $C_1(I)\wedge$\\
 & $C_2\stackrel{\triangle}{=} \left(\neq,e\right)$ & $\neg C_2(I)$\\
\hline
\end{tabular}
\caption{The translations of Allen relationships to the IKC framework.}
\label{tab:allen-relations}
\end{table}