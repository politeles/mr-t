%2. General Preliminaries
As time is often thought of as following an axis (a time axis), it is often (as in the presented work) seen as an ordered set of time points, called \emph{instants}~\cite{Dyreson1994},~\cite{Jensen1998}. In this paper, instants will be noted using lowercase letters, whereas time intervals will be noted using uppercase letters. A \emph{time interval} $I$ is then an interval subset of this ordered set of instants. In this paper, such a time interval starting at (and containing) instant $s$ and ending at (and containing) instant $e$ is noted $I = \left[s, e\right]$.

The presented work allows time intervals to be subject to uncertainties. These uncertainties are always seen as caused by a (partial) lack of knowledge: the exact time interval is not known, eventhough there is only one correct time interval and as such no variability. Confidence about which interval is the correct one in the context of such uncertainties is modelled using possibility theory~\cite{DidierDubois1988a}. Possibility is always interpreted as plausibility given a (partial) lack of knowledge in the presented work.

To be able to distinguish easily, time intervals not subject to uncertainties will be called \emph{crisp time intervals} in the presented work.