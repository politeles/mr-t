In this paper, a comparison is presented between two different frameworks designed to represent time intervals subject to uncertainty and evaluate temporal relationships between such intervals and crisp intervals: the triangular model (TM) framework and the ill-known constraint (IKC) framework. It is concluded that:

\begin{itemize}
	\item With respect to representation, both frameworks differ only slightly, with the TM framework allowing easier human assessments, due to its approach including visualization.
	\item With respect to temporal relationship evaluation, the TM framework allows easy human assessments in several situations, but the IKC framework seems more fitted for complex reasoning, due to its modular build.
\end{itemize}

Future research might deal with the evaluation of temporal relationships between several time intervals subject to uncertainty.

TODO: MORE FUTURE WORK


%In this work we have presented an analysis between two different frameworks that deal and represent imperfect temporal information. Besides the differences in the visualization of imperfect time intervals, we obtained that the temporal reasoning in both frameworks is equivalent. It is also important to notice that whereas the ill-known constraint framework deal with the thirteen Allen's relations, the triangular model distinguishes fifteen possible relational zones and therefore, it is necessary to set a correspondence between the Allen's relations in both frameworks. 

%Future research will deal with the analysis between two imperfect time points in each framework and with the representation of open time intervals. 