In this section, some basic concepts are introduced concerning possibilistic variables and fuzzy numbers and intervals, after which the framework of set evaluation by ill-known constraints \cite{Pons2011} is explained. The section concludes with a brief introduction to temporal databases.


\subsection{\label{subsec:possibilistic-variables}Possibilistic Variables}
Possibilistic variables rely on possibility theory \cite{Dubois1988a}. A \emph{possibilistic variable} is defined as follows \cite{Pons2011}.

\begin{definition}
A possibilistic variable $X$ over a universe $U$ is defined as a variable taking exactly one value in $U$, but for which this value is (partially) unknown. Its possibility distribution $\pi_X$ on $U$ models the available knowledge about the value that $X$ takes: for each $u\in U$, $\pi_X(u)$ represents the possibility that $X$ takes the value $u$. In this work, this possibility is interpreted as a measure of how plausible it is that $X$ takes the value $u$, given (partial) knowledge about the value $X$ takes.
\end{definition}

The exact value a possibilistic variable takes, which is (partially) unknown, is called an \emph{ill-known value} in this work \cite{Dubois1988a}.

When a possibilistic variable is defined on the powerset $\Pow(R)$ of some universe $R$, the unique value the variable takes will be a crisp set and its possibility distribution on the powerset $\Pow(R)$ will describe the possibility of each crisp subset of $R$ to be the value the variable takes. This exact value (a crisp set) the variable takes, is now called an \emph{ill-known set} \cite{Dubois1988a}.


A specific application of possibilistic variables is obtained when the universe under consideration is the set of Boolean values $\mathbb{B}$ = $\{T,F\}$. Indeed, any Boolean proposition $p$ takes just one value in $\mathbb{B}$. If the knowledge about which value this proposition $p$ will take is given by a possibility distribution $\pi_p$, the proposition can be seen as a possibilistic variable. As the interest lies with the case where the proposition holds, the possibility and necessity that $p$ = $T$ (the proposition holds) demand most attention. This possibility and necessity is noted here as:
\begin{align}
\text{Possibility that $p$ = $T$ (p holds):} \hspace{50pt} & Pos(p) = \pi_p(T) \label{propholdsposs} \\
\text{Necessity that $p$ = $T$ (p holds):} \hspace{50pt} & Nec(p) = 1-\pi_p(F) \label{propholdsnecc}
\end{align}

This work will deal with ill-known intervals. These are ill-known sets, defined and represented via a start and end point, which will be ill-known values. The elements of the set are the values between the start and end point. A closed ill-known interval with start point defined by possibilistic variable $X$ and end point by possibilistic variable $Y$ is noted here $\left[X, Y\right]$. The correspondences and transitions between the representations of ill-known sets, between the representations of ill-known intervals and between the representations of an ill-known set and an ill-known interval are part of the authors current research.

\subsection{\label{subsec:fuzzy-numbers}Fuzzy Numbers and Fuzzy Intervals}
Among others, Dubois and Prade~\cite{Dubois1983} use fuzzy sets \cite{Zadeh1965} to define a \emph{fuzzy interval}:
\begin{definition}
A fuzzy interval is a fuzzy set $M$, defined by a membership function $\mu_{M}$, on the set of real numbers $\mathbb{R}$ such that:
\begin{eqnarray}
\mu_{M} : & \!\!\!\!\!\!\!\!\!\!\!\!\!\!\!\!\!\!\!\!\!\!\!\!\!\!\!\!\!\!\!\!\!\!\!\!\!\!\!\!\!\!\!\!\!\!\!\!\!\! \mathbb{R} \rightarrow \left[0,1\right] \nonumber \\ 
\forall (u,v)\in\mathbb{R}^2: \forall w \in [u,v]:&\mu_M(w) \geq\min(\mu_M(u),\mu_M(v))  \\
\exists m \in \mathbb{R} : & \!\!\!\!\!\!\!\!\!\!\!\!\!\!\!\!\!\!\!\!\!\!\!\!\!\!\!\!\!\!\!\!\!\!\!\!\!\!\!\!\!\!\!\!\!\!\!\! \mu_M(m)=1 
\end{eqnarray}
\end{definition}
If this modal value $m$ is unique, then $M$ is referred to as a \emph{fuzzy number}. In other words, if the core of a fuzzy interval is a singleton, it is referred to as a fuzzy number.

The most convenient form of the membership function of a fuzzy number is a triangular form. It can be shown that such a membership function $\mu_M$ for a fuzzy number $M$ is convex and normalized. Three real values, denoted by $a$, $b$ and $D$, suffice to represent a triangular membership function of a fuzzy number and in this work, a fuzzy number defined as such will be noted as $\left[D, a, b \right]$. Here:
\begin{itemize}
\item
$D$ denotes the single value in the core of $M$
\item
$D-a$ is then $\inf \{u \in \mathbb{R} : \mu_{M}(u) > 0\}$
\item
$D+b$ is then $\sup \{u \in \mathbb{R} : \mu_{M}(u) > 0\}$
\end{itemize}

\subsection{Interval Evaluation by Ill-known Constraints}
The problem of interval evaluation is more generally explained in \cite{Pons2011}: the need exists to know if all points in a crisp interval $I$ reside between the boundaries of an ill-known interval $\left[ X , Y \right]$. In \cite{Pons2011}, the notion of an \emph{ill-known constraint} is introduced:

\begin{definition}
Given a universe $U$, an ill-known constraint $C$ on a set $A \subseteq U$ is specified by means of a binary relation $R \subseteq U^{2}$ and a fixed, ill-known value denoted by its possibilistic variable $V$ over $U$, i.e.:
\begin{align}
C \triangleq (R,V)
\end{align}
Set $A$ now satisfies the constraint if and only if:
\begin{align}
\forall a \in A : (a,V) \in R
\end{align}
\end{definition}

An example of an ill-known constraint is $C_{ex} \triangleq (<, X)$. Some set $A$ then satisfies $C_{ex}$ if $\forall a \in A : a < X$, given possibilistic variable $X$.

The satisfaction of a constraint $C \triangleq (R,V)$ by a set $A$ is basically still a Boolean matter, but due to the uncertainty about the ill-known value $V$, it can be uncertain whether $C$ is satisfied by $A$ or not \cite{Pons2011}. In fact, this satisfaction now behaves as a proposition. Based on the possibility distribution $\pi_{V}$ of $V$, the possibility and necessity that $A$ satisfies $C$ can be found. This proposition can thus be seen as a possibilistic variable on $\mathbb{B}$. The required possibility and necessity are:

\vspace{-10pt}

\begin{align}
\Pos(A\text{ satisfies }C) & = \min_{a \in A}\left(\sup_{(a,w) \in R}\pi_{V}(w)\right) \label{ill-known-pos}\\
\Nec(A\text{ satisfies }C) & = \min_{a \in A}\left(\inf_{(a,w) \notin R} 1-\pi_{V}(w)\right) \label{ill-known-nec}
\end{align}

Now, e.g., to check if crisp interval $I = \left[j, k\right]$ is included in $\left[X, Y\right]$, 2 ill-known constraints are constructed:

\vspace{-10pt}

\begin{eqnarray}
C_1 & \triangleq\left(\geq,X\right)\\
C_2 & \triangleq\left(\leq,Y\right)
\end{eqnarray}

To calculate the possibility and necessity concerning a conjunction of constraints, the $\min$ operator can be used. The possibility and necessity of $I$ being included in $\left[X, Y\right]$ are now: 

\vspace{-10pt}

\begin{align}
\label{eq:interval-pos}
\Pos(I\text{ satisfies }C_1\ and\ C_2) & = \min_{a \in I}\left(\sup_{a \geq w}\pi_{X}(w),\sup_{a \leq v}\pi_{Y}(v)\right)\\
\label{eq:interval-nec}
\Nec(I\text{ satisfies }C_1\ and\ C_2) & = \min_{a \in I}\left(\inf_{a < w} 1-\pi_{X}(w),\inf_{a > v} 1-\pi_{Y}(v)\right).
\end{align}

\subsection{Temporal Databases}
A \emph{temporal database} is a database that manages some aspects of time in its schema \cite{Dyreson1994}. The reality a temporal database tries to model, contains some temporal notions which have to be handled specifically in order to maintain a consistent modelling behavior. A \emph{chronon} is the shortest duration of time supported by the database. Time can be represented either as points~\cite{DuBois1989} or intervals ~\cite{Garrido2009},\cite{Bohlen1998} that may be subject to imperfection.

The temporal notions in temporal databases can be classified into four types based on their interpretation and modelling purpose. \emph{User-defined time} has no specific impact on the database consistency, but the other types do:

\begin{itemize}
	\item
	\emph{transaction time}~\cite{Jensen1991}: time when the fact is stored in the database.
	\item
	\emph{valid time}~\cite{Snodgrass1984}: time when the fact is true in the modelled reality.
	\item
	\emph{decision time}~\cite{Nascimento1995}: time when an event was decided to happen. 
\end{itemize}
	
Database models can also be classified into \emph{bi-temporal} (both valid and transaction-time) or \emph{tri-temporal}  (bi-temporal and decision time) models. 