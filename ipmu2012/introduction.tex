Dealing with time in any information system~\cite{Bolour82} is usually a difficult task. First of all, the concept of \emph{time} itself is a complex notion. Klein~\cite{klein94} studied the concept of time in language whereas Devos~\cite{devos94} modelled vague temporal expressions by means of fuzzy sets~\cite{zadeh65}.  Several research works~\cite{cruyssen97,devos98,Cru97} have been done in obtaining theoretical models that allow the representation of time in an information system. Although the main tool for the representation of temporal information are fuzzy sets~\cite{Virant199639,35337,343607,springerlink:10.1007/978-3-540-39964-3_57,Dubois:jucs_9_9:fuzziness_and_uncertainty_in}, some other tools, like rough sets~\cite{pawlak95} have been used~\cite{Qia09}.

In the other hand, the addition of time to a relational database leads to several problems: the modelling of a time-variant object has an impact on the database consistency. Different models have been proposed \cite{Jensen94,TSQL,Sarda90,Jensen91,Snodgrass84,Nascimento95} but all of them consider the time as a crisp value. 
There exist a glossary~\cite{Dyreson1994} with the common terms in the temporal database community.
Further works introduced the concept of vagueness, imprecission in the different models~\cite{Cru97,624013,fuzz2009,gal01}. 







There are several proposals in the literature for inferring uncertainty about sets from uncertainty about values. It is of particular interest for fuzzy temporal databases, the application on temporal intervals. The transformation from two ill-known temporal points to a ill-known set has been discussed in \cite{Garrido2009}. In this work we will analyze the major pitfalls of these approaches and the benefit of using the mechanism proposed in \cite{Pon11}.

The paper is organized as follows. In Section \ref{sec:preliminaries}, some preliminaries and concepts about the possibilistic evaluation framework and the temporal databases are presented. 
Section \ref{sec:proposal} presents the possibilistic model for valid-time databases.
Finally, Section \ref{sec:conclusions} presents the main conclusions and some lines for the future work in this research.