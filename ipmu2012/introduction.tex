Dealing with time in information systems is usually a difficult task~\cite{Bolour1982}.
First of all, the concept of \emph{time} itself is very complex. More towards information systems, several proposals have been concerned with obtaining theoretical models that allow the representation of time in information systems~\cite{VanderCruyssen1997a}. Notably, temporal relations between time intervals and to a lesser extent time points, were studied by Allen~\cite{Allen1983}.

One issue in these attempts at temporal modelling is of course the inherent complexities of the concept of time. Another is the possible inherent imperfections in the description of notions of time. The description of the temporal notion in a sentence like `The Alhambra, which I can see from my room, is a few hundred years old.' contains imperfection caused by the vagueness of the used expression. The description of the temporal notion in a sentence like `The Alhambra was finished somewhere between 1301 A.D. and 1400 A.D.' contains imperfection caused by the imprecision in the used expression about the exact finishing day of the building. Though precise boundaries are given, the exact day on which the building was finished is unknown. An example of uncertainty is given by the following sentence: `I believe that the Alhambra was finished in 1300 A.D.'

The transition between temporal granularities~\cite{Lin1997a} is also considered as a source of imprecision in temporal modelling and some proposals consider granularity as the base of the temporal model~\cite{VanderCruyssen1997a}.

To allow information systems to cope with these imperfections, there are some approaches that work with probability~\cite{Dekhtyar2001},~\cite{Parisi2010} whereas some other approaches adopt fuzzy sets for the representation of temporal information~\cite{Billiet2011},\cite{Dubois2003}. The temporal relations studied by Allen were recently fuzzified by several authors~\cite{Ohlbach2004},\cite{Schockaert2008}.


Next to the issues in temporal modelling, the inclusion of time in a relational database system leads to several more practical problems, as the modelling of the temporal aspects of a time-variant object has an impact on the database consistency. Different models have been proposed \cite{Jensen1991},\cite{Snodgrass1984},\cite{Nascimento1995} that consider time as a crisp notion. Some of these models uses TSQL~\cite{Jensen1995} (a temporal extension to SQL) as query language. In some specific works, authors consider the necessity of allowing imprecision in temporal models~\cite{VanderCruyssen1997a},\cite{Garrido2009}.

Most of these models use the concepts of fuzzy numbers and fuzzy intervals to represent imprecise temporal notions and considerable attention has thus been given to the problem of transforming two fuzzy numbers which represent the boundaries of an imprecise time interval into one fuzzy interval representing this interval. There are several proposals for this transformation, but some, like those presented in~\cite{Garrido2009} are misleading. 

In this work we will propose a possibilistic valid-time model for relational databases, based on the framework presented in~\cite{Pons2011}, which deals with the inference of uncertainty about the evaluation of sets from the uncertainty about the exact values used in evaluation. 

The remainder of the paper is organized as follows. In section \ref{sec:preliminaries}, some preliminaries and concepts are presented. 
Section \ref{sec:proposal} presents the possibilistic model for valid-time databases. In this section, both representation and querying are introduced.
Finally, section \ref{sec:futher-research} presents the main conclusions and some lines for future work in this research.